
\documentclass[a4paper,UKenglish,cleveref, autoref, thm-restate]{lipics-v2021}
\usepackage{microtype}
\usepackage{amsmath,amssymb}
\usepackage{stmaryrd}
\nolinenumbers{}

\usepackage{agda}

\usepackage{dsfont}
\usepackage{newunicodechar}
\newunicodechar{λ}{\ensuremath{\mathnormal\lambda}}
\newunicodechar{σ}{\ensuremath{\mathnormal\sigma}}
\newunicodechar{τ}{\ensuremath{\mathnormal\tau}}
\newunicodechar{π}{\ensuremath{\mathnormal\pi}}
\newunicodechar{ℕ}{\ensuremath{\mathbb{N}}}
\newunicodechar{∷}{\ensuremath{::}}
\newunicodechar{≡}{\ensuremath{\equiv}}
\newunicodechar{≅}{\ensuremath{\cong}}
\newunicodechar{∀}{\ensuremath{\forall}}
\newunicodechar{ᴸ}{\ensuremath{^L}}
\newunicodechar{ᴿ}{\ensuremath{^R}}
\newunicodechar{ʳ}{\ensuremath{^r}}
\newunicodechar{ᶻ}{\ensuremath{^Z}}
\newunicodechar{ₗ}{\ensuremath{_l}}
\newunicodechar{ⱽ}{\ensuremath{^V}}
\newunicodechar{⟧}{\ensuremath{\rrbracket}}
\newunicodechar{⟦}{\ensuremath{\llbracket}}
\newunicodechar{⊤}{\ensuremath{\top}}
\newunicodechar{⊥}{\ensuremath{\bot}}
\newunicodechar{₁}{\ensuremath{_1}}
\newunicodechar{₂}{\ensuremath{_2}}
\newunicodechar{₃}{\ensuremath{_3}}
\newunicodechar{₄}{\ensuremath{_4}}
\newunicodechar{₅}{\ensuremath{_5}}
\newunicodechar{₆}{\ensuremath{_6}}
\newunicodechar{₇}{\ensuremath{_7}}
\newunicodechar{₈}{\ensuremath{_8}}
\newunicodechar{₉}{\ensuremath{_9}}
\newunicodechar{∈}{\ensuremath{\in}}
\newunicodechar{₀}{\ensuremath{_0}}
\newunicodechar{′}{\ensuremath{'}}
\newunicodechar{ˢ}{\ensuremath{^S}}
\newunicodechar{ᴬ}{\ensuremath{^A}}
\newunicodechar{∘}{\ensuremath{\circ}}
\newunicodechar{𝟙}{\ensuremath{\mathds{1}}}  
\newunicodechar{𝟘}{\ensuremath{\mathds{O}}}
% \newunicodechar{𝟙}{\ensuremath{\mathbb{I}}}  
% \newunicodechar{𝟘}{\ensuremath{\mathbb{O}}}
\newunicodechar{ᴾ}{\ensuremath{^P}}
\newunicodechar{ᵀ}{\ensuremath{^T}}
\newunicodechar{⊎}{\ensuremath{\uplus}}
\newunicodechar{ι}{\ensuremath{\iota}}
\newunicodechar{⇐}{\ensuremath{\Leftarrow}}
\newunicodechar{⇒}{\ensuremath{\Rightarrow}}
\newunicodechar{➙}{\ensuremath{\to^P}}
\newunicodechar{Δ}{\ensuremath{\Delta}}
\newunicodechar{∅}{\ensuremath{\emptyset}}
\newunicodechar{⁺}{\ensuremath{^+}}
\newunicodechar{𝕏}{\ensuremath{\mathbb{X}}}
%\newunicodechar{·}{\ensuremath{\cdot}} %seems to be defined already!
\newunicodechar{∙}{\ensuremath{\cdot}}
\newunicodechar{⁇}{\ensuremath{?}}
\newunicodechar{‼}{\ensuremath{!}}
\newunicodechar{⊕}{\ensuremath{\oplus}}
\newunicodechar{ℤ}{\ensuremath{\mathbb{Z}}}
\newunicodechar{μ}{\ensuremath{\mu}}
\newunicodechar{∃}{\ensuremath{\exists}}
\newunicodechar{;}{\ensuremath{\fatsemi}}
\newunicodechar{Σ}{\ensuremath{\Sigma}}
\newunicodechar{ᴿ}{\ensuremath{^R}}
\newunicodechar{ᵢ}{\ensuremath{_i}}
\newunicodechar{≢}{\ensuremath{\nequiv}}
\newunicodechar{≟}{\ensuremath{\stackrel{{\tiny?}}{=}}}
\newunicodechar{≤}{\ensuremath{\le}}
\newunicodechar{ᵇ}{\ensuremath{^b}}
\newunicodechar{𝓣}{\ensuremath{\mathcal{T}}}
\newunicodechar{𝓔}{\ensuremath{\mathcal{E}}}
\newunicodechar{Γ}{\ensuremath{\Gamma}}
\newunicodechar{γ}{\ensuremath{\gamma}}
\newunicodechar{⊔}{\ensuremath{\sqcup}}
\newunicodechar{α}{\ensuremath{\alpha}}
\newunicodechar{η}{\ensuremath{\eta}}
\newunicodechar{ω}{\ensuremath{\omega}}
\newunicodechar{◁}{\ensuremath{\lhd}}
%PT: seems to be already defined
%\newcommand{\lambdabar}{{\mkern0.75mu\mathchar '26\mkern -9.75mu\lambda}}
\newunicodechar{ƛ}{\ensuremath{\lambdabar}}
\newunicodechar{Λ}{\ensuremath{\mathnormal\Lambda}}
\newunicodechar{ρ}{\ensuremath{\rho}}
\newunicodechar{𝓖}{\ensuremath{\mathcal{G}}}
\newunicodechar{ℓ}{\ensuremath{\ell}}
\newunicodechar{♯}{\ensuremath{\sharp}}
\newunicodechar{⇓}{\ensuremath{\Downarrow}}
\newunicodechar{𝓥}{\ensuremath{\mathcal{V}}}
\newunicodechar{∧}{\ensuremath{\wedge}}
\newunicodechar{ₛ}{\ensuremath{_s}}
\newunicodechar{χ}{\ensuremath{\chi}}
\newunicodechar{⊨}{\ensuremath{\models}}
\newunicodechar{⦂}{\ensuremath{\mathbf{:}}}
\newunicodechar{ς}{\ensuremath{\varsigma}}
\newunicodechar{𝓓}{\ensuremath{\mathcal{D}}}
\newunicodechar{∎}{\ensuremath{\square}}
\newunicodechar{■}{\ensuremath{\blacksquare}}
\newunicodechar{↠}{\ensuremath{\twoheadrightarrow}}
\newunicodechar{↪}{\ensuremath{\hookrightarrow}}
\newunicodechar{ⁱ}{\ensuremath{^i}}
\newunicodechar{Π}{\ensuremath{\Pi}}
\newunicodechar{∋}{\ensuremath{\ni}}
\newunicodechar{∍}{\ensuremath{\ni}}
\newunicodechar{‵}{\ensuremath{`}}
\newunicodechar{⨆}{\ensuremath{\bigsqcup}}
\newunicodechar{δ}{\ensuremath{\delta}}
\newunicodechar{κ}{\ensuremath{\kappa}}
\newunicodechar{⦃}{\ensuremath{\{\!|}}  % \lBrace from stix
\newunicodechar{⦄}{\ensuremath{|\!\}}}  % \rBrace from stix
\newunicodechar{⌊}{\ensuremath{\lfloor}}
\newunicodechar{⌋}{\ensuremath{\rfloor}}
\newunicodechar{𝟎}{\ensuremath{\mathbf{0}}}
\newunicodechar{𝟏}{\ensuremath{\mathbf{1}}}
\newunicodechar{𝟐}{\ensuremath{\mathbf{2}}}
\newunicodechar{β}{\ensuremath{\beta}}
\newunicodechar{≥}{\ensuremath{\geq}}
\newunicodechar{ₒ}{\ensuremath{_o}}
\newunicodechar{∊}{\ensuremath{\in}}
\newunicodechar{;}{\ensuremath{;}}
\newunicodechar{⋯}{\ensuremath{\mkern2mu \cdotp \mkern-2mu \cdotp \mkern-2mu \cdotp \mkern1mu }}
\newunicodechar{⊢}{\ensuremath{\vdash}}
\newunicodechar{∶}{\ensuremath{:}}
\newunicodechar{★}{\ensuremath{\star}}
\newunicodechar{ᴷ}{\ensuremath{^K}}
\newunicodechar{ϕ}{\ensuremath{\varphi}}
\newunicodechar{ᴮ}{\ensuremath{^B}}
\newunicodechar{+}{}
\newunicodechar{ᵣ}{\ensuremath{_r}}
\newunicodechar{ₓ}{\ensuremath{_x}}
\newunicodechar{ₜ}{\ensuremath{_t}}
\newunicodechar{𝓡}{\ensuremath{\mathcal{R}}}
\newunicodechar{𝓢}{\ensuremath{\mathcal{S}}}
\newunicodechar{𝓚}{\ensuremath{\mathcal{K}}}
\newunicodechar{ᴹ}{\ensuremath{^M}}
\newunicodechar{ᵗ}{\ensuremath{^t}}
\newunicodechar{ξ}{\ensuremath{\xi}}
\newunicodechar{≐}{\ensuremath{\doteq}}
\newunicodechar{⸴}{\ensuremath{,}}
\newunicodechar{ᴱ}{\ensuremath{^E}}
\newunicodechar{⨟}{\ensuremath{\fatsemi}}
\newunicodechar{𝒜}{\ensuremath{\mathcal{A}}}
\newunicodechar{𝒮}{\ensuremath{\mathcal{S}}}
\newunicodechar{∪}{\ensuremath{\cup}}
\newunicodechar{∨}{\ensuremath{\lor}}
\input{agdamacros}
\input{agda-generated}
\input{agda-generated-examples}
\input{agda-generated-scoped}
\input{agda-generated-typed}

\definecolor{agdablue}{HTML}{0000CD}
\DisableLigatures[-]{encoding=T1}
\newcommand{\bsym}[1]{\textcolor{agdablue}{#1}}
\newcommand{\tdot}{\mkern2mu \textcolor{agdablue}{\cdotp} \mkern-2mu \textcolor{agdablue}{\cdotp} \mkern-2mu \textcolor{agdablue}{\cdotp} \mkern1mu}

\bibliographystyle{plainurl}
\title{Working Title: Automated Reasoning over Substitutions in Agda}

\author{Marius Weidner}{University of Freiburg, Germany}{weidner@cs.uni-freiburg.de}{0009-0008-1152-165X}{}
\authorrunning{Marius Weidner}
\Copyright{Marius Weidner}

\keywords{Explicit Substitutions, Rewrite Systems, Agda}

\ccsdesc[300]{Theory of computation~Equational logic and rewriting}

\begin{document}
\AgdaNoSpaceAroundCode{}

\maketitle

\begin{abstract}
      Mechanizing the metatheory of programming languages in proof assistants is
      difficult because of recurring challenges with variable binding. Every
      formalization requires proving and correctly applying standard lemmas about substitutions such
      as those describing composition and interaction with binders.
      Historically researchers treated substitution as a meta level operation
      but explicit substitution calculi bring these operations into the
      syntax. The $σ$-calculus provides reduction rules that ensure
      confluence and strong normalization so that users can compare normal
      forms instead of applying lemmas manually.

      Tools like \textsc{Autosubst 2} provide this procedure via tactics to produce
      \emph{propositional} equalities. We instead register $σ$-calculus rules as
      native reduction rules in Agda using its REWRITE mechanism to produce
      \emph{definitional} equalities. Native embedding is challenging because it
      requires precise control over reduction behavior and care to maintain the
      logical consistency of Agda.

      We treat renamings and substitutions separately at first in a scoped syntax
      setting which is safe and see that the concepts help a lot when working with
      intrinsically typed syntax: If substitution appears as an operation on the
      \emph{index} of intrinsically typed syntax then rewriting the $σ$-calculus laws
      avoids transfer hell in subsequent substitution lemmas. Finally we investigate
      how to adapt the $σ$-calculus to a setting where substitutions and renamings
      are treated uniformly from a proof engineering perspective. This uniform
      treatment is less safe because more work is needed to verify the required
      properties of the resulting rewrite system.
\end{abstract}

\section{Introduction}\label{sec:introduction}

% - sigma calculus is perfect for embeding via rewrite rules (in copnstrast to stark autosubst), bc we 
%   have confluence and termination (cite schäfer)
% -- though we need extension with renamings (both useful and to obtain structural recrusion)
% -- strongly believed to be confluent and termianting 

% - tranfer hell in intrinisc syntax, when type substitution is required
% current trend is to combine renamings and substitutions via kits 
% -- trying to adapt the rules to this setting, find interesting symmetries in the rules with kits
% -- able to prove subject reduction for system f sub, without any lemma application
% -- although we 

\subsection*{Structure}
The paper is organized into several sections.\ \cref{sec:pre} introduces the
theoretical background and tools.\ \cref{sec:mis} presents an embedding of the
$σ$-calculus as native reduction rules into Agda for multi-sorted, scoped
syntax, treating renamings and substitutions separately.\ \cref{sec:its} shows
how the rewriting approach prevents the transfer hell problem for complex
intrinsically typed syntax.\ \cref{sec:ags} explains how to adapt the laws to a
setting where renamings and substitutions can be treated uniformly.\ Finally,
we outline directions for future work in \cref{sec:wrk}, discuss related work
in \cref{sec:rwk} and conclude the paper in \cref{sec:con}.

\subsection*{Contributions}
The main contributions of this work are
\begin{itemize}
      \item an embedding of the $σ$-calculus with first-class renamings into Agda, in a
            scoped and multi-sorted setting, using rewrite rules, exemplified with System~F
            (\cref{sec:mis});
      \item proof of the XXX laws for substitutions for expressions in intrinsically typed
            System~F without entering transfer hell
      \item a novel version of the $σ$-calculus, where the laws are stated in a Kit- and
            Map-based setting. We effectively lift the distinction between variables and
            terms, as well as between renamings and substitutions, from the syntactic to
            the type level.
\end{itemize}

\noindent\textbf{The supplement contains the complete Agda code underlying this paper.}

\section{Preliminaries}\label{sec:pre}
This section introduces the concepts and tools required for the developments in
this work. We summarize
\begin{itemize}
      \item Agda and language features relevant to our work: rewrite rules and opaque
            blocks;
      \item The $σ_{SP}$-calculus;
      \item Multi-sorted and scoped syntaxes in Agda.
\end{itemize}

\subsection{Agda and Relevant Language Features}\label{sec:pre-agd}
Agda is a dependently typed functional language and proof assistant based on
Martin-Löf type theory~\cite{MARTINLOF197573}, where proofs are verified via
type checking. Proofs are significantly simplified when terms are
\textit{definitionally equal} (i.e.\ reduce to the same term). When terms are
not definitionally equal, users must resort to manual equational reasoning
using propositional equality, which is a cumbersome process as Agda lacks
automated proof-discharge tactics. Maximizing definitional equalities thus
directly reduces manual effort.

Beyond standard features like indexed inductive types and dependent pattern
matching~\cite{10.1145/3236770}, this work is based on two more Agda-specific
language features.

% \subsection*{Instance Arguments}
% Instance arguments (\AgdaSymbol{\{\{\dots\}\}}) are resolved via an automated,
% type-directed search similar to Haskell type
% classes~\cite{10.1145/75277.75283}. Agda infers these arguments from values
% marked \AgdaKeyword{instance}. Consider a record for types with a default
% value: \EDefault{} Opening this record with \EDefFields{} allows Agda to
% resolve the \AgdaField{default} field automatically: \noindent\begin{minipage}[t]{0.48\linewidth}
%       \raggedright{}
%       \setlength{\mathindent}{0pt}
%       \EDefInst{}
% \end{minipage}
% \begin{minipage}[t]{0.48\linewidth}%
%       \raggedright{}%
%       \setlength{\mathindent}{0pt}%
%       \EDefInstS{}%
% \end{minipage}
% 
% \noindent These instances enable implicit usage of default values for natural numbers and strings:
% \begin{minipage}[t]{0.48\linewidth}
%       \raggedright{}
%       \EDefEx{}
% \end{minipage}
% \begin{minipage}[t]{0.48\linewidth}
%       \raggedright{}
%       \EDefExS{}
% \end{minipage}

% \noindent Agda infers the appropriate instance (\AgdaFunction{default–ℕ} or \AgdaFunction{default–String}) from the required type, ensuring these equations hold definitionally.

\subsection*{Rewrite Rules}
User-defined rewrite rules~\cite{10.1145/3434341, cockx:LIPIcs.TYPES.2019.2}
extend Agda’s computation by treating propositional equalities as reduction
rules. A function $\AgdaFunction{eq} : (x_1 : A_1) \ldots \to
      \AgdaFunction{f}\, p_1 \ldots p_n \, \AgdaFunction{≡} \, v$ can be registered
as a rewrite rule if:
\begin{enumerate}
      \item The head \AgdaFunction{f} is a postulate, defined function, or constructor with
            distinct variable type parameters.
      \item Every variable $x_i$ appears in a pattern position among $p_1, \ldots,
                  p_n$\footnote{The Agda documentation includes a precise definition for pattern
                  positions:
                  \url{https://agda.readthedocs.io/en/latest/language/rewriting.html}}.
      \item The left side is neutral (it does not further reduce).
\end{enumerate}

\noindent For example, the right identity of addition satisfies these criteria: \ERewrite{} Once registered via \ERewriteIt{}, Agda treats \AgdaBound{n} \AgdaFunction{+} \AgdaNumber{0} as definitionally equal to \AgdaBound{n}: \ERewriteEx{} Users must ensure rules are terminating and confluent to retain logical consistency. While Agda provides conservative confluence checks via \texttt{--confluence-check}, termination is not verified, but also not a critical property\cite{10.1145/3434341}.

\subsection*{Opaque Blocks}
Opaque definitions~\cite{gratzer2022controlling} treat code as non-unfolding
symbols. Definitions within an \AgdaKeyword{opaque} block are visible but
treated abstractly during type checking: \EOpaque{} Outside this block,
\AgdaFunction{forty-two} does not reduce to \AgdaNumber{42} during equality
checks: \EOpaqueExO{} However, the \AgdaKeyword{unfolding} keyword can be used
to manually expose the underlying definition when needed: \EOpaqueExT{}

\subsection{The $\sigma_{SP}$-Caclulus}

\begin{figure}[t]
      \centering
      \scriptsize % 1. Use smaller font
      \setlength{\jot}{0pt} % No vertical gap between equation lines

      % 2. Kill all vertical space around math environments
      \setlength{\abovedisplayskip}{0pt}
      \setlength{\belowdisplayskip}{0pt}
      \setlength{\abovedisplayshortskip}{0pt}
      \setlength{\belowdisplayshortskip}{0pt}
      \setlength{\arraycolsep}{2pt} % Tighter horizontal columns

      % Header 1
      {\raggedright\ \textbf{Symbols} \par}
      \vspace{2pt} % Minimal gap

      \begin{minipage}[t]{0.48\textwidth}
            \raggedright
            \textbf{Variables} $x$
            \begin{align*}
                  \text{innermost} \quad & \textcolor{agdablue}{\mathsf{zero}}    \\
                  \text{outer} \quad     & \textcolor{agdablue}{\mathsf{suc}}\; x
            \end{align*}
            \vspace{2pt}
            \textbf{Terms} $t$
            \begin{align*}
                  \text{var} \quad    & \textcolor{agdablue}{\texttt{\`{}}}x \\
                  \text{abs} \quad    & \textcolor{agdablue}{λ.}\, t         \\
                  \text{app} \quad    & t_1 \ t_2                            \\
                  \text{lookup} \quad & x \textcolor{agdablue}{\;\&\;} σ     \\
                  \text{subst} \quad  & t \tdot{} σ
            \end{align*}
      \end{minipage}
      \hfill
      \begin{minipage}[t]{0.48\textwidth}
            \raggedright
            \textbf{Substitutions} $σ$
            \begin{align*}
                  \text{id} \quad   & \textcolor{agdablue}{\mathsf{id}}                                                                           \\
                  \text{ext} \quad  & t \textcolor{agdablue}{\cdot} σ                                                                             \\
                  \text{comp} \quad & σ_1 \ \textcolor{agdablue}{;} \  σ_2                                                                        \\
                  \text{wk} \quad   & \textcolor{agdablue}{\mathsf{wk}}                                                                           \\
                  \text{lift} \quad & \textcolor{agdablue}{\uparrow}σ ≐ ((\textcolor{agdablue}{\texttt{\`{}}}\textcolor{agdablue}{\mathsf{zero}})
                  \textcolor{agdablue}{\cdot} (σ \ \textcolor{agdablue}{;} \
                  \textcolor{agdablue}{\mathsf{wk}}))
            \end{align*}
      \end{minipage}

      \vspace{0.6em} % Reduced gap between top and bottom halves
      {\raggedright \textbf{Reduction} \par}
      \vspace{2pt}

      \begin{minipage}[t]{0.38\textwidth}
            \raggedright
            \textbf{Definitional Laws}
            \begin{alignat*}{2}
                  \mathsf{id}      \quad          & x \textcolor{agdablue}{\;\&\;} \textcolor{agdablue}{\mathsf{id}}                                     &  & = \textcolor{agdablue}{\texttt{\`{}}} x                                       \\
                  \mathsf{ext}_\mathsf{Z}  \quad  & \textcolor{agdablue}{\mathsf{zero}} \textcolor{agdablue}{\;\&\;} (t \textcolor{agdablue}{\cdot} σ)   &  & = t                                                                           \\
                  \mathsf{ext}_\mathsf{S}   \quad & \textcolor{agdablue}{\mathsf{suc}}\;x \textcolor{agdablue}{\;\&\;} (t \textcolor{agdablue}{\cdot} σ) &  & = x \textcolor{agdablue}{\;\&\;} σ                                            \\
                  \mathsf{comp}    \quad          & x \textcolor{agdablue}{\;\&\;} (σ_1 \ \textcolor{agdablue}{;} \  σ_2)                                &  & = (x \textcolor{agdablue}{\;\&\;} σ_1) \tdot σ_2                              \\
                  \mathsf{wk}      \quad          & x \textcolor{agdablue}{\;\&\;} \textcolor{agdablue}{\mathsf{wk}}                                     &  & = \textcolor{agdablue}{\texttt{\`{}}}(\textcolor{agdablue}{\mathsf{suc}} \ x)
            \end{alignat*}
            \vspace{2pt}
            \textbf{Traversal Laws}
            \begin{alignat*}{2}
                  \mathsf{var} \quad & (\textcolor{agdablue}{\texttt{\`{}}} x)                                 & \tdot σ
                                     & = x \textcolor{agdablue}{\;\&\;} σ                                                 \\
                  \mathsf{lam} \quad & (\textcolor{agdablue}{λ.}\, t)                                          & \tdot  σ
                                     & = \textcolor{agdablue}{λ.} (t  \tdot (\textcolor{agdablue}{\uparrow}σ))            \\
                  \mathsf{app} \quad & (t_1 \ t_2)                                                             & \tdot σ
                                     & = (t_1  \tdot σ) \ (t_2  \tdot σ)
            \end{alignat*}
      \end{minipage}
      \hfill
      \begin{minipage}[t]{0.60\textwidth}
            \raggedright
            \textbf{Interaction Laws}
            \begin{alignat*}{2}
                  \mathsf{id}_\mathsf{L} \quad         & \textcolor{agdablue}{\mathsf{id}} \ \textcolor{agdablue}{;} \  σ                                                                                                    &  & = σ                                                                                \\
                  \mathsf{id}_\mathsf{R} \quad         & σ \ \textcolor{agdablue}{;} \  \textcolor{agdablue}{\mathsf{id}}                                                                                                    &  & = σ                                                                                \\
                  \mathsf{assoc}          \quad        & (σ_1 \ \textcolor{agdablue}{;} \  σ_2) \ \textcolor{agdablue}{;} \  σ_3                                                                                             &  & = σ_1 \ \textcolor{agdablue}{;} \  (σ_2 \ \textcolor{agdablue}{;} \  σ_3)          \\
                  \mathsf{dist}           \quad        & (t \textcolor{agdablue}{\cdot} σ_1) \ \textcolor{agdablue}{;} \  σ_2                                                                                                &  & = (t \tdot σ_2) \textcolor{agdablue}{\cdot} (σ_1 \ \textcolor{agdablue}{;} \  σ_2) \\
                  \mathsf{interact}       \quad             & \textcolor{agdablue}{\mathsf{wk}} \ \textcolor{agdablue}{;} \  (t \textcolor{agdablue}{\cdot} σ)                                                                    &  & = σ                                                                                \\
                  \eta\mathsf{\text{-}id}        \quad & (\textcolor{agdablue}{\texttt{\`{}}\mathsf{zero}}) \textcolor{agdablue}{\cdot} \textcolor{agdablue}{\mathsf{wk}}                                                    &  & = \textcolor{agdablue}{\mathsf{id}}                                                \\
                  \eta\mathsf{\text{-}law}       \quad & (\textcolor{agdablue}{\mathsf{zero}} \textcolor{agdablue}{\;\&\;} σ) \textcolor{agdablue}{\cdot} (\textcolor{agdablue}{\mathsf{wk}} \ \textcolor{agdablue}{;} \  σ) &  & = σ
            \end{alignat*}
            \vspace{2pt}
            \textbf{Monad Laws}
            \begin{alignat*}{2}
                  \mathsf{r\text{-}id} \quad  & t  \tdot \mathsf{id}      &  & = t                                          \\
                  \mathsf{comp}         \quad & (t  \tdot σ_1)  \tdot σ_2 &  & = t  \tdot (σ_1 \textcolor{agdablue}{;} σ_2)
            \end{alignat*}
      \end{minipage}
      \caption{The three-sorted $σ$-calculus}\label{fig:pre-sig}
\end{figure}

The $\sigma$-calculus~\cite{10.1145/96709.96712} formalizes substitutions via
explicit reduction rules. We utilize a three-sorted variant of the
$σ_{SP}$-calculus~\cite{10.1145/2676724.2693163}, which provides a sound,
confluent~\cite{10.1145/226643.226675}, and
complete~\cite{10.1145/2676724.2693163} foundation. Our instantiation for the
$λ$-calculus is presented in \cref{fig:pre-sig}.

The syntax distinguishes between de~Bruijn variables, terms, and parallel
substitutions. Parallel substitutions map variables to terms and support
composition. Unlike standard accounts, we include explicit syntax for variable
usage, requiring distinct symbols for \bsym{\textsf{suc}} and
\bsym{\textsf{wk}} to shift variables or substitutions, respectively. Applying
\bsym{\textsf{wk}} to a term shifts all internal variables. We also define a
\textit{lifting} shorthand to push substitutions under binders; lifting
preserves the bound variable while weakening the substitution's terms to
maintain correct scoping.

Following~\cite{Stark:2020:Mechanising}, we categorize reduction rules into
four groups:
\begin{itemize}
      \item \textbf{Definitional}: Variable lookup in substitutions.
      \item \textbf{Interaction}: Equalities between substitution expressions.
      \item \textbf{Traversal}: Propagation through terms and under binders.
      \item \textbf{Monad}: Behavior of substitutions applied to terms.
\end{itemize}
We omit object-language reductions (e.g., $β$-reduction) to focus strictly on
substitution behavior. While renamings as first-class constructs will break
completeness~\cite{10.1145/3293880.3294101} later, we will prioritize a large space of
solvable equations going forward. We maintain that our three-sorted variant remains
terminating and confluent.

TODO: TALK ABOUT EXTENSION WITH FIRST CLASS RENAMINGS BECAUSE NOT STRUCTUALLY DEFINED

\subsection{Multi-Sorted and Scoped Syntaxes in Agda}\label{sec:pre-syn}
\begin{figure}[t]
      \centering
      \footnotesize % Standard for dense Agda code blocks in LIPIcs
      \begin{minipage}[t]{0.40\linewidth}
            \raggedright{}
            % Optional: add a small negative \vspace if your code commands have built-in padding
            \EScoped{}
      \end{minipage}
      \hfill
      \begin{minipage}[t]{0.58\linewidth}
            \raggedright{}
            \EMultiSorted{}
      \end{minipage}
      \vspace{-1em} % Pull the caption closer to the code to save vertical space
      \caption{Classically Scoped Syntax vs. Multi-Sorted and Scoped Syntax}\label{fig:pre-svm}
\end{figure}

Unlike $λ$-calculus, many languages feature multiple syntactic categories.
System~F~\cite{girard1972, 10.1007/3-540-06859-7_148}, for instance, has
expressions, types, and kinds. To prevent scoping errors, we utilize
\textit{intrinsically scoped syntax}, where inductive syntax definitions are
indexed by free variables to ensure terms are correctly scoped by construction.
\Cref{fig:pre-svm} (left) shows a standard implementation: types are indexed by
type-variable counts, and expressions by both type and expression variable
counts. Variables use de Bruijn indices (\AgdaDatatype{Fin} \AgdaBound{n}).
This separate-datatype approach requires defining many substitution
permutations (e.g., type-in-type, type-in-expr, expr-in-expr).

To avoid this combinatorial explosion, we define the uniform, multi-sorted
syntax in \cref{fig:pre-svm} (right). Terms form a single datatype
(\AgdaBound{S} \AgdaDatatype{⊢} \AgdaBound{s}) indexed by a target sort
\AgdaBound{s} and a \AgdaDatatype{Scope} \AgdaBound{S} (a list of variable
sorts). Variables are represented across multiple sorts simultaneously by
membership proofs (\AgdaBound{S} \AgdaDatatype{∋} \AgdaBound{s}) that act as de
Bruijn indices. This unified structure enables renamings and substitutions to
operate uniformly across all sorts.

\section{The Scoped Sigma Calculus with First-Class Renamings in Agda}\label{sec:mis}
This section presents a direct embedding of the $σ$-calculus rules with
first-class renamings, following the conceptual theory underlying
\textsc{Autosubst 2}, as native reduction rules in Agda using its rewrite
mechanism. We use the multi-sorted and scoped version of SystemF
from~\cref{fig:pre-svm} as an example for any custom syntax. The approach
described in this section directly translates to any multi-sorted and scoped
syntax definition.

\subsection{Scoped and Multi-Sorted Renaming and Substitution}\label{sec:mis-sub}
\begin{figure}[t]
      \centering
      \begin{minipage}[t]{0.43\linewidth}
            \raggedright{}
            \ERen{}
            \ESub{}
      \end{minipage}
      \hfill
      \begin{minipage}[t]{0.54\linewidth}
            \raggedright{}
            \ESubInst{}
      \end{minipage}
      \caption{Scoped and Multi-Sorted Renaming and Substitution (excerpt)}\label{fig:mis-ras}
\end{figure}

We define renamings (\AgdaFunction{\_→ᴿ\_}) and substitutions
(\AgdaFunction{\_→ˢ\_}) uniformly for all variable sorts by indexing them over
scopes. A renaming from scope \AgdaBound{S₁} to \AgdaBound{S₂} maps each
variable in \AgdaBound{S₁} to a variable of the same sort in \AgdaBound{S₂}.
Substitutions map variables to terms of the corresponding sort.

We first define renaming primitives like identity (\AgdaFunction{idᴿ}),
weakening (\AgdaFunction{wkᴿ}), and extension (\AgdaFunction{\_·ᴿ\_}).
Substitution primitives mirror these but use the superscript S. Weakening
remains strictly a renaming. We choose the definition order carefully for
structural recursion. We use weakening to define the lifting of substitutions
(\AgdaFunction{\_↑ˢ\_}).

We provide four composition primitives to support every combination of
renamings and substitutions. These include renaming with renaming
(\AgdaFunction{\_;ᴿᴿ\_}), renaming with substitution (\AgdaFunction{\_;ᴿˢ\_}),
substitution with renaming (\AgdaFunction{\_;ˢᴿ\_}), and substitution with
substitution (\AgdaFunction{\_;ˢˢ\_}). We keep these separate even though the
first two can be defined uniformly.

Traversal functions (\AgdaFunction{\_⋯ᴿ\_} and \AgdaFunction{\_⋯ˢ\_}) are
defined by structural induction on the syntax. Each clause applies a
transformation to a term and lifts the operation when traversing binders. These
clauses match the $σ$-calculus laws. We want these functions to reduce during
type checking.

Renaming and substitution primitives are declared opaque to block their
reduction. This abstraction is cruical for rewriting $σ$-calculus laws where
primitives must act as constant symbols. Blocking reduction makes rewrite rules
predictable. We also declare the types opaque to prevent misuse as functions.
We use explicit application syntax (\AgdaFunction{\_\&ᴿ\_} and
\AgdaFunction{\_\&ˢ\_}) to allow rewriting on renaming or substitution applied
to a variable.

\subsection{The Sigma Calculus with First-Class Renamings}\label{sec:mis-sig}
\begin{figure}[t]
      \centering
      \begin{minipage}[t]{0.58\linewidth}
            \raggedright{}
            \EDefLaws{}
            \EMonadLaws{}
            \EInteractLaws{}
      \end{minipage}
      \hfill
      \begin{minipage}[t]{0.35\linewidth}
            \raggedright{}
            \ERewriteSys{}
      \end{minipage}
      \caption{The Scoped and Multi-Sorted $σ$-calculus with First-Class Renamings}\label{fig:mis-sig}
\end{figure}

We could restrict renamings to the definition of substitutions. However
reasoning about them directly is often useful for instance in the context
morphism lemma\cite{goguen1997candidates}. We treat renamings as first class
constructs similar to \textsc{Autosubst 2}. Figure\ref{fig:mis-sig} presents
the $σ$-calculus laws with first class renamings.

\textbf{Definitional
      laws} now involve the primitives for both renamings and substitutions. Many
\textbf{Interaction laws} share a similar shape. For example we require eight
associativity laws, for all combinations of the four composition operators.
\textbf{Traversal laws} are given via the traversal definitions in the previous
section. Most laws are proven independently of the concrete syntax, except
\textbf{Monad laws}, which must be shown by induction over the syntax. We
highlight these syntax dependent laws in red throughout this work.

\textbf{Coincidence
      laws} emerge from the interaction between renamings and
substitutions\cite{10.1145/3293880.3294101}. They allow us to recover renamings
from their embedding as substitutions. Extracting arbitrary renamings from
deeply nested substitutions would require a dedicated solving strategy, which
we leave for future work but consider important for a more complete account. In
her PhD thesis,\cite{Stark:2020:Mechanising} proves the
\AgdaFunction{coincidence} law by induction over the syntax. We found the law
is not syntax-dependent when the monad laws are available: \ECoincidenceProof{} The remaining coincidence laws follow analogously.

The proofs of the other laws are standard and we refer to the supplement, blue
laws are proven syntax independently while red laws require structural
induction. To this end, we unfold the definitions that are declared opaque.
Inside the proofs we make use function extensionality which is admissable to
the theory underlying Agda.

% \subsection{Confluence}
% Unfortunately Agda cannot (yet) automatically verify confluence using its built
% in triangle criterion check\cite{10.1145/3434341} and in her PhD thesis, Stark
% leaves verifying confluence for the extension with first-class renamings as
% future work. But confluence is a critical property because it preserves the
% logical consistency of the system and deserves extra care. To this end, we
% verified confluence externally using the \textsc{Hakusan} checker available via
% \textsc{CoCoWeb}. This checker also generates a certificate for the result that
% undergoes an additional correctness check using \textsc{CeTA}. We tranlated the
% rewrite system into a format for first order term rewriting systems to enable
% this analysis. The resulting \textsc{.trs} file is included in the supplement.
% 
% On an interesting side note: We also employed QuickCheck to search for
% counterexamples to confluence. Obviously no counterexamples were found but the
% resulting Haskell code is quite interesting in its own right. This code is
% available in the supplement and online\footnote{Link to GitHub: redacted for
%       anonymous review}.

\subsection{Discussion}\label{sec:mis-dis}
The laws presented in \cref{sec:mis-sig} are an extension of the classic
$σ$-calculus rules to a setting with renamings. They fulfill the requirements
to be registered as rewrite rules without further modification. This is a
consequence of marking the relevant symbols opaque. Inside the proofs of the
laws themselves, the definitions can still unfold when needed, while externally
they behave as uninterpreted symbols and are reduced only by the $σ$-calculus
laws. Separating renamings and substitutions causes some duplication. Traversal
and monad laws must be defined twice. Every subsequent result also needs to be
proven for both cases.

This construction reveals a broader concept that serves as a general recipe for
integrating algebraic laws into the proof assistants.
\begin{itemize}
      \item Define symbols for some structure.
      \item Establish confluent rewrite rules that describe custom algebraic laws.
      \item Instantiate the structure with an underlying model to prove the laws.
      \item Hide the reductions of the model to preserve confluence and symbolic behavior.
\end{itemize}

The structure is instantiated to prove correctness but the underlying
reductions need to be hidden from the outside. This abstraction prevents the
system from introducing unwanted simplifications during type checking. For
example lookup should stay as a symbol instead of reducing to function
application. Similarly \AgdaFunction{wk} should remain symbolic instead of
reducing to \AgdaInductiveConstructor{suc}.

This concept of Dedukti~\cite{assaf2023deduktilogicalframeworkbased} style constants and rewrite rules with hidden models
could be applied to other structures like natural numbers, lists or sets. Such
an approach would provide more definitional equalities for indexed types like
vectors. To this end, Agda would benefit greatly from an extension with
rewriting modulo commutativity to support more confluent systems. But then
there is also more work needed to ensure the confluence checker accepts more
complex systems and handles opaque definitions differently.

\section{Avoiding Transfer Hell in Complex Intrinsically Typed Syntax}\label{sec:its}
Rewriting $σ$-calculus laws improves intrinsically typed syntax in addition to
scoped syntax. We demonstrate this by reformulating System F as an example for
any complex syntax, where substitution appears on indices, and prove the monad
laws for expression substitution. These proofs are simplified significantly if
the $σ$-calculus laws for types are rewritten first. This development
represents a significant step toward using intrinsically typed syntax over
complex languages for proofs in Agda. Previous works like
\cite{10.1145/3678000.3678201} required thousands of lines of code to prove
substitution laws. The length of these proofs stems from the transfer lemma
\AgdaFunction{subst} appearing inside the \emph{signature} of expression
substitution lemmas. This leads to the infamous transfer hell where users must
manually manage shifting transfers. Our rewriting approach mitigates this
issue.

\subsection{Substitutions for Complex Intriniscally Typed Syntax}
Types in System F are well formed by construction so a scoped syntax is
sufficient. We restrict the syntax from section~\ref{sec:pre-syn} to the
necessary sorts \TSorts. We then apply the definitions for substitutions and
rewrite rules from section~\ref{sec:mis}. Following the approach
of~\cite{10.1007/978-3-030-33636-3_10} we provide an intrinsically typed
definition for System F expressions.%\TTypeSyn{} 

\noindent\begin{minipage}[t]{0.35\linewidth}
      \raggedright{}
      \TExprCtx{}
\end{minipage}
\hfill
\begin{minipage}[t]{0.58\linewidth}
      \raggedright{}
      \TExprSyn{}
\end{minipage}

The \AgdaDatatype{Ctx} stores well scoped types and includes the
\AgdaInductiveConstructor{\_⸴★} constructor for explicit type weakening. Well
typed variables \AgdaDatatype{\_∈\_} similarly include explicit type weakening
via \AgdaInductiveConstructor{suc★}. Expressions are indexed over their
respective types and utilize the shorthands \TWeaken\ and \TSubst\ on types.
For a more detailed explanation of these underlying definitions we refer
to~\cite{10.1007/978-3-030-33636-3_10}.

Similar to \cite{10.1007/978-3-030-33636-3_10} we define expression
substitutions indexed by type substitutions and renamings follow an analogous
pattern. \TExprSub{}

In constrast to~\cite{10.1007/978-3-030-33636-3_10} we can state all renaming
and substitution lemmas without using \AgdaFunction{subst} inside function
bodies. Composition of expression substitution serves as an
example.\footnote{You may compare our version to the version found
      in~\cite{10.1145/3678000.3678201} at
      \url{https://github.com/proglang/SystemF/blob/e56afb91bfc8dc314bdec6de39a44799fe7e52fa/src/StratF/ExprSubstitution.agda\#L144}}
\TSubComp{} Here the traversal \AgdaFunction{\_⋯ˢᴱ[\_]\_} applies a type
substitution (inside the brakets) and expression substitution to a term
simultaneously. Inside the body agda applies the
\AgdaFunction{compositionalityˢˢ} for type substitutions automatically to
coerce the \emph{Agda type} of the expression \AgdaBound{(σᴱ₁ \_ x)}\
\AgdaFunction{⋯ˢᴱ[}\ \AgdaBound{\_}\ \AgdaFunction{]}\ \AgdaBound{σᴱ₂} from
\AgdaDatatype{Expr}\ \AgdaBound{Γ}\ ((\AgdaBound{t} \AgdaFunction{⋯ˢ}
\AgdaBound{σ₁}) \AgdaFunction{⋯ˢ} \AgdaBound{σ₂}) to \AgdaDatatype{Expr}\
\AgdaBound{Γ}\ (\AgdaBound{t} \AgdaFunction{⋯ˢ} \AgdaBound{(σ₁}\
\AgdaFunction{;ˢˢ}\ \AgdaBound{σ₂)}).

\subsection{Avoiding Transfer Hell in Proofs of Expression Substitution Laws}
Even more, we state, as an example, the category laws of expression
substitutions, without \AgdaFunction{subst} in the signature and thereby
avoiding transfer hell:

Note that subst would have been required on the left of the equality to coerce
the index of the expression substitution using
\AgdaFunction{compositionalityˢˢ}\footnote{Compare this to the version found
      in~\cite{10.1145/3678000.3678201}:
      \url{https://github.com/proglang/SystemF/blob/e56afb91bfc8dc314bdec6de39a44799fe7e52fa/src/StratF/ExprSubstFusion.agda\#L95}}

\subsection{Discussion}
Rewriting $σ$-calculus laws on the indices of intrinsically typed syntax
simplifies substitution lemmas further down the index hiearchy.

Expression substitution $σ$-calculus laws could also be registered as rewrite
rules. This approach might provide subject reduction almost for free through
structural induction. A prototype of this method exists although it is omitted
for brevity. Confluence for the expression substitution rewrite system remains
unverified even though the property likely holds.

\section{Sigma Calculus Rules for a Uniform Treatment of Renamings and Substitutions}\label{sec:ags}
RECENT TREND TO DEFINE REN AND SUB UNIFORMELY CITE.. 
INVESTIGATE SIGMA LAWS IN THIS CONTEXT

\section{Further Work}\label{sec:wrk}

\section{Related Work}\label{sec:rwk}

\subsection*{Explicit Substitution Calculi}
There exist a variety of explicit substitution calculi, but we focus on the
original $σ$-calculus~\cite{10.1145/96709.96712}, which includes identity,
extension, shifting, and composition primitives, and the
$σ_{SP}$-calculus~\cite{10.1145/2676724.2693163}, which adds further reductions
to ensure confluence. Only the $σ_{SP}$-calculus is both sound and complete,
enabling all equations between terms to be derived via reduction. Both
completeness and confluence are established through mechanized
proofs~\cite{10.1145/2676724.2693163, Stark:2020:Mechanising}.
The variantions used in this work, with first-class renamings and Kits 
have not yet been formally verified to be confluent and are definitely not complete.
~\cite{Stark:2020:Mechanising} conjecture their addition with first-class renmaings to 
be confluent and so do we. 

\subsection*{Substitution Libraries for Proof Assistants}
\cite{10.1145/3776704} in section 7.1 similarly to our embed the autosubst 2 laws
(sigma calc with first-class renamings) into Rocq using local rewrite rules. they do give a model for renamings and substitutions as functions, was we do, but they are not able to prove the laws given the underlying model (or even have a mechanism to  do so).

We have already discussed \textsc{Autosubst 2}~\cite{10.1145/3293880.3294101,
      schafer2015autosubst, Stark:2020:Mechanising} which forms the direct
theoretical foundation of our work.

There exists another Agda library with a similar goal to \textsc{Agdasubst},
called \textsc{Abstract Binding Trees}, used for instance by
\cite{siek2021parameterized}. It permits constructing an object language from a
generic syntax, similar to the approach in \cref{sec:ext-uni}, but it supports
only extrinsic scoping and single-sorted syntaxes. The library provides
abstractions over arbitrary predicates about the object language, rather than
focusing on specific relations such as
typing~\cite{saffrich:LIPIcs.ITP.2024.32}. The library also offers incomplete
experimental support for rewriting using $\sigma$-calculus laws, but only
handles renamings and substitutions separately, similar to the approach in
\cref{sec:mis-sig}, rather than in a uniform treatment.

More distantly related is \textsc{Nominal 2}~\cite{Nominal2-AFP}, an Isabelle
package that addresses the $α$-equivalence problem not via de Bruijn indices,
but using nominal datatypes and freshness conditions. Their approach is closer
to the intuitive notion of variables, but the tool does not provide any
strategy for solving equations involving terms and substitutions.

\subsection*{Abstractions over Variables/Terms and Renamings/Substitutions}
We already discussed the original work on Kits~\cite{ren-sub} and their
adaptation to multi-sorted and scoped syntax~\cite{saffrich:LIPIcs.ITP.2024.32}
throughout \cref{sec:ags}.

The intrinsically scoped version of this approach also appears in recent work,
for example in a paper written by \cite{altenkirch2025substitution}, where a
uniform treatment of renaming and substitution is explored in an intrinsically
typed setting. Their work also includes a structure and definitions similar to
the structural kit order we defined in \cref{sec:ags-sko}. They additionally
unify variables and terms, which we deliberately do not do, as variables are
library-defined while terms are user-supplied. Furthermore they hide the
structual dependence of renamings and substitutions in the kit structure
itself, using terminating checking directly, to make Agda see this dependence.
The approach by saffrich instead hides it behind instead hide it in instance
resolution. A further application of the Kit approach is given by
\cite{Wood_2021}, who use it to reason about metatheory in the context of
linear languages.% TODO

Another line of research concerns generic programming
approaches~\cite{allais2021typescopesafeuniverse, 10.1145/3018610.3018613},
which allow the construction of syntax trees from general building blocks
supporting sums, products, and binder introduction. The generic programming
approach also allows for a generic structure that abstracts not only over
renamings and substitutions, but also over semantics and other relevant
language structures. The generic syntax approach can be adapted to a scoped and
multi-sorted setting~\cite{saffrich:LIPIcs.ITP.2024.32}. Importantly, generic
programming and the \textsc{Agdasubst} library are orthogonal, as shown
in~\cref{sec:ext-uni}.

\section{Conclusion}\label{sec:con}
WE WOULD LOVE INTEGRATED CONFLUENCE CHECKING IN AGDA ALSO COMMUTATIVITY
REWRITING AND THE CHECK ONLY LASTED 0.16s, SO AGDA SHOULD BE ABLE TO DO SO
INTERERESTING: HOW TO CHECK CONFLUENCE IN PRESENCE OF OPAQUE (UNFOLDING MAY
BREAK IT)!!!
VERIFY SIGMA CALC WITH FIRST CLASS RENAMINGS IS CONFLUENT 
ADD MORE COINCIDENCES 
We have shown how to take advantage of the $σ$-calculus laws to automatically
solve equations inside proofs about syntaxes with binding in Agda, similar to
\textsc{Autosubst 2} in Rocq.

We first embedded the $σ$-calculus with first-class renamings directly into
Agda as reduction rules, using its rewrite mechanism to yield definitional
equalities for a formalization of System~F.

We then wanted to unify proofs about renamings and substitutions, which led to
the current state of the \textsc{Agdasubst} library. To achieve our goal, we
introduced the notions of Kits and Maps from prior work and adapted the
$σ$-calculus laws to a Kit and Map setting.

We demonstrated how \textsc{Agdasubst} can be used to prove properties
involving substitutions for formal languages, including subject reduction for
System F with subtyping using a single substitution-preserves-typing lemma, and
provided an instantiation of a scoped universe of syntaxes with binding.

The \textsc{Agdasubst} library does not constitute a stable release candidate
yet. We have outlined potential directions for future development and plan to
continue improving the library over time.

\bibliography{references}

\end{document}
