\documentclass[screen,nonacm]{acmart}
\usepackage[capitalize,noabbrev]{cleveref}
\usepackage{agda}

\input{agda-generated}
\newcommand\Aamp{\AgdaFunction{\ensuremath{\&}}}
\newcommand\Att{\AgdaInductiveConstructor{tt}}
\newcommand\Ado{\AgdaKeyword{do}}
\newcommand\AZ{\AgdaDatatype{ℤ}}
\newcommand\Aint{\AgdaInductiveConstructor{int}}
\newcommand\Abool{\AgdaInductiveConstructor{bool}}
\newcommand\Atend{\AgdaInductiveConstructor{end}}
\newcommand\Atsend[2]{\AgdaInductiveConstructor{‼{\textcolor{black}{\ensuremath{#1}}}∙{\textcolor{black}{\ensuremath{#2}}}}}
\newcommand\Atrecv[2]{\AgdaInductiveConstructor{⁇{\textcolor{black}{\ensuremath{#1}}}∙{\textcolor{black}{\ensuremath{#2}}}}}
\newcommand\Atcfsend[1]{\AgdaInductiveConstructor{‼{\textcolor{black}{\ensuremath{#1}}}}}
\newcommand\Atcfrecv[1]{\AgdaInductiveConstructor{⁇{\textcolor{black}{\ensuremath{#1}}}}}
\newcommand\Atcfcomp[2]{\AgdaInductiveConstructor{{\textcolor{black}{\ensuremath{#1}}};{\textcolor{black}{\ensuremath{#2}}}}}
\newcommand\Atcfskip{\AgdaInductiveConstructor{skip}}
\newcommand\ACEND{\AgdaInductiveConstructor{END}}
\newcommand\ACCLOSE{\AgdaInductiveConstructor{CLOSE}}
\newcommand\ACfork{\AgdaInductiveConstructor{fork}}
\newcommand\ACconnect{\AgdaInductiveConstructor{connect}}
\newcommand\ACterminate{\AgdaInductiveConstructor{terminate}}
\newcommand\ACdelegateIN{\AgdaInductiveConstructor{delegateIN}}
\newcommand\ACdelegateOUT{\AgdaInductiveConstructor{delegateOUT}}
\newcommand\ACtransmit{\AgdaInductiveConstructor{transmit}}
\newcommand\ACbranch{\AgdaInductiveConstructor{branch}}
\newcommand\ACclose{\AgdaInductiveConstructor{close}}
\newcommand\ACSEND{\AgdaInductiveConstructor{SEND}}
\newcommand\ACRECV{\AgdaInductiveConstructor{RECV}}
\newcommand\ACSELECT{\AgdaInductiveConstructor{SELECT}}
\newcommand\ACCHOICE{\AgdaInductiveConstructor{CHOICE}}
\newcommand\AFin{\AgdaDatatype{Fin}}
\newcommand\AXCommand{\AgdaDatatype{XCmd}}
\newcommand\ACommand{\AgdaDatatype{Cmd}}
\newcommand\ASession{\AgdaDatatype{Session}}
\newcommand\ASplit{\AgdaDatatype{Split}}
\newcommand\AMSession{\AgdaDatatype{MSession}}
\newcommand\ASetOne{\AgdaDatatype{Set$_1$}}
\newcommand\ACheck{\AgdaFunction{Check}}
\newcommand\ACausality{\AgdaFunction{Causality}}
\newcommand\ACheckDual{\AgdaFunction{CheckDual0}}
\newcommand\Aexec{\AgdaFunction{exec}}
\newcommand\Amu{\AgdaInductiveConstructor{\ensuremath{\mu}}}
\newcommand\AMU{\AgdaInductiveConstructor{LOOP}}
\newcommand\AUNROLL{\AgdaInductiveConstructor{UNROLL}}
\newcommand\ACONTINUE{\AgdaInductiveConstructor{CONTINUE}}
\newcommand\Aback{\AgdaInductiveConstructor{\ensuremath{`}}}
\newcommand\Amanyunaryp{\AgdaFunction{many-unaryp}}
\newcommand\Arestart{\AgdaFunction{restart}}
\newcommand\ASerialize{\AgdaRecord{Serialize}}
\newcommand\ARawMonad{\AgdaRecord{RawMonad}}
\newcommand\AReaderT{\AgdaRecord{ReaderT}}
\newcommand\AStateT{\AgdaRecord{StateT}}
\newcommand\Aput{\AgdaFunction{put}}
\newcommand\Amodify{\AgdaFunction{modify}}
\newcommand\Aget{\AgdaFunction{get}}
\newcommand\Aproject{\AgdaFunction{project}}
\newcommand\AlocateSplit{\AgdaFunction{locate-split}}
\newcommand\Aadjust{\AgdaFunction{adjust}}
\newcommand\Aid{\AgdaFunction{id}}
\newcommand\AprimRecv{\AgdaFunction{primRecv}}
\newcommand\Aleafp{\AgdaFunction{leafp}}
\newcommand\Abranchp{\AgdaFunction{branchp}}
\newcommand\Atreep{\AgdaFunction{treep}}
\newcommand\AIntTree{\AgdaDatatype{IntTree}}
\newcommand\AIntTreeF{\AgdaFunction{IntTreeF}}
\newcommand\ACLeaf{\AgdaInductiveConstructor{Leaf}}
\newcommand\ACBranch{\AgdaInductiveConstructor{Branch}}
\newcommand\AtoN{\AgdaFunction{toℕ}}
\newcommand\Asplit{\AgdaFunction{split}}
\newcommand\AisValue{\AgdaDatatype{isValue}}
\newcommand\AValue{\AgdaDatatype{Value}}
\newcommand\Ajoin{\AgdaFunction{join}}
\newcommand\AExpr{\AgdaDatatype{Expr}}
\newcommand\AVar{\AgdaDatatype{Var}}
\newcommand\ACExpr{\AgdaFunction{CExpr}}
\newcommand\Acont{\AgdaFunction{cont}}
\newcommand\AlevelEnv{\AgdaFunction{levelEnv}}
\newcommand\Adollar{\AgdaOperator{\AgdaFunction{\AgdaUnderscore{}\$\AgdaUnderscore{}}}}
\newcommand\AVSem{\AgdaOperator{\AgdaFunction{𝓥⟦\AgdaUnderscore{}⟧}}}
\newcommand\AESem{\AgdaOperator{\AgdaFunction{𝓔⟦\AgdaUnderscore{}⟧}}}
\newcommand\AESemx[1]{\AgdaOperator{\AgdaFunction{𝓔⟦#1⟧}}}
\newcommand\ADSemx[1]{\AgdaOperator{\AgdaFunction{𝓓⟦#1⟧}}}
\newcommand\AGSem{\AgdaOperator{\AgdaDatatype{𝓖⟦\AgdaUnderscore{}⟧}}}
\newcommand\ATSem{\AgdaOperator{\AgdaDatatype{𝓣⟦\AgdaUnderscore{}⟧}}}
\newcommand\Ainn{\AgdaDatatype{inn}}
\newcommand\ADEnv{\AgdaDatatype{Env\ensuremath{^*}}}
\newcommand\AapplyEnv{\AgdaFunction{apply-env}}
\newcommand\AREL{\AgdaFunction{REL}}
\newcommand\ARelEnv{\AgdaFunction{RelEnv}}
\newcommand{\AV}{\AgdaFunction{𝓥}}
\newcommand{\Asubst}{\AgdaFunction{subst}}
\newcommand{\Asubstlo}{\AgdaFunction{substl$\omega$}}
\newcommand{\Acong}{\AgdaFunction{cong}}
\newcommand{\Atrans}{\AgdaFunction{trans}}
\newcommand{\Arefl}{\AgdaInductiveConstructor{refl}}
\newcommand{\AValueDown}{\AgdaFunction{Value-\ensuremath{\Downarrow}}}
\newcommand{\AGLookup}{\AgdaFunction{𝓖-lookup}}
\newcommand{\ACsubClosed}{\AgdaFunction{Csub-closed}}
\newcommand{\ATsubClosed}{\AgdaFunction{Tsub-closed}}
\newcommand{\Ahere}{\AgdaInductiveConstructor{here}}
\newcommand{\Athere}{\AgdaInductiveConstructor{there}}
\newcommand{\Anull}{\AgdaInductiveConstructor{∅}}
\newcommand{\ATRen}{\AgdaFunction{TRen}}
\newcommand{\ATSub}{\AgdaFunction{TSub}}
\newcommand{\ATren}{\AgdaFunction{Tren}}
\newcommand{\ATsub}{\AgdaFunction{Tsub}}
\newcommand{\ATliftR}{\AgdaFunction{Tliftᴿ}}
\newcommand{\ATliftS}{\AgdaFunction{Tliftˢ}}
\newcommand{\ATidR}{\AgdaFunction{Tidᴿ}}
\newcommand{\ATidS}{\AgdaFunction{Tidˢ}}
\newcommand{\AERen}{\AgdaFunction{ERen}}
\newcommand{\AESub}{\AgdaFunction{ESub}}
\newcommand{\AEren}{\AgdaFunction{Eren}}
\newcommand{\AEsub}{\AgdaFunction{Esub}}
\newcommand{\AEliftR}{\AgdaFunction{Eliftᴿ}}
\newcommand{\AEliftS}{\AgdaFunction{Eliftˢ}}
\newcommand{\AEliftRL}{\AgdaFunction{Eliftᴿ-l}}
\newcommand{\AEliftSL}{\AgdaFunction{Eliftˢ-l}}
\newcommand{\AEidR}{\AgdaFunction{Eidᴿ}}
\newcommand{\AEidS}{\AgdaFunction{Eidˢ}}
\newcommand{\AcompositionalityTSubTSub}{\AgdaFunction{compositionality-TSub-TSub}}
\newcommand{\AcompositionalityESubESub}{\AgdaFunction{compositionality-ESub-ESub}}
\newcommand{\ALRVsub}{\AgdaFunction{LRVsub}}
\newcommand{\AEEsingleSubstPreserves}{\AgdaFunction{EEsingle-subst-preserves}}
\newcommand{\AETsingleSubstPreserves}{\AgdaFunction{ETsingle-subst-preserves}}

\newcommand*\ACode[1]{\AgdaFontStyle{#1}}
\newcommand*\AField[1]{\AgdaField{#1}}
\newcommand*\ACon[1]{\AgdaInductiveConstructor{#1}}
\newcommand*\AKw[1]{\AgdaKeyword{#1}}
\newcommand*\AFun[1]{\AgdaFunction{#1}}
\newcommand*\ACom[1]{\AgdaComment{#1}}

\newcommand\ALvl{\AgdaDatatype{Lvl}}
\newcommand\ASet{\AgdaDatatype{Set}}
\newcommand\AProp{\AgdaDatatype{Prop}}
\newcommand\ASetzero{\AgdaDatatype{zero}}
\newcommand\ASetsuc{\AgdaDatatype{suc}}
\newcommand\ASetlub{\AgdaDatatype{⊔}}
\newcommand\ASeti[1]{\AgdaDatatype{Set}\ensuremath{\,{#1}}}
\newcommand\ASeto{\AgdaDatatype{Setω}}
\newcommand\ASetoo{\AgdaDatatype{Setω\ensuremath{_1}}}
\newcommand\ASetooo{\AgdaDatatype{Setω\ensuremath{_2}}}
\newcommand\Acoe{\AgdaFunction{coe}}
\newcommand\ALevel{\AgdaFunction{Level}}
\newcommand\Atskip{\AgdaInductiveConstructor{tskip}}
\newcommand\Alskip{\AgdaInductiveConstructor{lskip}}
\newcommand{\ATwk}{\AgdaFunction{Twk}}
\newcommand\Aeq{\AgdaFunction{eq}}
\newcommand\AMode{\AgdaDatatype{Mode}}
\newcommand\Afin{\AgdaInductiveConstructor{fin}}
\newcommand\Aany{\AgdaInductiveConstructor{any}}
\newcommand\Asuc{\AgdaInductiveConstructor{suc}}
\newcommand\Azero{\AgdaInductiveConstructor{zero}}
\newcommand\AMutualOrd{\AgdaDatatype{MutualOrd}}
\newcommand\AAcc{\AgdaDatatype{Acc}}
\newcommand\Aacc{\AgdaInductiveConstructor{acc}}
\newcommand\Alookup{\AgdaFunction{lookup}}
\newcommand\ALwk{\AgdaFunction{Lwk}}
\newcommand\ANat{\AgdaInductiveConstructor{Nat}}
\newcommand\AElir{\AgdaDatatype{El\ensuremath{^{ir}}}}
\newcommand\AUp{\AgdaInductiveConstructor{U'}}
\newcommand\APi{\AgdaInductiveConstructor{\ensuremath{\Pi}'}}
\newcommand\ANatp{\AgdaInductiveConstructor{ℕ'}}
\newcommand\ANN{\AgdaInductiveConstructor{ℕ}}
\newcommand\ATL{\AgdaFunction{TL}}
\newcommand\AElLiftLe{\AgdaFunction{ElLift\ensuremath{\le}}}
\newcommand\ATTT{\AgdaFunction{⟦[]TT⟧T}}
\newcommand\ALTT{\AgdaFunction{⟦[]LT⟧T}}
\newcommand\Acrucial{\AgdaFunction{crucial}}
\newcommand\AtoLevel{\AgdaFunction{toLevel}}
\newcommand\ALeLubLub{\AgdaFunction{≤-lublub}}
\newcommand\ALtSucLim{\AgdaFunction{<-suc-lim}}
\newcommand\ALim{\AgdaDatatype{Lim}}
\newcommand\ALift{\AgdaDatatype{Lift}}
\newcommand\ABoundedLevel{\AgdaDatatype{BoundedLevel}}

%%% Local Variables:
%%% mode: latex
%%% TeX-master: "main"
%%% End:

\usepackage{dsfont}
\usepackage{newunicodechar}
\newunicodechar{λ}{\ensuremath{\mathnormal\lambda}}
\newunicodechar{σ}{\ensuremath{\mathnormal\sigma}}
\newunicodechar{τ}{\ensuremath{\mathnormal\tau}}
\newunicodechar{π}{\ensuremath{\mathnormal\pi}}
\newunicodechar{ℕ}{\ensuremath{\mathbb{N}}}
\newunicodechar{∷}{\ensuremath{::}}
\newunicodechar{≡}{\ensuremath{\equiv}}
\newunicodechar{≅}{\ensuremath{\cong}}
\newunicodechar{∀}{\ensuremath{\forall}}
\newunicodechar{ᴸ}{\ensuremath{^L}}
\newunicodechar{ᴿ}{\ensuremath{^R}}
\newunicodechar{ʳ}{\ensuremath{^r}}
\newunicodechar{ᶻ}{\ensuremath{^Z}}
\newunicodechar{ₗ}{\ensuremath{_l}}
\newunicodechar{ⱽ}{\ensuremath{^V}}
\newunicodechar{⟧}{\ensuremath{\rrbracket}}
\newunicodechar{⟦}{\ensuremath{\llbracket}}
\newunicodechar{⊤}{\ensuremath{\top}}
\newunicodechar{⊥}{\ensuremath{\bot}}
\newunicodechar{₁}{\ensuremath{_1}}
\newunicodechar{₂}{\ensuremath{_2}}
\newunicodechar{₃}{\ensuremath{_3}}
\newunicodechar{₄}{\ensuremath{_4}}
\newunicodechar{₅}{\ensuremath{_5}}
\newunicodechar{₆}{\ensuremath{_6}}
\newunicodechar{₇}{\ensuremath{_7}}
\newunicodechar{₈}{\ensuremath{_8}}
\newunicodechar{₉}{\ensuremath{_9}}
\newunicodechar{∈}{\ensuremath{\in}}
\newunicodechar{₀}{\ensuremath{_0}}
\newunicodechar{′}{\ensuremath{'}}
\newunicodechar{ˢ}{\ensuremath{^S}}
\newunicodechar{ᴬ}{\ensuremath{^A}}
\newunicodechar{∘}{\ensuremath{\circ}}
\newunicodechar{𝟙}{\ensuremath{\mathds{1}}}  
\newunicodechar{𝟘}{\ensuremath{\mathds{O}}}
% \newunicodechar{𝟙}{\ensuremath{\mathbb{I}}}  
% \newunicodechar{𝟘}{\ensuremath{\mathbb{O}}}
\newunicodechar{ᴾ}{\ensuremath{^P}}
\newunicodechar{ᵀ}{\ensuremath{^T}}
\newunicodechar{⊎}{\ensuremath{\uplus}}
\newunicodechar{ι}{\ensuremath{\iota}}
\newunicodechar{⇐}{\ensuremath{\Leftarrow}}
\newunicodechar{⇒}{\ensuremath{\Rightarrow}}
\newunicodechar{➙}{\ensuremath{\to^P}}
\newunicodechar{Δ}{\ensuremath{\Delta}}
\newunicodechar{∅}{\ensuremath{\emptyset}}
\newunicodechar{⁺}{\ensuremath{^+}}
\newunicodechar{𝕏}{\ensuremath{\mathbb{X}}}
%\newunicodechar{·}{\ensuremath{\cdot}} %seems to be defined already!
\newunicodechar{∙}{\ensuremath{\cdot}}
\newunicodechar{⁇}{\ensuremath{?}}
\newunicodechar{‼}{\ensuremath{!}}
\newunicodechar{⊕}{\ensuremath{\oplus}}
\newunicodechar{ℤ}{\ensuremath{\mathbb{Z}}}
\newunicodechar{μ}{\ensuremath{\mu}}
\newunicodechar{∃}{\ensuremath{\exists}}
\newunicodechar{;}{\ensuremath{\fatsemi}}
\newunicodechar{Σ}{\ensuremath{\Sigma}}
\newunicodechar{ᴿ}{\ensuremath{^R}}
\newunicodechar{ᵢ}{\ensuremath{_i}}
\newunicodechar{≢}{\ensuremath{\nequiv}}
\newunicodechar{≟}{\ensuremath{\stackrel{{\tiny?}}{=}}}
\newunicodechar{≤}{\ensuremath{\le}}
\newunicodechar{ᵇ}{\ensuremath{^b}}
\newunicodechar{𝓣}{\ensuremath{\mathcal{T}}}
\newunicodechar{𝓔}{\ensuremath{\mathcal{E}}}
\newunicodechar{Γ}{\ensuremath{\Gamma}}
\newunicodechar{γ}{\ensuremath{\gamma}}
\newunicodechar{⊔}{\ensuremath{\sqcup}}
\newunicodechar{α}{\ensuremath{\alpha}}
\newunicodechar{η}{\ensuremath{\eta}}
\newunicodechar{ω}{\ensuremath{\omega}}
\newunicodechar{◁}{\ensuremath{\lhd}}
%PT: seems to be already defined
%\newcommand{\lambdabar}{{\mkern0.75mu\mathchar '26\mkern -9.75mu\lambda}}
\newunicodechar{ƛ}{\ensuremath{\lambdabar}}
\newunicodechar{Λ}{\ensuremath{\mathnormal\Lambda}}
\newunicodechar{ρ}{\ensuremath{\rho}}
\newunicodechar{𝓖}{\ensuremath{\mathcal{G}}}
\newunicodechar{ℓ}{\ensuremath{\ell}}
\newunicodechar{♯}{\ensuremath{\sharp}}
\newunicodechar{⇓}{\ensuremath{\Downarrow}}
\newunicodechar{𝓥}{\ensuremath{\mathcal{V}}}
\newunicodechar{∧}{\ensuremath{\wedge}}
\newunicodechar{ₛ}{\ensuremath{_s}}
\newunicodechar{χ}{\ensuremath{\chi}}
\newunicodechar{⊨}{\ensuremath{\models}}
\newunicodechar{⦂}{\ensuremath{\mathbf{:}}}
\newunicodechar{ς}{\ensuremath{\varsigma}}
\newunicodechar{𝓓}{\ensuremath{\mathcal{D}}}
\newunicodechar{∎}{\ensuremath{\square}}
\newunicodechar{■}{\ensuremath{\blacksquare}}
\newunicodechar{↠}{\ensuremath{\twoheadrightarrow}}
\newunicodechar{↪}{\ensuremath{\hookrightarrow}}
\newunicodechar{ⁱ}{\ensuremath{^i}}
\newunicodechar{Π}{\ensuremath{\Pi}}
\newunicodechar{∋}{\ensuremath{\ni}}
\newunicodechar{∍}{\ensuremath{\ni}}
\newunicodechar{‵}{\ensuremath{`}}
\newunicodechar{⨆}{\ensuremath{\bigsqcup}}
\newunicodechar{δ}{\ensuremath{\delta}}
\newunicodechar{κ}{\ensuremath{\kappa}}
\newunicodechar{⦃}{\ensuremath{\{\!|}}  % \lBrace from stix
\newunicodechar{⦄}{\ensuremath{|\!\}}}  % \rBrace from stix
\newunicodechar{⌊}{\ensuremath{\lfloor}}
\newunicodechar{⌋}{\ensuremath{\rfloor}}
\newunicodechar{𝟎}{\ensuremath{\mathbf{0}}}
\newunicodechar{𝟏}{\ensuremath{\mathbf{1}}}
\newunicodechar{𝟐}{\ensuremath{\mathbf{2}}}
\newunicodechar{β}{\ensuremath{\beta}}
\newunicodechar{≥}{\ensuremath{\geq}}
\newunicodechar{ₒ}{\ensuremath{_o}}
\newunicodechar{∊}{\ensuremath{\in}}
\newunicodechar{;}{\ensuremath{;}}
\newunicodechar{⋯}{\ensuremath{\mkern2mu \cdotp \mkern-2mu \cdotp \mkern-2mu \cdotp \mkern1mu }}
\newunicodechar{⊢}{\ensuremath{\vdash}}
\newunicodechar{∶}{\ensuremath{:}}
\newunicodechar{★}{\ensuremath{\star}}
\newunicodechar{ᴷ}{\ensuremath{^K}}
\newunicodechar{ϕ}{\ensuremath{\varphi}}
\newunicodechar{ᴮ}{\ensuremath{^B}}
\newunicodechar{+}{}
\newunicodechar{ᵣ}{\ensuremath{_r}}
\newunicodechar{ₓ}{\ensuremath{_x}}
\newunicodechar{ₜ}{\ensuremath{_t}}
\newunicodechar{𝓡}{\ensuremath{\mathcal{R}}}
\newunicodechar{𝓢}{\ensuremath{\mathcal{S}}}
\newunicodechar{𝓚}{\ensuremath{\mathcal{K}}}
\newunicodechar{ᴹ}{\ensuremath{^M}}
\newunicodechar{ᵗ}{\ensuremath{^t}}
\newunicodechar{ξ}{\ensuremath{\xi}}

\begin{document}

\title{Agdasubst: Automated Equational Reasoning for Substitutions over Multi-Sorted Syntaxes with Scoped Bindings}

\author{Marius Weidner}
\email{weidner@cs.uni-freiburg.de}
\affiliation{%
  \institution{University of Freiburg}
  \country{Germany}} 
\orcid{0009-0008-1152-165X} % chktex 8
\authornote{
  I hereby declare that I am the sole author and composer of my thesis and that
  no other sources or learning aids, other than those listed, have been used.\\
  Furthermore, I declare that I have acknowledged the work of others by providing
  detailed references of said work.\\
  I hereby also declare that my thesis has not been prepared for another
  examination or assignment, either wholly or excerpts thereof.
  \\
  \\
  \begin{tabular}{p{\textwidth/2} l}
    Freiburg, XX.XX.2025       & yadda yadda                \\ % \includegraphics[width=0.1\textwidth]{signature.png} \\
    \rule{\textwidth/3}{0.4pt} & \rule{\textwidth/3}{0.4pt} \\
    Place, Date                & Signature
  \end{tabular}
}

\begin{abstract}
  Working with substitutions for syntax with binders is a recurring aspect of
  programming language formalization. However, defining and reasoning about
  substitutions, renamings, and their associated laws is often repetitive and
  tedious.

  We present an Agda framework that automatically derives the substitution
  infrastructure for intrinsically scoped, multi-sorted syntax with binders. From
  a single syntax definition, our library derives renaming and substitution
  operations along with corresponding equational laws.

  The derived equational laws are based on the confluent and terminating
  \(\sigma\)-calculus rewriting system for terms with substitutions. Our library
  treats renamings and substitutions uniformly via a generic traversal based on
  McBride’s Kit abstraction, derives equational laws using reflection, and use Agda’s \texttt{REWRITE} mechanism to embed
  suitable proven equations as custom computation rules.
\end{abstract}

\maketitle

\section{Introduction}\label{sec:introduction}

Formalizing programming languages with binders requires careful treatment of
variable binding, substitution, and renaming. While conceptually
straightforward, these operations are notoriously tedious and repetitive to
define and reason about.

This work presents an Agda framework that automatically derives the
substitution infrastructure and standard lemmas needed to work with
intrinsically scoped syntax involving binders.

Intrinsically scoped syntax encodes variable binding and scope information
directly in the type of the syntax definition, unlike extrinsically scoped
approaches, where well-formedness is enforced through additional extrinsic
predicates on the syntax. It sits between extrinsically scoped syntax and
intrinsically typed syntax (i.e.\ encode both scope and typing information in
the type of the syntax), ensuring that terms are well scoped by construction
without committing to the full complexity of typing.

Given a syntax definition as an intrinsically scoped, multi-sorted data type,
where all syntactic categories are encoded uniformly within a single indexed
type, our library derives operations and lemmas for parallel renamings /
substitutions, allowing multiple variables to be substituted simultaneously in
a single traversal.

We derive equations inspired by the \(\sigma\)-calculus, a confluent and
terminating rewriting system for terms with substitutions. We make use of
Agda’s \texttt{REWRITE} mechanism to embed selected equational lemmas as custom
computation rules. This allows equational laws to be applied automatically
during reduction.

Our system treats renaming and substitution uniformly using the Kit
abstraction, originating from work by McBride. In this approach, renaming and
substitution are instances of a single generic traversal function that pushes a
mapping (from variables to variables or terms) through a term. This traversal
is structurally recursive and guaranteed to preserve scope.

\subsection*{Contributions}

\begin{itemize}
  \item \cref{sec:multi} shows how to embed suitable variants of the
        sigma calculus into Agda using rewrite rules, explained through the example of System F.

  \item \cref{sec:kits} extends \cref{sec:multi} to reasoning about Kits instead of
        renamings and substitution

  \item \cref{sec:agdasubst} provides an overview of \texttt{Agdasubst} along with examples,
        explaining how to automatically derive relevant equations using
        reflection~\cite{saffrich:LIPIcs.ITP.2024.32} or enocde programs using generic 
        syntax~\cite{allais2021typescopesafeuniverse}.
        We conduct three case studies:
        System F with and without subtyping and a dependently typed lambda calculus.

  \item In \cref{sec:intrinsic}, we investigate the intrinsically \emph{typed} setting.
        We observe that reasoning about substitutions at deeper levels of the indexing
        hierarchy substantially simplifies, but also requires an intuitive extension of
        the rule set. We also discuss why a reflection based library for the intrinsic
        setting would be hard to build.
\end{itemize}

\textbf{The supplement contains the complete Agda code underlying this paper.}

\section{Preliminaries}\label{sec:preliminaries}

\begin{itemize}
  \item Agda and its rewrite mechanism, which enables the embedding of custom
        $\beta$-reduction rules;
  \item Explicit substitutions, in particular the $\sigma_{SP}$-calculus, a terminating
        and confluent rewrite system for terms with substitutions;
  \item Multi-sorted syntaxes and substitutions, along with abstractions over both,
        including generic scoped syntaxes~\cite{allais2021typescopesafeuniverse} and
        Kit abstractions that unify substitutions and
        renamings~\cite{stronglytyped,ren-sub}.
\end{itemize}

\subsection{Agda and Its Ability to \AgdaPragma{REWRITE}} 

\subsection{Explicit Substitutions and the $\sigma_{SP}$-Calculus}

\subsection{Mutli-Sorted Syntax and Substitutions}

\subsection*{Scope-Safe Universes of Syntaxes with Binding}

\subsection*{Abstractions for Multi-Sorted Substitutions}

\section{Reasoning with Mutli-Sorted Substitutions in Agda}\label{sec:multi}

\subsection{Discussion}\label{sec:discussion-1}

\section{Agdasubst: Reasoning, Automation and Abstraction over Mutli-Sorted Substitutions}\label{sec:agdasubst}

\subsection{Case Studies}

\subsection{Limitations}\label{sec:limitations}

\subsection{Discussion}\label{sec:discussion-2}

\section{The Intriguing Case of Intrinsically \emph{Typed} Syntax}\label{sec:intrinsic}

\subsection{Case Study: Soundness of the Big Step Semantics with Respect to Native Agda Semantics}\label{sec:soundness}

\subsection{Discussion}\label{sec:discussion-3}

\section{Related Work}\label{sec:related-work}

\subsection{Autosubst}\label{sec:autosubst}

\citet{10.1145/3293880.3294101}

\subsection{Other Framkeworks}\label{sec:related-frameworks}

\section{Conclusions}\label{sec:conclusions}

\bibliographystyle{ACM-Reference-Format}
\bibliography{references}

\end{document}
\endinput
