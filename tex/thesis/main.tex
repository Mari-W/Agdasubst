\documentclass[screen,nonacm]{acmart}
\usepackage[capitalize,noabbrev]{cleveref}
\usepackage{agda}

\usepackage{dsfont}
\usepackage{newunicodechar}
\newunicodechar{λ}{\ensuremath{\mathnormal\lambda}}
\newunicodechar{σ}{\ensuremath{\mathnormal\sigma}}
\newunicodechar{τ}{\ensuremath{\mathnormal\tau}}
\newunicodechar{π}{\ensuremath{\mathnormal\pi}}
\newunicodechar{ℕ}{\ensuremath{\mathbb{N}}}
\newunicodechar{∷}{\ensuremath{::}}
\newunicodechar{≡}{\ensuremath{\equiv}}
\newunicodechar{≅}{\ensuremath{\cong}}
\newunicodechar{∀}{\ensuremath{\forall}}
\newunicodechar{ᴸ}{\ensuremath{^L}}
\newunicodechar{ᴿ}{\ensuremath{^R}}
\newunicodechar{ʳ}{\ensuremath{^r}}
\newunicodechar{ᶻ}{\ensuremath{^Z}}
\newunicodechar{ₗ}{\ensuremath{_l}}
\newunicodechar{ⱽ}{\ensuremath{^V}}
\newunicodechar{⟧}{\ensuremath{\rrbracket}}
\newunicodechar{⟦}{\ensuremath{\llbracket}}
\newunicodechar{⊤}{\ensuremath{\top}}
\newunicodechar{⊥}{\ensuremath{\bot}}
\newunicodechar{₁}{\ensuremath{_1}}
\newunicodechar{₂}{\ensuremath{_2}}
\newunicodechar{₃}{\ensuremath{_3}}
\newunicodechar{₄}{\ensuremath{_4}}
\newunicodechar{₅}{\ensuremath{_5}}
\newunicodechar{₆}{\ensuremath{_6}}
\newunicodechar{₇}{\ensuremath{_7}}
\newunicodechar{₈}{\ensuremath{_8}}
\newunicodechar{₉}{\ensuremath{_9}}
\newunicodechar{∈}{\ensuremath{\in}}
\newunicodechar{₀}{\ensuremath{_0}}
\newunicodechar{′}{\ensuremath{'}}
\newunicodechar{ˢ}{\ensuremath{^S}}
\newunicodechar{ᴬ}{\ensuremath{^A}}
\newunicodechar{∘}{\ensuremath{\circ}}
\newunicodechar{𝟙}{\ensuremath{\mathds{1}}}  
\newunicodechar{𝟘}{\ensuremath{\mathds{O}}}
% \newunicodechar{𝟙}{\ensuremath{\mathbb{I}}}  
% \newunicodechar{𝟘}{\ensuremath{\mathbb{O}}}
\newunicodechar{ᴾ}{\ensuremath{^P}}
\newunicodechar{ᵀ}{\ensuremath{^T}}
\newunicodechar{⊎}{\ensuremath{\uplus}}
\newunicodechar{ι}{\ensuremath{\iota}}
\newunicodechar{⇐}{\ensuremath{\Leftarrow}}
\newunicodechar{⇒}{\ensuremath{\Rightarrow}}
\newunicodechar{➙}{\ensuremath{\to^P}}
\newunicodechar{Δ}{\ensuremath{\Delta}}
\newunicodechar{∅}{\ensuremath{\emptyset}}
\newunicodechar{⁺}{\ensuremath{^+}}
\newunicodechar{𝕏}{\ensuremath{\mathbb{X}}}
%\newunicodechar{·}{\ensuremath{\cdot}} %seems to be defined already!
\newunicodechar{∙}{\ensuremath{\cdot}}
\newunicodechar{⁇}{\ensuremath{?}}
\newunicodechar{‼}{\ensuremath{!}}
\newunicodechar{⊕}{\ensuremath{\oplus}}
\newunicodechar{ℤ}{\ensuremath{\mathbb{Z}}}
\newunicodechar{μ}{\ensuremath{\mu}}
\newunicodechar{∃}{\ensuremath{\exists}}
\newunicodechar{;}{\ensuremath{\fatsemi}}
\newunicodechar{Σ}{\ensuremath{\Sigma}}
\newunicodechar{ᴿ}{\ensuremath{^R}}
\newunicodechar{ᵢ}{\ensuremath{_i}}
\newunicodechar{≢}{\ensuremath{\nequiv}}
\newunicodechar{≟}{\ensuremath{\stackrel{{\tiny?}}{=}}}
\newunicodechar{≤}{\ensuremath{\le}}
\newunicodechar{ᵇ}{\ensuremath{^b}}
\newunicodechar{𝓣}{\ensuremath{\mathcal{T}}}
\newunicodechar{𝓔}{\ensuremath{\mathcal{E}}}
\newunicodechar{Γ}{\ensuremath{\Gamma}}
\newunicodechar{γ}{\ensuremath{\gamma}}
\newunicodechar{⊔}{\ensuremath{\sqcup}}
\newunicodechar{α}{\ensuremath{\alpha}}
\newunicodechar{η}{\ensuremath{\eta}}
\newunicodechar{ω}{\ensuremath{\omega}}
\newunicodechar{◁}{\ensuremath{\lhd}}
%PT: seems to be already defined
%\newcommand{\lambdabar}{{\mkern0.75mu\mathchar '26\mkern -9.75mu\lambda}}
\newunicodechar{ƛ}{\ensuremath{\lambdabar}}
\newunicodechar{Λ}{\ensuremath{\mathnormal\Lambda}}
\newunicodechar{ρ}{\ensuremath{\rho}}
\newunicodechar{𝓖}{\ensuremath{\mathcal{G}}}
\newunicodechar{ℓ}{\ensuremath{\ell}}
\newunicodechar{♯}{\ensuremath{\sharp}}
\newunicodechar{⇓}{\ensuremath{\Downarrow}}
\newunicodechar{𝓥}{\ensuremath{\mathcal{V}}}
\newunicodechar{∧}{\ensuremath{\wedge}}
\newunicodechar{ₛ}{\ensuremath{_s}}
\newunicodechar{χ}{\ensuremath{\chi}}
\newunicodechar{⊨}{\ensuremath{\models}}
\newunicodechar{⦂}{\ensuremath{\mathbf{:}}}
\newunicodechar{ς}{\ensuremath{\varsigma}}
\newunicodechar{𝓓}{\ensuremath{\mathcal{D}}}
\newunicodechar{∎}{\ensuremath{\square}}
\newunicodechar{■}{\ensuremath{\blacksquare}}
\newunicodechar{↠}{\ensuremath{\twoheadrightarrow}}
\newunicodechar{↪}{\ensuremath{\hookrightarrow}}
\newunicodechar{ⁱ}{\ensuremath{^i}}
\newunicodechar{Π}{\ensuremath{\Pi}}
\newunicodechar{∋}{\ensuremath{\ni}}
\newunicodechar{∍}{\ensuremath{\ni}}
\newunicodechar{‵}{\ensuremath{`}}
\newunicodechar{⨆}{\ensuremath{\bigsqcup}}
\newunicodechar{δ}{\ensuremath{\delta}}
\newunicodechar{κ}{\ensuremath{\kappa}}
\newunicodechar{⦃}{\ensuremath{\{\!|}}  % \lBrace from stix
\newunicodechar{⦄}{\ensuremath{|\!\}}}  % \rBrace from stix
\newunicodechar{⌊}{\ensuremath{\lfloor}}
\newunicodechar{⌋}{\ensuremath{\rfloor}}
\newunicodechar{𝟎}{\ensuremath{\mathbf{0}}}
\newunicodechar{𝟏}{\ensuremath{\mathbf{1}}}
\newunicodechar{𝟐}{\ensuremath{\mathbf{2}}}
\newunicodechar{β}{\ensuremath{\beta}}
\newunicodechar{≥}{\ensuremath{\geq}}
\newunicodechar{ₒ}{\ensuremath{_o}}
\newunicodechar{∊}{\ensuremath{\in}}
\newunicodechar{;}{\ensuremath{;}}
\newunicodechar{⋯}{\ensuremath{\mkern2mu \cdotp \mkern-2mu \cdotp \mkern-2mu \cdotp \mkern1mu }}
\newunicodechar{⊢}{\ensuremath{\vdash}}
\newunicodechar{∶}{\ensuremath{:}}
\newunicodechar{★}{\ensuremath{\star}}
\newunicodechar{ᴷ}{\ensuremath{^K}}
\newunicodechar{ϕ}{\ensuremath{\varphi}}
\newunicodechar{ᴮ}{\ensuremath{^B}}
\newunicodechar{+}{}
\newunicodechar{ᵣ}{\ensuremath{_r}}
\newunicodechar{ₓ}{\ensuremath{_x}}
\newunicodechar{ₜ}{\ensuremath{_t}}
\newunicodechar{𝓡}{\ensuremath{\mathcal{R}}}
\newunicodechar{𝓢}{\ensuremath{\mathcal{S}}}
\newunicodechar{𝓚}{\ensuremath{\mathcal{K}}}
\newunicodechar{ᴹ}{\ensuremath{^M}}
\newunicodechar{ᵗ}{\ensuremath{^t}}
\newunicodechar{ξ}{\ensuremath{\xi}}
\newunicodechar{≐}{\ensuremath{\doteq}}
\newunicodechar{⸴}{\ensuremath{,}}
\newunicodechar{ᴱ}{\ensuremath{^E}}
\newunicodechar{⨟}{\ensuremath{\fatsemi}}
\newunicodechar{𝒜}{\ensuremath{\mathcal{A}}}
\newunicodechar{𝒮}{\ensuremath{\mathcal{S}}}
\newunicodechar{∪}{\ensuremath{\cup}}
\newunicodechar{∨}{\ensuremath{\lor}}
\input{agdamacros}
\input{agda-generated}
\input{agda-generated-examples}

\definecolor{agdablue}{HTML}{0000CD}
\newcommand{\bsym}[1]{\textcolor{agdablue}{#1}}
\newcommand{\tdot}{\mkern2mu \textcolor{agdablue}{\cdotp} \mkern-2mu \textcolor{agdablue}{\cdotp} \mkern-2mu \textcolor{agdablue}{\cdotp} \mkern1mu}

\begin{document}

\title{Agdasubst: Reasoning for Substitutions over Scoped and Multi-Sorted Syntaxes}

\author{Marius Weidner}
\email{weidner@cs.uni-freiburg.de}
\affiliation{%
      \institution{University of Freiburg}
      \country{Germany}}
\orcid{0009-0008-1152-165X} % chktex 8
\authornote{
      I hereby declare that I am the sole author and composer of my thesis and that
      no other sources or learning aids, other than those listed, have been used.\\
      Furthermore, I declare that I have acknowledged the work of others by providing
      detailed references of said work.\\
      I hereby also declare that my thesis has not been prepared for another
      examination or assignment, either wholly or excerpts thereof.
      \\
      \\
      \begin{tabular}{p{\textwidth/2} l}
            Freiburg, XX.XX.2025       & yadda yadda                \\ % \includegraphics[width=0.1\textwidth]{signature.png} \\
            \rule{\textwidth/3}{0.4pt} & \rule{\textwidth/3}{0.4pt} \\
            Place, Date                & Signature
      \end{tabular}
}

\begin{abstract}

\end{abstract}

\maketitle

\section{Introduction}\label{sec:introduction}

Mechanizing the metatheory of programming languages within interactive theorem
provers frequently presents a significant and recurring technical challenge
when handling variable binding and substitution. While these details are often
ignored in pen-and-paper proofs, their concrete formalization in proof
assistants demands rigorous definitions and proofs as well as a lot of manual
effort. This burden is notable even for fundamental properties such as subject
reduction of simple calculi.

Historically, substitution has been treated as an meta-level operation. To
formalize this, explicit substitution calculi CITE were introduced to bring
substitutions down to a syntactic level. These calculi define substitution
behavior through a set of reduction rules. With the $σ_{SP}$-calculus CITE, an
extension of the $σ$-calculus, a sound and complete CITE rewrite system for
$λ$-calculus terms with substitution has been discovered. Applying the
reduction rules from left to right until a fixed point has reached, guarantees
a normal form that can be compared syntactically to another.

Explicit substitution calculi and the formalization of meta theory in proof
assistants are usually based on de Bruijn indices, which provide an automatic
and name-free method for managing variable binding and α-equivalence, thereby
automatically resolving name clashes.

A core issue in mechanizing the theory for substitutions is the need for
separation of renamings (substitutions that map variables only to other
variables) from general substitutions (replacing variables with full terms), to
allow for a structurally recursive definition. This separation often leads to a
code duplication, where key lemmas, such as substitution preserves typing, must
be proven twice, once for renamings and then again for general substitutions
using the result about renamings. The problem is further compounded in
languages supporting multiple variable sorts, where a combinatory explosion of
variable sorts and syntactic categories can lead to numerous near-identical
substitution definitions and lemmas.

The "Kit" abstraction, introduced by McBride CITE, serves to unify renamings
and substitutions. Automation frameworks have been developed to address this
boilerplate. For Coq, the Autosubst framework [Rambeling 2, 465, 467, 382] and
its successor, Autosubst 2, automate much of this boilerplate [Rambeling 2,
            465, 467, 382]. Autosubst 2 functions as an external code generator capable of
handling multi-sorted and mutually inductive syntax, generating definitions for
de Bruijn terms and parallel or "vectorised" substitutions [Rambeling 2, 465,
            466, 383, 476, 480, 486, 509]. While vector substitutions aim to simplify the
equational theory for multiple sorts [Rambeling 2, 476, 487], Autosubst 2 still
operates in "two worlds," separating renamings and substitutions, and thus
requires initial proofs for renamings before tackling general substitution
properties [Rambeling 2, 466, 475, 495]. Agda's introduction of userined
REWRITE rules presents a promising avenue for integrating σ-calculus rules
natively [Rambeling 2, 466, 214, 246, 431, 479]. These rules allow users to
extend the language's definitional equality with new computation rules,
effectively turning propositional equalities into computational ones [Rambeling
            2, 466, 222, 399, 403, 431]. However, the power of userined rewrite rules also
introduces challenges: arbitrary rules can undermine crucial metatheoretical
properties such as consistency, decidability of type checking, or even subject
reduction, particularly if they are not confluent [Rambeling 2, 215, 228, 229,
            231, 249, 404, 440, 432]. Confluence is particularly important for the
practical feasibility of type checking [Rambeling 2, 215, 228, 229, 231, 249,
            404, 440]. The literature presents a novel Agda library that directly addresses
these challenges by integrating a version of the σ-calculus into Agda
      [Rambeling 2]. This approach draws inspiration from Saffrich's work on
multi-sorted substitutions and McBride's "kit" abstraction, which unifies
renamings and substitutions [Rambeling 2, 364, 386, 398, 422]. The library aims
to resolve the "two-world" problem within Agda by providing a unified
abstraction for renamings and substitutions [Rambeling 2, 358, 479]. This
enables fundamental lemmas, such as "substitution preserves typing," to be
proven once for both concepts, significantly reducing the proof burden
      [Rambeling 2, 358]. By employing Agda's REWRITE rules, the library focuses on
natively embedding a σ-calculus-adjacent rewriting system, ensuring that
substitution operations are not just provable but are definitionally available
within the language [Rambeling 2, 479]. The framework facilitates the reliable
extension of Agda's definitional equality for complex binding structures, while
acknowledging and providing a basis for addressing the inherent metatheoretical
challenges (e.g., confluence and subject reduction) posed by userined rewrite
rules, with the aim for a system where confluence can be checked modularly and
automatically [Rambeling 2, 215].

\subsection*{Structure}

\begin{itemize}
      \item \cref{sec:pre} introduces the theory and tools used in this work;
      \item \cref{sec:mis} shows how to embed a scoped and multi-sorted version of the $\sigma$-calculus in Agda, treating renamings and substitutions separately;
      \item \cref{sec:ags} builds on \cref{sec:mis} and explains the internals of the \texttt{Agdasubst} library, where renamings and substitutions are treated uniformly;
      \item \cref{sec:ext} shows how to build on top of the \texttt{Agdasubst} library and includes case studies.
\end{itemize}

\textbf{The supplement contains the complete Agda code underlying this paper.}

\section{Preliminaries}\label{sec:pre}
This section introduces the foundational concepts and tools required for the
developments in this work. Our work relies on features of the Agda proof
assistant that go beyond standard dependent types, as well as on syntactic
frameworks for modeling substitutions and binders. We summarize the following
components:

\begin{itemize}
      \item Agda features relevant to this work: \AgdaKeyword{instance} arguments,
            \AgdaKeyword{opaque} blocks, and the \AgdaKeyword{REWRITE} mechanism for
            defining custom reduction rules;
      \item The $\sigma_{SP}$-calculus, an explicit substitution calculus that is
            terminating and confluent, serving as a foundation for syntactic reasoning
            about substitution.
\end{itemize}

\subsection{Agda, \AgdaKeyword{instance} Arguments, \AgdaKeyword{opaque} Blocks and \AgdaKeyword{REWRITE}
      Rules}\label{sec:pre-agd} Agda\footnote{CITE} is a dependently typed
programming language and proof assistant based on Martin-Löf type theory. It
allows users to write programs and proofs in a functional language, verifying
correctness through type checking.

In addition to its well-known features such as indexed and parameterized
inductive data types, dependent pattern matching, implicit arguments and
records, Agda also supports three less commonly used but important features
relevant to this work: \AgdaKeyword{instance} arguments, \AgdaKeyword{opaque}
blocks and \AgdaKeyword{REWRITE} rules.

\subsection*{\AgdaKeyword{instance} Arguments}

Instance arguments, written using double braces \verb|{{...}}|, differ from
implicit arguments \verb|{...}| in that they are automatically resolved via an
instance search algorithm, rather than simple implicit argument resolution
limited to the current scope. When a function expects an instance argument,
Agda attempts to infer and insert it from values in scope marked with the
\AgdaKeyword{instance} keyword. This mechanism enables type-directed
programming patterns similar to type classes in Haskell.

As an example, consider the following definition of a type class for types with
a default value: \EDefault{} The record can be opened using instance argument
resolution for its fields: \EDefFields{} We can now declare an instance for
natural numbers: \EDefInst{} This enables implicit usage of the default value
via instance search: \EDefEx{} Here, Agda infers from the type \verb|ℕ| of
\verb|0| that the instance \verb|default–ℕ| applies for \verb|default|,
inserting it automatically to conclude \verb|default ≡ 0| holds definitionally.

\subsection*{\AgdaKeyword{opaque} Blocks}

Agda also supports opaque definitions, which allow certain parts of the code to
be treated as non-unfolding symbols, similar to postulates. This feature is
useful for controlling definitional equality and can be used to prevent
specific definitions from being unfolded during type checking or conversion
checking.

A block is marked as opaque by preceding it with the \AgdaKeyword{opaque}
keyword. All definitions within the block become opaque, meaning they are
excluded from definitional equality checks. However, their names and types
remain visible and can still be used throughout the code. As an example
consider: \EOpaque{}

In this example, \verb|forty-two| is defined as the numeral \verb|42|, but it
is opaque outside the block. This means that although \verb|forty-two| still
has type \verb|ℕ| and can be used in expressions and proofs, it does not reduce
to \verb|5| outside the block during definitional equality checks: \EOpaqueExO

In subsequent opaque blocks, the type checker can be instructed to selectively
unfold previously opaque definitions using the \AgdaKeyword{unfolding} keyword.

Using \AgdaKeyword{unfolding}, the definition will again reduce as expected:
\EOpaqueExT{}

\subsection*{\AgdaKeyword{REWRITE} Rules}

Another feature of Agda are userined \AgdaKeyword{REWRITE} rules\footnote{CITE}
that extend the system’s notion of definitional equality by integrating them
directly into type checking. A rewrite rule in Agda is a named equation of the
form
\[
      \texttt{eq} : (x_1 : A_1)\, \ldots\, (x_k : A_k) \to f\, p_1\, \ldots\, p_n = v
\]
and must satisfy the following \emph{three conditions}\footnote{CITE}:
\begin{enumerate}
      \item The head symbol $f$ must be a \emph{postulate}, a \emph{defined function}, or a
            \emph{constructor} applied to \emph{fully general parameters}, i.e., the
            parameters must be distinct variables.
      \item Each variable $x_1, \ldots, x_k$ must occur at least once in a \emph{pattern
                  position} among $p_1, \ldots, p_n$ (see CITE for the precise definition of
            pattern position).
      \item The left-hand side $f\, p_1\, \ldots\, p_n$ must be \emph{neutral}, i.e., it
            should not reduce any further under existing definitions.
\end{enumerate}

For example: \ERewrite{} The type of the lemma satisfies the conditions above
and can be registered as a rewrite rule using the \AgdaKeyword{REWRITE} pragma:
\ERewriteIt{}

From this point forward, Agda treats \verb|n + 0| as \emph{definitionally
      equal} to \verb|n|: \ERewriteEx

To ensure consistency, Agda requires that all rewrite rules satisfy the three
conditions stated above, and that the combined set of rules is both
\emph{terminating} and \emph{confluent}. However, termination is not currently
checked by Agda; it is only observable indirectly when type checking fails to
terminate. Confluence can be checked using the
\texttt{--local-confluence-check} and \texttt{--confluence-check} flags, but
both checks are necessarily conservative.

\subsection{Explicit Substitution Calculi}\label{sec:pre-sig}

Explicit substitution calculi were introduced to formalize the traditionally
meta-theoretic treatment of substitutions. Unlike the standard approach, where
substitution is defined at the meta-level, these calculi treat substitutions as
syntactic constructs and define their behavior by reduction rules.

Several explicit substitution systems have been proposed. The
$\sigma$-calculus~\cite{CITE3}, the first of its kind, introduces substitution
primitives such as identity, extension, shifting, and composition. The
$\sigma^\Uparrow$-calculus~\cite{CITE4} extends this with a primitive for
lifting.

The $\sigma_{SP}$-calculus is of particular interest because it includes
additional rules to ensure confluence. It is sound, confluent, and
complete~\cite{CITE}. As a result, any valid equality involving substitutions
can be derived using its rules alone.

In the following, we present our own three-sorted version of the
$\sigma$-calculus, using the $\lambda$-calculus as the object language. Our
system closely follows the $\sigma_{SP}$-calculus.

\subsection*{The three-sorted $\sigma$-calculus}

We present a three-sorted version of the $\sigma_{SP}$-calculus in
Figure~\ref{fig:pre:sig}, using the $\lambda$-calculus as the object language.

The syntax is sorted into de~Bruijn variables, terms, and substitution
primitives. Variables~$x$ range over de~Bruijn indices, $t$ over terms, and
$\sigma$ over substitutions.

We use de~Bruijn variables to obtain $\alpha$-equivalence for free. The
expression $\textcolor{agdablue}{\uparrow} \sigma$ is syntactic sugar for
$(\texttt{\`{}}\textcolor{agdablue}{\mathsf{zero}}) \textcolor{agdablue}{\cdot}
      (\sigma \ \textcolor{agdablue}{;} \ \textcolor{agdablue}{\mathsf{wk}})$. Unlike
most presentations in the literature, we include explicit syntax for variable
usage within terms. This requires distinguishing between \bsym{\textsf{suc}}
and \bsym{\textsf{wk}}, depending on whether they appear in the sort of
variables or substitutions.

The reduction rules are grouped into four categories:
\begin{itemize}
      \item \textbf{Definitional laws} describe how variables are looked up in substitutions.
      \item \textbf{Interaction laws} express equalities between substitution expressions.
      \item \textbf{Traversal laws} describe how substitutions propagate through term syntax and under binders.
      \item \textbf{Monad laws} describe the behavior of substitutions when applied to terms.
\end{itemize}

\begin{figure}[t]
      \centering
      \small
      {\raggedright \textbf{\Large Symbols} \par}
      \vspace{1.5em}

      \begin{minipage}[t]{0.48\textwidth}
            \raggedright
            \textbf{Variables} $x$
            \begin{align*}
                  \text{innermost variable} \quad & \textcolor{agdablue}{\mathsf{zero}}    \\
                  \text{outer variables} \quad    & \textcolor{agdablue}{\mathsf{suc}}\; x
            \end{align*}\\
            \vspace{1em}
            \textbf{Terms} $t$
            \begin{align*}
                  \text{variable usage} \quad           & \textcolor{agdablue}{\texttt{\`{}}}x  \\
                  \text{abstraction} \quad              & \textcolor{agdablue}{\lambda.}\, t    \\
                  \text{application} \quad              & t_1 \ t_2                             \\
                  \text{substitution lookup} \quad      & x \textcolor{agdablue}{\;\&\;} \sigma \\
                  \text{substitution application} \quad & t \tdot{} \sigma
            \end{align*}
      \end{minipage}
      \hfill
      \begin{minipage}[t]{0.48\textwidth}
            \raggedright
            \textbf{Substitutions} $\sigma$
            \begin{align*}
                  \text{identity} \quad    & \textcolor{agdablue}{\mathsf{id}}              \\
                  \text{extension} \quad   & t \textcolor{agdablue}{\cdot} \sigma           \\
                  \text{weakening} \quad   & \textcolor{agdablue}{\mathsf{wk}}              \\
                  \text{composition} \quad & \sigma_1 \ \textcolor{agdablue}{;} \  \sigma_2
            \end{align*}
      \end{minipage}

      {\raggedright \vspace{2.0em} \textbf{\Large Reduction} \par}
      \vspace{1.5em}
      \begin{minipage}[t]{0.48\textwidth}
            \raggedright
            \textbf{Definitional Laws}
            \begin{alignat*}{2}
                  \mathsf{id}      \quad          & x \textcolor{agdablue}{\;\&\;} \textcolor{agdablue}{\mathsf{id}}                                          &  & = \textcolor{agdablue}{\texttt{\`{}}} x                    \\
                  \mathsf{ext}_\mathsf{Z}  \quad  & \textcolor{agdablue}{\mathsf{zero}} \textcolor{agdablue}{\;\&\;} (t \textcolor{agdablue}{\cdot} \sigma)   &  & = t                                                        \\
                  \mathsf{ext}_\mathsf{S}   \quad & \textcolor{agdablue}{\mathsf{suc}}\;x \textcolor{agdablue}{\;\&\;} (t \textcolor{agdablue}{\cdot} \sigma) &  & = x \textcolor{agdablue}{\;\&\;} \sigma                    \\
                  \mathsf{wk}      \quad          & x \textcolor{agdablue}{\;\&\;} \textcolor{agdablue}{\mathsf{wk}}                                          &  & = \texttt{\`{}}(\textcolor{agdablue}{\mathsf{suc}} \ x)    \\
                  \mathsf{comp}    \quad          & x \textcolor{agdablue}{\;\&\;} (\sigma_1 \ \textcolor{agdablue}{;} \  \sigma_2)                           &  & = (x \textcolor{agdablue}{\;\&\;} \sigma_1) \tdot \sigma_2
            \end{alignat*}\\
            \vspace{1em}
            \textbf{Traversal Laws}
            \begin{alignat*}{2}
                  \mathsf{var} \quad & (\textcolor{agdablue}{\texttt{\`{}}} x)                                            & \tdot \sigma
                                     & = x \textcolor{agdablue}{\;\&\;} \sigma                                                            \\
                  \mathsf{lam} \quad & (\textcolor{agdablue}{\lambda.}\, t)                                               & \tdot  \sigma
                                     & = \textcolor{agdablue}{\lambda.} (t  \tdot (\textcolor{agdablue}{\uparrow}\sigma))                 \\
                  \mathsf{app} \quad & (t_1 \ t_2)                                                                        & \tdot \sigma
                                     & = (t_1  \tdot \sigma) \ (t_2  \tdot \sigma)
            \end{alignat*}

      \end{minipage}
      \hfill
      \begin{minipage}[t]{0.48\textwidth}
            \raggedright
            \textbf{Interaction Laws}
            \begin{alignat*}{2}
                  \mathsf{comp\text{-}id}_\mathsf{L} \quad & \textcolor{agdablue}{\mathsf{id}} \ \textcolor{agdablue}{;} \  \sigma                                                                                                         &  & = \sigma                                                                                                                                                                                                           \\
                  \mathsf{comp\text{-}id}_\mathsf{R} \quad & \sigma \ \textcolor{agdablue}{;} \  \textcolor{agdablue}{\mathsf{id}}                                                                                                         &  & = \sigma                                                                                                                                                                                                           \\
                  \mathsf{associativity}          \quad    & (\sigma_1 \ \textcolor{agdablue}{;} \  \sigma_2) \ \textcolor{agdablue}{;} \  \sigma_3                                                                                        &  & = \sigma_1 \ \textcolor{agdablue}{;} \  (\sigma_2 \ \textcolor{agdablue}{;} \  \sigma_3)                                                                                                                           \\
                  \mathsf{distributivity}           \quad  & (t \textcolor{agdablue}{\cdot} \sigma_1) \ \textcolor{agdablue}{;} \  \sigma_2                                                                                                &  & = (t \mkern2mu \textcolor{agdablue}{\cdotp} \mkern-2mu \textcolor{agdablue}{\cdotp} \mkern-2mu \textcolor{agdablue}{\cdotp} \sigma_2) \textcolor{agdablue}{\cdot} (\sigma_1 \ \textcolor{agdablue}{;} \  \sigma_2) \\
                  \mathsf{interact}       \quad            & \textcolor{agdablue}{\mathsf{wk}} \ \textcolor{agdablue}{;} \  (t \textcolor{agdablue}{\cdot} \sigma)                                                                         &  & = \sigma                                                                                                                                                                                                           \\
                  \eta\mathsf{\text{-}id}        \quad     & (\textcolor{agdablue}{\texttt{\`{}}\mathsf{zero}}) \textcolor{agdablue}{\cdot} \textcolor{agdablue}{\mathsf{wk}}                                                              &  & = \textcolor{agdablue}{\mathsf{id}}                                                                                                                                                                                \\
                  \eta\mathsf{\text{-}law}       \quad     & (\textcolor{agdablue}{\mathsf{zero}} \textcolor{agdablue}{\;\&\;} \sigma) \textcolor{agdablue}{\cdot} (\textcolor{agdablue}{\mathsf{wk}} \ \textcolor{agdablue}{;} \  \sigma) &  & = \sigma
            \end{alignat*}\\
            \vspace{1em}
            \textbf{Monad Laws}
            \begin{alignat*}{2}
                  \mathsf{right\text{-}id} \quad          & t  \tdot
                  \mathsf{id}                             &           & = t        \\
                  \mathsf{compositionality}         \quad & (t  \tdot
                  \sigma_1)  \tdot
                  \sigma_2                                &           & = t  \tdot
                  (\sigma_1 ; \sigma_2)
            \end{alignat*}
      \end{minipage}
      \caption{The three-sorted $\sigma$-calculus.}
      \label{fig:pre:sig}
\end{figure}

We do not include reduction rules for the $\lambda$-calculus itself, such as
$\beta$-reduction. Our focus is not on the $\lambda\sigma$-calculus[CITE], but
on the behavior of substitutions applied to terms. The reduction strategy of
the object language is part of the language and is left unspecified. The only
term-dependent rules we need are those that describe how substitutions traverse
terms.

We believe the three-sorted $\sigma$-calculus is terminating and confluent.
Although we do not give a formal proof, it is a reasonable variant of the
$\sigma_{SP}$-calculus.

Renamings, introduced in the next section, are needed to define substitutions
structurally in Agda. They are believed to break completeness~\cite{CITE}. For
this reason, we do not consider completeness a central property in this work.
Still, we try to preserve as much completeness as possible in what follows.

\section{The Multi-Sorted and Scoped Sigma Calculus in Agda}\label{sec:mis}

This section gives a direct embedding of the $\sigma$-calculus rules as native
reduction rules in Agda using the \AgdaKeyword{REWRITE} mechanism.

When adapting the calculus from \cref{sec:pre-sig} to custom syntaxes in Agda,
we face three main challenges:

\begin{itemize}
      \item \textbf{Multiple sorts:} Unlike the simply typed $\lambda$-calculus, many
            object languages have more than one sort of syntax. For example, System~F
            has kinds, types and expressions, with several kinds of substitution: type-in-type,
            type-in-expression, and expression-in-expression.

      \item \textbf{Scope safety:} To avoid scoping related errors, we want the
            syntax to enforce correct scoping intrinsically. This is usually done in Agda by
            defining the syntax as a family of types indexed by its set of free variables.

      \item \textbf{Structural recursion:} In Agda, substitutions for syntaxes with binders
            cannot be defined directly by structural recursion.
            To obtain well-founded definitions, we need to define renamings first, which are
            substitutions that only contain variables.
\end{itemize}

We first address the first two points by introducing intrinsically scoped and
multi-sorted syntaxes, defined as indexed datatypes in Agda. We use System~F as
a running example. After that, we address the third challenge by defining
substitutions in terms of renamings, which are likewise scoped and sorted.
Finally, we revisit the $\sigma$-calculus and adapt its laws to this setting.

\subsection{Mutli-Sorted and Scoped Syntaxes}\label{sec:mis-sub}

We start with a standard intrinsically scoped syntax for System~F. Types are
indexed by the number of free type variables, and expressions are indexed by
both free type and expression variables. Variables are represented by de Bruijn
indices, which are numbers less than the number of free variables. This ensures
that only well scoped terms can be formed. The left side of \cref{fig:mis:svm}
shows this version.

This approach is simple but has drawbacks. Types and expressions are separate
datatypes, so we would need to define multiple substitution operations:
type-in-type, expr-in-expr, and type-in-expr. These substitutions need to be
handled and related manually, which adds boilerplate.

To avoid this, we use a multi sorted syntax. Terms form a single datatype
indexed by a list of variable sorts and a target sort. The type $S \vdash s$
describes a term of sort $s$ with free variables of sorts in $S$. Variable
positions are given by $S \ni s$, a proof that a variable of sort $s$ appears
in $S$ at some position and acts as a de Bruijn index. This gives us well
scoped and well sorted terms by construction. The right side of
\cref{fig:mis:svm} shows this version.

With this unified syntax, we can define generic operations that work uniformly
across sorts. Next, we will use this to our advantage and define scoped and
sorted renamings and substitutions.

\begin{figure}[h]
      \centering
      \begin{minipage}[t]{0.48\linewidth}
            \raggedright{}
            \EScoped{}
      \end{minipage}
      \hfill
      \begin{minipage}[t]{0.48\linewidth}
            \raggedright{}
            \EMultiSorted{}
      \end{minipage}
      \caption{Classically Scoped Syntax vs. Multi-Sorted and Scoped Syntax}
      \label{fig:mis:svm}
\end{figure}

\subsection{Scoped and Multi Sorted Renamings and Substitutions}
We define renamings and substitutions uniformly over all variable sorts by
indexing them over scopes, as shown in \cref{fig:mis:ras}. A renaming from
scope $S_1$ to $S_2$ maps each variable of a given sort in $S_1$ to a variable
of the same sort in $S_2$. Similarly, a substitution maps variables to terms of
the corresponding sort.

The order of definitions is chosen carefully. We first introduce the renaming
type and its primitive operations. Substitutions are then defined in terms of
renamings, using the weakening renaming primitive to define lifting in a
structurally recursive manner.

To support all combinations of renamings and substitutions, we provide four
forward composition primitives: renaming composed with renaming, renaming with
substitution, substitution with renaming, and substitution with substitution.

The traversal functions, \texttt{\_⋯ᴿ\_} for renaming and \texttt{\_⋯ˢ\_} for
substitution, are defined by induction on the syntax. Each traversal applies a
renaming or substitution to a term. The clauses in these definitions correspond
directly to the traversal laws of the $\sigma$-calculus (with duplications for renamings), and therefore, we want these functions to reduce during type checking.

All renaming and substitution primitives are declared \texttt{opaque} in Agda,
which blocks their reduction. This abstraction is essential for formulating the
$\sigma$-calculus laws, where the primitives must act as uninterpreted symbols
rather than as concrete functions. Blocking reduction in this way allows
rewrite rules to apply predictably. 
The types for renamings and substitutions are also declared \texttt{opaque} to prevent misuse as actual functions. Their map-like behavior remains accessible through the lookup operator.

Together, this setup yields a uniform treatment for renaming and
substitution. The main drawback is
the unavoidable duplication of primitives and the need for four composition
operators.

\begin{figure}[h]
      \centering
      \begin{minipage}[t]{0.48\linewidth}
            \raggedright{}
            \ERen{}
      \end{minipage}
      \hfill
      \begin{minipage}[t]{0.48\linewidth}
            \raggedright{}
            \ESub{}
      \end{minipage}
      \caption{Scoped and Multi-Sorted Renamings and Substitutions}
      \label{fig:mis:ras}
\end{figure}
% straight forward way to embed the sigma calculus in agda, but with ren and subs
% multiple laws 
% recall syntax from preliminaries 
% define substitutions etc
% proven in supplement
% note induction on t in monad laws
% funext dependence
% they all are rewritable
\begin{figure}[h]
      \centering

      \begin{minipage}[t]{0.38\linewidth}
            \raggedright{}
            \EDefLaws{}
      \end{minipage}
      \hfill
      \begin{minipage}[t]{0.58\linewidth}
            \raggedright{}
            \EMonadLaws{}
      \end{minipage}

      \vspace{1em}
      \begin{minipage}[t]{0.38\linewidth}
            \raggedright{}
            \ERewriteSys{}
      \end{minipage}
      \hfill
      \raisebox{7.75em}[0pt][0pt]{%
            \begin{minipage}[t]{0.58\linewidth}
                  \raggedright{}
                  \EInteractLaws{}
            \end{minipage}
      }
      \caption{The Scoped and Multi-Sorted $\sigma$-calculus with First-Class Renamings and Substitutions.}
      \label{fig:six-frame-layout}
\end{figure}

\subsection{}

\subsection{Discussion}\label{sec:rsn-dis}
% putting 1 and 2 together to gain new library
% that enables one to prove stuff easy for sub
% and ren in proof theoretic setting

\section{The Agdasubst Library}\label{sec:ags}
% work based on McBrides kits and saffrichs compose kits

% \subsection{Related Work: Abstractions for Multi-Sorted Substitutions}

% shows the implementation
% explain main idea
% running example will be system f
% show how to but ren and sub together!

\subsection{Kits and Traversal}\label{sec:ags-kit}

% tags
% symbols are abstract
% ren and sub instances
% symbols in explicit substitutions are not meta -> they are
% symbols, make them opaque
% footnote? % functions != fields % locking

\subsection{Composition and Compositionality}\label{sec:ags-kit}
% compose kits (symbols etc)
% new operator lookup/traverse `/id
% we will need extensions to the calculus

% commutativity issue
% postulated the uniqueness
% bounded semi lattice structure to calculate the 3rd kit of the compose kit 
% footnote tried only 2 but not general enough

\subsection{The Final Set of Rules}\label{sec:ags-rls}
\begin{figure}[t]
      \centering
      \small
      {\raggedright \textbf{\Large Symbols} \par}
      \vspace{1.5em}

      \begin{minipage}[t]{0.48\textwidth}
            \raggedright{}
            \textbf{Sorts} $s$ \\
            \ESortTy{}\\
            \vspace{0.5em}
            \emph{.. custom object language sorts ..}\\
            \vspace{0.5em}
            \textbf{Scopes} $S$ \\
            \EScopeDefTy{}\\
            \vspace{0.5em}
            \EScopeDef{}\\
            \vspace{0.5em}
            \textbf{Variables} $x$ \\
            \EVarsTy{}\\
            \vspace{0.5em}
            \EVars{}\\
            \vspace{0.5em}
            \textbf{Terms} $t$ \\
            \ETmC{}\\
            \vspace{0.5em}
            \EVarC{}\\
            \emph{.. custom sorted object language ..}\\
            \vspace{1.0em}
      \end{minipage}
      \hfill
      \begin{minipage}[t]{0.48\textwidth}
            \raggedright{}
            \textbf{Kits} $K$ \\
            \EKitDefTy{}\\
            \vspace{0.5em}
            \EKitDef{}\\
            \vspace{0.5em}
            \textbf{Variables or Terms} $q$\\
            \EVarTrmTy{}\\
            \vspace{0.5em}
            \EVarTrmApp{}\\
            \vspace{0.5em}
            \textbf{Renamings or Substitutions} $ϕ$\\
            \EPrimsTy{}\\
            \vspace{0.5em}
            \EPrims{}\\
            \vspace{0.5em}
      \end{minipage}

      {\raggedright \vspace{2.0em} \textbf{\Large Reduction} \par}
      \vspace{1.5em}
      \begin{minipage}[t]{0.48\textwidth}
            \raggedright
            \textbf{Kit Laws}\\
            \ETypeLevel{}\\
            \vspace{0.5em}
            \textbf{Definitional Laws}\\
            \vspace{0.5em}
            \EDefLawTy{}\\
            \vspace{0.5em}
            \EDefLaw{}\\

      \end{minipage}
      \hfill
      \begin{minipage}[t]{0.48\textwidth}
            \raggedright
            \textbf{Interaction Laws}\\
            \EInteraction{}\\
            \vspace{0.5em}
            \textbf{Monad Laws}\\
            \EMonad{}\\
            \vspace{0.5em}
            \textbf{Traversal Laws}\\
            \ETravL{}\\
            \emph{.. custom object language traversal ..}
      \end{minipage}
      \caption{The \emph{Idealized} Scoped and Multi-Sorted $\sigma$-Calculus with \AgdaDatatype{Kits}.}
      \label{fig:ags:fin}
\end{figure}

\subsection{Limitations}\label{sec:ags-lim}
% assoc problem, because we dont know the kit on the left
% new symbols introduce need for more rules 
% agda does not deem the calculus confluent
% % -> see how many critical pairs the sigma calculus even has! 

\section{Extensions}\label{sec:ext}
\subsection{Generic Syntax}\label{sec:ext-uni}
\subsection{Typing Kits}\label{sec:ext-tyk}
% generic syntax allows to prove substitution lemmas only once. 
% sometimes inconvenient to use in agda
% generics and 
% typing kits
% room for more! e.g. linear types / resources etc or semantics e.g. confluence

% \section{Case Studies}\label{sec:cas}
% \subsection{System F}\label{sec:cas-sf}
% \cite{girard1972}
% \subsection{System F with Subtyping}\label{sec:cas-sfsub}
% 
% \subsection{Martin-Löf Type Theory}\label{sec:cas-mltt}

\section{Related Work}\label{sec:rel}
% the next works are essential building blocks for agdasubst and we build on both
% only really take the idea of autosubst

% \subsection{Autosubst 2}\label{sec:rel-aut}
% Autosubst 2 (coq)
% Nominal 2 (isabelle)
% Abstract Binding Tree Library (agda)

\section{Future Work}\label{sec:wrk}
% make it more complete?
% fix the assoc problem (see last sections)
% obviously improve user exp (i.e. derive code, documentation etc)

\subsection{Extensions to the Calculus}\label{sec:wrk-ext}
% variadic bindings
% lists / map
% .. the kathrins thesis
\subsection{Meta Theory}\label{sec:wrk-prp}
% meta theoretics
% termination
% confluence
% make it more complete?
\subsection{More Generality}\label{sec:wrk-prp}
% abstracting away the sorting to arbitrary, well behaved telescopes (by axiom)
% suitable for intrinsic typing

\section{Conclusions}\label{sec:con}

\bibliographystyle{ACM-Reference-Format}
\bibliography{references}

\end{document}
