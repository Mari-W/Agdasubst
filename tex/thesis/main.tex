\documentclass[screen,nonacm]{acmart}
\usepackage[capitalize,noabbrev]{cleveref}
\usepackage{agda}

\usepackage{dsfont}
\usepackage{newunicodechar}
\newunicodechar{λ}{\ensuremath{\mathnormal\lambda}}
\newunicodechar{σ}{\ensuremath{\mathnormal\sigma}}
\newunicodechar{τ}{\ensuremath{\mathnormal\tau}}
\newunicodechar{π}{\ensuremath{\mathnormal\pi}}
\newunicodechar{ℕ}{\ensuremath{\mathbb{N}}}
\newunicodechar{∷}{\ensuremath{::}}
\newunicodechar{≡}{\ensuremath{\equiv}}
\newunicodechar{≅}{\ensuremath{\cong}}
\newunicodechar{∀}{\ensuremath{\forall}}
\newunicodechar{ᴸ}{\ensuremath{^L}}
\newunicodechar{ᴿ}{\ensuremath{^R}}
\newunicodechar{ʳ}{\ensuremath{^r}}
\newunicodechar{ᶻ}{\ensuremath{^Z}}
\newunicodechar{ₗ}{\ensuremath{_l}}
\newunicodechar{ⱽ}{\ensuremath{^V}}
\newunicodechar{⟧}{\ensuremath{\rrbracket}}
\newunicodechar{⟦}{\ensuremath{\llbracket}}
\newunicodechar{⊤}{\ensuremath{\top}}
\newunicodechar{⊥}{\ensuremath{\bot}}
\newunicodechar{₁}{\ensuremath{_1}}
\newunicodechar{₂}{\ensuremath{_2}}
\newunicodechar{₃}{\ensuremath{_3}}
\newunicodechar{₄}{\ensuremath{_4}}
\newunicodechar{₅}{\ensuremath{_5}}
\newunicodechar{₆}{\ensuremath{_6}}
\newunicodechar{₇}{\ensuremath{_7}}
\newunicodechar{₈}{\ensuremath{_8}}
\newunicodechar{₉}{\ensuremath{_9}}
\newunicodechar{∈}{\ensuremath{\in}}
\newunicodechar{₀}{\ensuremath{_0}}
\newunicodechar{′}{\ensuremath{'}}
\newunicodechar{ˢ}{\ensuremath{^S}}
\newunicodechar{ᴬ}{\ensuremath{^A}}
\newunicodechar{∘}{\ensuremath{\circ}}
\newunicodechar{𝟙}{\ensuremath{\mathds{1}}}  
\newunicodechar{𝟘}{\ensuremath{\mathds{O}}}
% \newunicodechar{𝟙}{\ensuremath{\mathbb{I}}}  
% \newunicodechar{𝟘}{\ensuremath{\mathbb{O}}}
\newunicodechar{ᴾ}{\ensuremath{^P}}
\newunicodechar{ᵀ}{\ensuremath{^T}}
\newunicodechar{⊎}{\ensuremath{\uplus}}
\newunicodechar{ι}{\ensuremath{\iota}}
\newunicodechar{⇐}{\ensuremath{\Leftarrow}}
\newunicodechar{⇒}{\ensuremath{\Rightarrow}}
\newunicodechar{➙}{\ensuremath{\to^P}}
\newunicodechar{Δ}{\ensuremath{\Delta}}
\newunicodechar{∅}{\ensuremath{\emptyset}}
\newunicodechar{⁺}{\ensuremath{^+}}
\newunicodechar{𝕏}{\ensuremath{\mathbb{X}}}
%\newunicodechar{·}{\ensuremath{\cdot}} %seems to be defined already!
\newunicodechar{∙}{\ensuremath{\cdot}}
\newunicodechar{⁇}{\ensuremath{?}}
\newunicodechar{‼}{\ensuremath{!}}
\newunicodechar{⊕}{\ensuremath{\oplus}}
\newunicodechar{ℤ}{\ensuremath{\mathbb{Z}}}
\newunicodechar{μ}{\ensuremath{\mu}}
\newunicodechar{∃}{\ensuremath{\exists}}
\newunicodechar{;}{\ensuremath{\fatsemi}}
\newunicodechar{Σ}{\ensuremath{\Sigma}}
\newunicodechar{ᴿ}{\ensuremath{^R}}
\newunicodechar{ᵢ}{\ensuremath{_i}}
\newunicodechar{≢}{\ensuremath{\nequiv}}
\newunicodechar{≟}{\ensuremath{\stackrel{{\tiny?}}{=}}}
\newunicodechar{≤}{\ensuremath{\le}}
\newunicodechar{ᵇ}{\ensuremath{^b}}
\newunicodechar{𝓣}{\ensuremath{\mathcal{T}}}
\newunicodechar{𝓔}{\ensuremath{\mathcal{E}}}
\newunicodechar{Γ}{\ensuremath{\Gamma}}
\newunicodechar{γ}{\ensuremath{\gamma}}
\newunicodechar{⊔}{\ensuremath{\sqcup}}
\newunicodechar{α}{\ensuremath{\alpha}}
\newunicodechar{η}{\ensuremath{\eta}}
\newunicodechar{ω}{\ensuremath{\omega}}
\newunicodechar{◁}{\ensuremath{\lhd}}
%PT: seems to be already defined
%\newcommand{\lambdabar}{{\mkern0.75mu\mathchar '26\mkern -9.75mu\lambda}}
\newunicodechar{ƛ}{\ensuremath{\lambdabar}}
\newunicodechar{Λ}{\ensuremath{\mathnormal\Lambda}}
\newunicodechar{ρ}{\ensuremath{\rho}}
\newunicodechar{𝓖}{\ensuremath{\mathcal{G}}}
\newunicodechar{ℓ}{\ensuremath{\ell}}
\newunicodechar{♯}{\ensuremath{\sharp}}
\newunicodechar{⇓}{\ensuremath{\Downarrow}}
\newunicodechar{𝓥}{\ensuremath{\mathcal{V}}}
\newunicodechar{∧}{\ensuremath{\wedge}}
\newunicodechar{ₛ}{\ensuremath{_s}}
\newunicodechar{χ}{\ensuremath{\chi}}
\newunicodechar{⊨}{\ensuremath{\models}}
\newunicodechar{⦂}{\ensuremath{\mathbf{:}}}
\newunicodechar{ς}{\ensuremath{\varsigma}}
\newunicodechar{𝓓}{\ensuremath{\mathcal{D}}}
\newunicodechar{∎}{\ensuremath{\square}}
\newunicodechar{■}{\ensuremath{\blacksquare}}
\newunicodechar{↠}{\ensuremath{\twoheadrightarrow}}
\newunicodechar{↪}{\ensuremath{\hookrightarrow}}
\newunicodechar{ⁱ}{\ensuremath{^i}}
\newunicodechar{Π}{\ensuremath{\Pi}}
\newunicodechar{∋}{\ensuremath{\ni}}
\newunicodechar{∍}{\ensuremath{\ni}}
\newunicodechar{‵}{\ensuremath{`}}
\newunicodechar{⨆}{\ensuremath{\bigsqcup}}
\newunicodechar{δ}{\ensuremath{\delta}}
\newunicodechar{κ}{\ensuremath{\kappa}}
\newunicodechar{⦃}{\ensuremath{\{\!|}}  % \lBrace from stix
\newunicodechar{⦄}{\ensuremath{|\!\}}}  % \rBrace from stix
\newunicodechar{⌊}{\ensuremath{\lfloor}}
\newunicodechar{⌋}{\ensuremath{\rfloor}}
\newunicodechar{𝟎}{\ensuremath{\mathbf{0}}}
\newunicodechar{𝟏}{\ensuremath{\mathbf{1}}}
\newunicodechar{𝟐}{\ensuremath{\mathbf{2}}}
\newunicodechar{β}{\ensuremath{\beta}}
\newunicodechar{≥}{\ensuremath{\geq}}
\newunicodechar{ₒ}{\ensuremath{_o}}
\newunicodechar{∊}{\ensuremath{\in}}
\newunicodechar{;}{\ensuremath{;}}
\newunicodechar{⋯}{\ensuremath{\mkern2mu \cdotp \mkern-2mu \cdotp \mkern-2mu \cdotp \mkern1mu }}
\newunicodechar{⊢}{\ensuremath{\vdash}}
\newunicodechar{∶}{\ensuremath{:}}
\newunicodechar{★}{\ensuremath{\star}}
\newunicodechar{ᴷ}{\ensuremath{^K}}
\newunicodechar{ϕ}{\ensuremath{\varphi}}
\newunicodechar{ᴮ}{\ensuremath{^B}}
\newunicodechar{+}{}
\newunicodechar{ᵣ}{\ensuremath{_r}}
\newunicodechar{ₓ}{\ensuremath{_x}}
\newunicodechar{ₜ}{\ensuremath{_t}}
\newunicodechar{𝓡}{\ensuremath{\mathcal{R}}}
\newunicodechar{𝓢}{\ensuremath{\mathcal{S}}}
\newunicodechar{𝓚}{\ensuremath{\mathcal{K}}}
\newunicodechar{ᴹ}{\ensuremath{^M}}
\newunicodechar{ᵗ}{\ensuremath{^t}}
\newunicodechar{ξ}{\ensuremath{\xi}}
\newunicodechar{≐}{\ensuremath{\doteq}}
\newunicodechar{⸴}{\ensuremath{,}}
\newunicodechar{ᴱ}{\ensuremath{^E}}
\newunicodechar{⨟}{\ensuremath{\fatsemi}}
\newunicodechar{𝒜}{\ensuremath{\mathcal{A}}}
\newunicodechar{𝒮}{\ensuremath{\mathcal{S}}}
\newunicodechar{∪}{\ensuremath{\cup}}
\newunicodechar{∨}{\ensuremath{\lor}}
\input{agdamacros}
\input{agda-generated}
\input{agda-generated-examples}

\definecolor{agdablue}{HTML}{0000CD}
\newcommand{\bsym}[1]{\textcolor{agdablue}{#1}}
\newcommand{\tdot}{\mkern2mu \textcolor{agdablue}{\cdotp} \mkern-2mu \textcolor{agdablue}{\cdotp} \mkern-2mu \textcolor{agdablue}{\cdotp} \mkern1mu}

\begin{document}

\title{Agdasubst: Automated Reasoning for Unified Renaming and Substitution over Scoped and Multi-Sorted Syntax}

\author{Marius Weidner}
\email{weidner@cs.uni-freiburg.de}
\affiliation{%
      \institution{University of Freiburg}
      \country{Germany}}
\orcid{0009-0008-1152-165X} % chktex 8
\authornote{
      I hereby declare that I am the sole author and composer of my thesis and that
      no other sources or learning aids, other than those listed, have been used.\\
      Furthermore, I declare that I have acknowledged the work of others by providing
      detailed references of said work.\\
      I hereby also declare that my thesis has not been prepared for another
      examination or assignment, either wholly or excerpts thereof.
      \\
      \\
      \begin{tabular}{p{\textwidth/2} l}
            Freiburg, XX.XX.2025       & yadda yadda                \\ % \includegraphics[width=0.1\textwidth]{signature.png} \\
            \rule{\textwidth/3}{0.4pt} & \rule{\textwidth/3}{0.4pt} \\
            Place, Date                & Signature
      \end{tabular}
}

\begin{abstract}
      Mechanizing the metatheory of programming languages in proof assistants poses recurring challenges for variable binding and substitution. Every formalization requires proving standard lemmas about substitutions and applying them correctly. Explicit substitution calculi provide syntactic reduction rules that ensure confluence and strong normalization, allowing equations between terms to be compared via normal forms and giving an automatic procedure that eliminates manual lemma applications in most cases.
      Directly embedding these laws as reduction rules in Agda using its rewrite mechanism is challenging due to the need for precise control over which definitions reduce and which remain symbolic, the dependence of substitutions on renamings for structurally recursive definitions, and the need to support multi-sorted languages. We address the latter challenges by adapting the $σ_{SP}$-calculus to intrinsically scoped, multi-sorted syntax, first handling renamings and substitutions separately, then presenting a uniform treatment that allows results about substitutions to be proven once for both, avoiding duplication.

      \subsection*{ABSTRACT IN GERMAN}
      Die Mechanisierung der Metatheorie von Programmiersprachen in Beweisassistenten
      stellt wiederkehrende Herausforderungen bei der Behandlung von Variablenbindung
      und Substitution dar. Jede Formalisierung erfordert den Beweis von
      Standardlemmata über Substitutionen sowie deren korrekte Anwendung. Explizite
      Substitutionskalküle liefern syntaktische Reduktionsregeln, die Konfluenz und
      starke Normalisierung sicherstellen. Dadurch können Gleichungen zwischen Termen
      über Normalformen verglichen werden, was in den meisten Fällen ein
      automatisches Verfahren ermöglicht und die manuelle Anwendung von Lemmata
      überflüssig macht. Die direkte Einbettung dieser Gesetze als Reduktionsregeln
      in Agda mittels seines Rewrite-Mechanismus ist jedoch herausfordernd, da eine
      präzise Kontrolle darüber erforderlich ist, welche Definitionen reduziert
      werden und welche symbolisch bleiben, Substitutionen für strukturell rekursive
      Definitionen von Renamings abhängen und zudem Sprachen mit mehreren Sorten
      unterstützt werden müssen. Wir lösen diese Herausforderungen, indem wir den
      $σ_{SP}$-Kalkül an intrinsisch gescoptes, mehrsortige Syntax anpassen. Zunächst
      behandeln wir Renamings und Substitutionen getrennt, bevor wir eine
      einheitliche Behandlung betrachten, die es erlaubt, Resultate über
      Substitutionen einmalig für beide zu beweisen und so Duplikationen zu
      vermeiden.
\end{abstract}

\maketitle

\section{Introduction}\label{sec:introduction}

Mechanizing the metatheory of programming languages within interactive theorem
provers often presents recurring technical challenges when handeling variable
binding and substitution. While such details are usually ignored in
pen-and-paper proofs, their mechanization in proof assistants requires precise
definitions and a significant amount of manual effort.

Historically, substitution has been treated as an meta-level operation. To
formalize this, explicit substitution calculi CITE were introduced to bring
substitutions down to a syntactic level. These calculi define substitution
behavior through a set of reduction rules. With the $σ_{SP}$-calculus CITE, an
extension of the $σ$-calculus CITE, a sound and complete CITE rewrite system
for $λ$-calculus terms with substitution has been discovered. Reduction in this
system is confluent and strongly normalizing. Repeatedly applying the rules
left-to-right yields a unique normal form, which can then be syntactically
compared for equality.

The $σ$-calculus laws generalize to custom syntaxes without much difficulty,
and it is tempting to assume that this would also make it easy to embed these
laws into proof assistants and then solve equations between terms with
substitution automatically by applying them from left to right, either via
taktics or embedding as native reduction. In practice, however, when we try to
embed the $\sigma$-calculus into Agda, two major challenges remain.

First, we need to distinguish renamings (substitutions mapping variables to
variables) from general substitutions (mapping variables to terms) in order to
obtain structurally recursive definitions~\cite{CITE}. This separation leads to
a code duplication, where key lemmas, such as substitution preserves typing,
must be proven twice, once for renamings and then again for substitutions using
the result about renamings.

Second, in languages with multiple variable sorts, the combination of several
syntactic categories and the separation of renamings and substitutions can lead
to a combinatorial explosion of near-identical substitution definitions and
lemmas.

When choosing a syntax representation, several options are available:
extrinsically scoped and typed, intrinsically scoped but extrinsically typed,
or intrinsically scoped and typed. We argue that the intrinsically scoped but
extrinsically typed approach is the sweatspot between these extremes. In Agda,
using a multi-sorted, intrinsically scoped syntax avoids the combinatorial
explosion of lemmas. As we will see, adapting the σ-calculus to a setting with
multi-sorted, scoped syntax and first-class renamings is not particularly
difficult, but already highly beneficial.

The remaining challenge caused by the separation of renamings and substitutions
is more difficult to address. Several proposals have been made to mitigate this
by unifying their treatment. We build on the notion of Kits~\cite{CITE1}, a
structure that captures the behaviour common to variables and terms. From Kits,
one can define a uniform structure for renamings and substitutions, called
Maps. This idea was later extended to scoped syntax~\cite{CITE2}. Our work
follows the latter developments. We adapt the $σ$-calculus to a setting with
Kits and Maps over scoped syntax, thereby treating renamings and substitutions
uniformly.
\subsection*{Structure}

\begin{itemize}
      \item \cref{sec:pre} introduces the theory and tools used in this work;
      \item \cref{sec:mis} shows how to embed a scoped and multi-sorted version of the $\sigma$-calculus in Agda, treating renamings and substitutions separately;
      \item \cref{sec:ags} explains the internals of the \textsc{Agdasubst} library, where renamings and substitutions can be treated uniformly;
      \item \cref{sec:ext} shows how to use the \textsc{Agdasubst} library and includes case studies;
      \item \cref{sec:rel} compares the \textsc{Agdasubst} library to similar works;
      \item \cref{sec:fwk} outlines possible future improvements to the \textsc{Agdasubst} library;
\end{itemize}

\noindent\textbf{The supplement contains the complete Agda code underlying this paper.}

\section{Preliminaries}\label{sec:pre}
This section introduces the concepts and tools required for the developments in
this work. We summarize the following components:
\begin{itemize}
      \item Lesser known Agda features relevant to this work: \AgdaKeyword{instance}
            arguments, \AgdaKeyword{opaque} blocks, and the \AgdaKeyword{REWRITE}
            mechanism.
      \item Explicit Substitution Calcului;
      \item Multi-Sorted and Scoped Syntaxes in Agda;
\end{itemize}

\subsection{Agda}\label{sec:pre-agd} Agda\footnote{CITE} is a dependently typed programming language and proof
assistant based on Martin-Löf type theory CITE. It allows users to write
programs and proofs in a functional language, verifying correctness through
type checking.

In addition to its well-known features such as indexed and parameterized
inductive data types, dependent pattern matching, implicit arguments and
records, Agda also supports three less commonly used but important features
relevant to this work: \AgdaKeyword{instance} arguments, \AgdaKeyword{opaque}
blocks and \AgdaKeyword{REWRITE} rules.

\subsection*{Instance Arguments}

Instance arguments, written using double braces \verb|{{...}}|, differ from
implicit arguments \verb|{...}| in that they are automatically resolved via an
instance search algorithm, rather than simple implicit argument resolution
limited to the current scope. When a function expects an instance argument,
Agda attempts to infer and insert it from values in scope marked with the
\AgdaKeyword{instance} keyword. This mechanism enables type-directed
programming patterns similar to type classes in Haskell.

As an example, consider the following definition of a type class for types with
a default value: \EDefault{} The record can be opened using instance argument
resolution for its fields: \EDefFields{} We can now declare an instance for
natural numbers: \EDefInst{} This enables implicit usage of the default value
via instance search: \EDefEx{} Here, Agda infers from the type \verb|ℕ| of
\verb|0| that the instance \verb|default–ℕ| applies for \verb|default|,
inserting it automatically to conclude \verb|default ≡ 0| holds definitionally.

\subsection*{\AgdaKeyword{opaque} Blocks}

Agda also supports opaque definitions, which allow certain parts of the code to
be treated as non-unfolding symbols, similar to postulates. This feature is
useful for controlling definitional equality and can be used to prevent
specific definitions from being unfolded during type checking or conversion
checking.

A block is marked as opaque by preceding it with the \AgdaKeyword{opaque}
keyword. All definitions within the block become opaque, meaning they are
excluded from definitional equality checks. However, their names and types
remain visible and can still be used throughout the code. As an example
consider: \EOpaque{}

In this example, \verb|forty-two| is defined as the numeral \verb|42|, but it
is opaque outside the block. This means that although \verb|forty-two| still
has type \verb|ℕ| and can be used in expressions and proofs, it does not reduce
to \verb|5| outside the block during definitional equality checks: \EOpaqueExO

In subsequent opaque blocks, the type checker can be instructed to selectively
unfold previously opaque definitions using the \AgdaKeyword{unfolding} keyword.

Using \AgdaKeyword{unfolding}, the definition will again reduce as expected:
\EOpaqueExT{}

\subsection*{\AgdaKeyword{REWRITE} Rules}

In the context of this work, the most import feature of Agda are user defined
\AgdaKeyword{REWRITE} rules\footnote{CITE} that extend the system’s notion of
definitional equality by integrating them directly into type checking. A
rewrite rule in Agda is a named equation of the form
\[
      \texttt{eq} : (x_1 : A_1)\, \ldots\, (x_k : A_k) \to f\, p_1\, \ldots\, p_n = v
\]
and must satisfy the following \emph{three conditions}\footnote{CITE}:
\begin{enumerate}
      \item The head symbol $f$ must be a \emph{postulate}, a \emph{defined function}, or a
            \emph{constructor} applied to \emph{fully general parameters}, i.e., the
            parameters must be distinct variables.
      \item Each variable $x_1, \ldots, x_k$ must occur at least once in a \emph{pattern
                  position} among $p_1, \ldots, p_n$ (see CITE for the precise definition of
            pattern position).
      \item The left-hand side $f\, p_1\, \ldots\, p_n$ must be \emph{neutral}, i.e., it
            should not reduce any further under existing definitions.
\end{enumerate}

For example: \ERewrite{} The type of the lemma satisfies the conditions above
and can be registered as a rewrite rule using the \AgdaKeyword{REWRITE} pragma:
\ERewriteIt{}

From this point forward, Agda treats \verb|n + 0| as \emph{definitionally
      equal} to \verb|n|: \ERewriteEx

To ensure consistency, Agda requires that all rewrite rules satisfy the three
conditions stated above, and that the combined set of rules is both
\emph{terminating} and \emph{confluent}. However, termination is not currently
checked by Agda; it is only observable indirectly when type checking fails to
terminate. Confluence can be checked using the
\texttt{--local-confluence-check} and \texttt{--confluence-check} flags, but
both checks are necessarily conservative.

\subsection{Explicit Substitution Calculi}\label{sec:pre-sig}

Explicit substitution calculi were introduced to formalize the traditionally
meta-theoretic treatment of substitutions. Unlike the standard approach, where
substitution is defined at the meta-level, these calculi treat substitutions as
syntactic constructs and define their behavior by reduction rules.

Several explicit substitution systems have been proposed. The
$\sigma$-calculus~\cite{CITE3}, the first of its kind, introduces substitution
primitives such as identity, extension, shifting, and composition. The
$\sigma^\Uparrow$-calculus~\cite{CITE4} extends this with a primitive for
lifting.

The $\sigma_{SP}$-calculus is of particular interest because it includes
additional rules to ensure confluence. It is sound, confluent, and
complete~\cite{CITE}. As a result, any valid equality involving substitutions
can be derived using its rules alone.

In the following, we present our own three-sorted version of the
$\sigma$-calculus, using the $\lambda$-calculus as the object language. Our
system closely follows the $\sigma_{SP}$-calculus.

\subsection*{The three-sorted $\sigma$-calculus}
\begin{figure}[t]
      \centering
      \small
      {\raggedright \textbf{\Large Symbols} \par}
      \vspace{1.5em}

      \begin{minipage}[t]{0.48\textwidth}
            \raggedright
            \textbf{Variables} $x$
            \begin{align*}
                  \text{innermost variable} \quad & \textcolor{agdablue}{\mathsf{zero}}    \\
                  \text{outer variables} \quad    & \textcolor{agdablue}{\mathsf{suc}}\; x
            \end{align*}\\
            \vspace{1em}
            \textbf{Terms} $t$
            \begin{align*}
                  \text{variable usage} \quad           & \textcolor{agdablue}{\texttt{\`{}}}x  \\
                  \text{abstraction} \quad              & \textcolor{agdablue}{\lambda.}\, t    \\
                  \text{application} \quad              & t_1 \ t_2                             \\
                  \text{substitution lookup} \quad      & x \textcolor{agdablue}{\;\&\;} \sigma \\
                  \text{substitution application} \quad & t \tdot{} \sigma
            \end{align*}
      \end{minipage}
      \hfill
      \begin{minipage}[t]{0.48\textwidth}
            \raggedright
            \textbf{Substitutions} $\sigma$
            \begin{align*}
                  \text{identity} \quad    & \textcolor{agdablue}{\mathsf{id}}              \\
                  \text{weakening} \quad   & \textcolor{agdablue}{\mathsf{wk}}              \\
                  \text{extension} \quad   & t \textcolor{agdablue}{\cdot} \sigma           \\
                  \text{composition} \quad & \sigma_1 \ \textcolor{agdablue}{;} \  \sigma_2
            \end{align*}
      \end{minipage}

      {\raggedright \vspace{2.0em} \textbf{\Large Reduction} \par}
      \vspace{1.5em}
      \begin{minipage}[t]{0.48\textwidth}
            \raggedright
            \textbf{Definitional Laws}
            \begin{alignat*}{2}
                  \mathsf{id}      \quad          & x \textcolor{agdablue}{\;\&\;} \textcolor{agdablue}{\mathsf{id}}                                          &  & = \textcolor{agdablue}{\texttt{\`{}}} x                    \\
                  \mathsf{wk}      \quad          & x \textcolor{agdablue}{\;\&\;} \textcolor{agdablue}{\mathsf{wk}}                                          &  & = \texttt{\`{}}(\textcolor{agdablue}{\mathsf{suc}} \ x)    \\
                  \mathsf{ext}_\mathsf{Z}  \quad  & \textcolor{agdablue}{\mathsf{zero}} \textcolor{agdablue}{\;\&\;} (t \textcolor{agdablue}{\cdot} \sigma)   &  & = t                                                        \\
                  \mathsf{ext}_\mathsf{S}   \quad & \textcolor{agdablue}{\mathsf{suc}}\;x \textcolor{agdablue}{\;\&\;} (t \textcolor{agdablue}{\cdot} \sigma) &  & = x \textcolor{agdablue}{\;\&\;} \sigma                    \\
                  \mathsf{comp}    \quad          & x \textcolor{agdablue}{\;\&\;} (\sigma_1 \ \textcolor{agdablue}{;} \  \sigma_2)                           &  & = (x \textcolor{agdablue}{\;\&\;} \sigma_1) \tdot \sigma_2
            \end{alignat*}\\
            \vspace{1em}
            \textbf{Traversal Laws}
            \begin{alignat*}{2}
                  \mathsf{var} \quad & (\textcolor{agdablue}{\texttt{\`{}}} x)                                            & \tdot \sigma
                                     & = x \textcolor{agdablue}{\;\&\;} \sigma                                                            \\
                  \mathsf{lam} \quad & (\textcolor{agdablue}{\lambda.}\, t)                                               & \tdot  \sigma
                                     & = \textcolor{agdablue}{\lambda.} (t  \tdot (\textcolor{agdablue}{\uparrow}\sigma))                 \\
                  \mathsf{app} \quad & (t_1 \ t_2)                                                                        & \tdot \sigma
                                     & = (t_1  \tdot \sigma) \ (t_2  \tdot \sigma)
            \end{alignat*}

      \end{minipage}
      \hfill
      \begin{minipage}[t]{0.48\textwidth}
            \raggedright
            \textbf{Interaction Laws}
            \begin{alignat*}{2}
                  \mathsf{comp\text{-}id}_\mathsf{L} \quad & \textcolor{agdablue}{\mathsf{id}} \ \textcolor{agdablue}{;} \  \sigma                                                                                                         &  & = \sigma                                                                                                                                                                                                           \\
                  \mathsf{comp\text{-}id}_\mathsf{R} \quad & \sigma \ \textcolor{agdablue}{;} \  \textcolor{agdablue}{\mathsf{id}}                                                                                                         &  & = \sigma                                                                                                                                                                                                           \\
                  \mathsf{associativity}          \quad    & (\sigma_1 \ \textcolor{agdablue}{;} \  \sigma_2) \ \textcolor{agdablue}{;} \  \sigma_3                                                                                        &  & = \sigma_1 \ \textcolor{agdablue}{;} \  (\sigma_2 \ \textcolor{agdablue}{;} \  \sigma_3)                                                                                                                           \\
                  \mathsf{distributivity}           \quad  & (t \textcolor{agdablue}{\cdot} \sigma_1) \ \textcolor{agdablue}{;} \  \sigma_2                                                                                                &  & = (t \mkern2mu \textcolor{agdablue}{\cdotp} \mkern-2mu \textcolor{agdablue}{\cdotp} \mkern-2mu \textcolor{agdablue}{\cdotp} \sigma_2) \textcolor{agdablue}{\cdot} (\sigma_1 \ \textcolor{agdablue}{;} \  \sigma_2) \\
                  \mathsf{interact}       \quad            & \textcolor{agdablue}{\mathsf{wk}} \ \textcolor{agdablue}{;} \  (t \textcolor{agdablue}{\cdot} \sigma)                                                                         &  & = \sigma                                                                                                                                                                                                           \\
                  \eta\mathsf{\text{-}id}        \quad     & (\textcolor{agdablue}{\texttt{\`{}}\mathsf{zero}}) \textcolor{agdablue}{\cdot} \textcolor{agdablue}{\mathsf{wk}}                                                              &  & = \textcolor{agdablue}{\mathsf{id}}                                                                                                                                                                                \\
                  \eta\mathsf{\text{-}law}       \quad     & (\textcolor{agdablue}{\mathsf{zero}} \textcolor{agdablue}{\;\&\;} \sigma) \textcolor{agdablue}{\cdot} (\textcolor{agdablue}{\mathsf{wk}} \ \textcolor{agdablue}{;} \  \sigma) &  & = \sigma
            \end{alignat*}\\
            \vspace{1em}
            \textbf{Monad Laws}
            \begin{alignat*}{2}
                  \mathsf{right\text{-}id} \quad          & t  \tdot
                  \mathsf{id}                             &           & = t        \\
                  \mathsf{compositionality}         \quad & (t  \tdot
                  \sigma_1)  \tdot
                  \sigma_2                                &           & = t  \tdot
                  (\sigma_1 ; \sigma_2)
            \end{alignat*}
      \end{minipage}
      \caption{The three-sorted $\sigma$-calculus.}
      \label{fig:pre:sig}
\end{figure}

We present a three-sorted version of the $\sigma_{SP}$-calculus in
Figure~\ref{fig:pre:sig}, using the $\lambda$-calculus as the object language.

The syntax is sorted into de~Bruijn variables, terms, and substitution
primitives. Variables~$x$ range over de~Bruijn indices, $t$ over terms, and
$\sigma$ over substitutions.

We use de~Bruijn variables to obtain $\alpha$-equivalence for free. The
expression $\textcolor{agdablue}{\uparrow} \sigma$ is syntactic sugar for
$(\textcolor{agdablue}{\texttt{\`{}}}\textcolor{agdablue}{\mathsf{zero}})
      \textcolor{agdablue}{\cdot} (\sigma \ \textcolor{agdablue}{;} \
      \textcolor{agdablue}{\mathsf{wk}})$. Unlike most presentations in the
literature, we include explicit syntax for variable usage within terms. This
requires distinguishing between \bsym{\textsf{suc}} and \bsym{\textsf{wk}},
depending on whether they appear in the sort of variables or substitutions.

The reduction rules are grouped into four categories:
\begin{itemize}
      \item \textbf{Definitional laws} describe how variables are looked up in substitutions.
      \item \textbf{Interaction laws} express equalities between substitution expressions.
      \item \textbf{Traversal laws} describe how substitutions propagate through term syntax and under binders.
      \item \textbf{Monad laws} describe the behavior of substitutions when applied to terms.
\end{itemize}

We do not include reduction rules for the $\lambda$-calculus itself, such as
$\beta$-reduction, as done for example in the $\lambda\sigma$-calculus [CITE].
Our focus is not on the reduction of the object language, but on the behavior
of substitutions applied to terms. The reduction strategy of the object
language is part of its definition and is left unspecified. The only
syntax-dependent rules we require are those that describe how substitutions
traverse terms.

We believe the three-sorted $\sigma$-calculus is terminating and confluent.
Although we do not give a formal proof, it is a reasonable variant of the
$\sigma_{SP}$-calculus.

Renamings, introduced in the next section, are needed to define substitutions
structurally in Agda. They are believed to break completeness~\cite{CITE}. For
this reason, we do not consider completeness a central property in this work.
Still, we try to preserve as much completeness as possible in what follows.

\subsection{Mutli-Sorted and Scoped Syntaxes}\label{sec:pre-syn}
\begin{figure}[t]
      \centering
      \begin{minipage}[t]{0.48\linewidth}
            \raggedright{}
            \EScoped{}
      \end{minipage}
      \hfill
      \begin{minipage}[t]{0.48\linewidth}
            \raggedright{}
            \EMultiSorted{}
      \end{minipage}
      \caption{Classically Scoped Syntax vs. Multi-Sorted and Scoped Syntax}
      \label{fig:pre:svm}
\end{figure}

Unlike the simply typed $\lambda$-calculus, many object languages have more
than one sort of syntax. For example, System~F has kinds, types and
expressions. Further, to avoid scoping related errors, we want syntaxes to
enforce correct scoping intrinsically. This is usually done in Agda by defining
the syntax as a family of types indexed by its set of free variables.

We start with a standard intrinsically scoped syntax for System~F. Types are
indexed by the number of free type variables, and expressions are indexed by
both free type and expression variables. Variables are represented by de Bruijn
indices, which are numbers less than the number of free variables. This ensures
that only well scoped terms can be formed. The left side of \cref{fig:pre:svm}
shows this version.

This approach is simple but has drawbacks. Types and expressions are separate
datatypes, so we would need to define multiple substitution operations:
type-in-type, expr-in-expr, and type-in-expr. These substitutions need to be
handled and related manually, which adds boilerplate.

To avoid this, we use a multi-sorted syntax. Terms form a single datatype
indexed by a list of variable sorts and a target sort. The type $S \vdash s$
describes a term of sort $s$ with free variables of sorts in $S$. Variable
positions are given by $S \ni s$, a proof that a variable of sort $s$ appears
in $S$ at some position. This position acts as a the de Bruijn index. This
gives us well scoped and well sorted terms by construction. The right side of
\cref{fig:pre:svm} shows this version.

With this unified syntax, we can define generic operations that work uniformly
across sorts. Next, we will use this to our advantage and define scoped and
sorted renamings and substitutions.

\section{The Multi-Sorted and Scoped Sigma Calculus in Agda with First-Class Renamings}\label{sec:mis}

This section gives a direct embedding of the $\sigma$-calculus rules as native
reduction rules in Agda using the \AgdaKeyword{REWRITE} mechanism for the
muti-sorted and scoped version of System~F from \cref{sec:pre:svm} as an
example for any custom syntax.

When adapting the $\sigma$-calculus from \cref{sec:pre-sig} to custom
multi-sorted and scoped syntaxes in Agda, we face the that, in Agda,
substitutions for syntaxes with binders cannot be defined directly by
structural recursion. To obtain well-founded definitions, we need to define
renamings first, which are substitutions that only contain variables.

We revisit the $\sigma$-calculus and adapt its laws to this setting.

\subsection{Scoped and Multi-Sorted Renamings and Substitutions}\label{sec:mis-sub}
\begin{figure}[t]
      \centering
      \begin{minipage}[t]{0.48\linewidth}
            \raggedright{}
            \ERen{}
      \end{minipage}
      \hfill
      \begin{minipage}[t]{0.48\linewidth}
            \raggedright{}
            \ESub{}
      \end{minipage}
      \caption{Scoped and Multi-Sorted Renamings and Substitutions}
      \label{fig:mis:ras}
\end{figure}

We define renamings and substitutions uniformly over all variable sorts by
indexing them over scopes, as shown in \cref{fig:mis:ras}. A renaming from
scope $S_1$ to $S_2$ maps each variable of a given sort in $S_1$ to a variable
of the same sort in $S_2$. Similarly, a substitution maps variables to terms of
the corresponding sort.

The order of definitions is chosen carefully. We first define the renaming type
and its primitive operations: identity, extension and weakening. Substitutions
are then defined in terms of renamings, using weakening to define lifting in a
structurally recursive manner.

To support all composition combinations of renamings and substitutions, we
provide four forward composition primitives: renaming composed with renaming,
renaming with substitution\footnote{The first two composition operators can
      technically be defined as a single definition together with composition of two
      renamings, they are simply forward function composition, but we choose to keep
      them separated.}, substitution with renaming, and substitution with
substitution.

The traversal functions, \texttt{\_⋯ᴿ\_} for renaming and \texttt{\_⋯ˢ\_} for
substitution, are defined by induction on the syntax. Each traversal applies a
renaming or substitution to a term. The clauses in these definitions correspond
directly to the traversal laws of the $\sigma$-calculus, and therefore, we want
these functions to reduce during type checking. When adapting this framework to
arbitrary custom syntaxes, the traversal functions are the only definitions
that must be provided by the user. All other definitions can be stated
independently of the concrete syntax.

All other renaming and substitution primitives are declared \texttt{opaque} in
Agda, which blocks their reduction. This abstraction is essential for
formulating the $\sigma$-calculus laws, where the primitives must act as
uninterpreted symbols rather than as concrete functions. Blocking reduction in
this way allows rewrite rules to apply predictably. The types for renamings and
substitutions are also declared \texttt{opaque} to prevent misuse as actual
functions. Their map-like behavior remains accessible through the
\texttt{\_\&\_}-operator.

\subsection{Scoped and Multi-Sorted Sigma Calculus with First-Class Renamings and Substitutions}\label{sec:mis-sig}
\begin{figure}[t]
      \centering

      \begin{minipage}[t]{0.38\linewidth}
            \raggedright{}
            \EDefLaws{}
            \ERewriteSys{}
      \end{minipage}
      \hfill
      \begin{minipage}[t]{0.58\linewidth}
            \raggedright{}
            \EInteractLaws{}
            \EMonadLaws{}
      \end{minipage}

      % \vspace{1em}
      % \begin{minipage}[t]{0.58\linewidth}
      %       \raggedright{}
      %       \EInteractLaws{}
      % \end{minipage}
      % \hfill 
      % \begin{minipage}[t]{0.38\linewidth}
      %       \raggedright{}
      %       \ERewriteSys{}
      % \end{minipage}
      \caption{The Scoped and Multi-Sorted $\sigma$-calculus with First-Class Renamings and Substitutions.}
      \label{fig:mis-sig}
\end{figure}

We now adapt the rules from \cref{sec:pre-sig} to the setting with renamings.
One might restrict the relevance of renamings to the structural definition of
substitutions and state the laws only for substitutions. However, it is often
useful to reason about renamings directly, for instance in the context morphism
lemma \cite{...}. We therefore treat renamings as first-class citizens.

\Cref{fig:mis-sig} presents the adapted $\sigma$-calculus laws. They all
satisfy the requirements from \cref{sec:pre-agd} to serve as valid rewrite
rules. The definitional laws now involve primitives for both renamings and
substitutions.

The interaction laws likewise cover all operators. Many of these laws are
similar in shape, e.g.\ we require eight associativity laws. Both definitional
and interaction laws can be proved independently of the concrete syntax. In
contrast, the monad laws must be shown by induction over the syntax and are
thus syntax-specific. We highlight this by marking these laws in red throughout
this work.

cite noticed that a further category emerges from the interaction between
renamings and substitutions. The coincidence laws govern the case where
substitutions are in fact renamings in disguise. The coincidence and
coincidence-fold laws allow us to recover renamings from their embedding as
substitutions via the identity substitution. Extracting arbitrary renamings
from deeply nested substitutions would require a dedicated solving strategy,
which we leave for future work, but consider important for a more complete
account.

In the implementation we makes use of function extensionality to prove the
$\sigma$-calculus laws. It is known that this assumption can be avoided, as
shown in \cite{...}.

\subsection{Discussion}\label{sec:mis-dis}
The laws presented in \cref{sec:mis-sig} are a straightforward extension of the
classic $\sigma$-calculus rules to a setting with renamings. They fulfill the
requirements for valid rewrite rules without further modification. This is a
consequence of marking the relevant symbols opaque: inside proofs of the laws
themselves, thedefinitions can still unfold when needed, while externally they
behave as uninterpreted symbols and are reduced only by the adapted
$\sigma$-calculus laws.

When reasoning solely about renamings or substitutions, the laws typically
solve almost all equations between terms automatically. Only when crossing the
boundary between renamings and substitutions, and thus relying on coincidence
laws, additional manual effort might sometimes become necessary. Fortunately,
such cases are rare in practice.

The main drawback of this approach is the unavoidable duplication of similar
definitions and proofs. Associativity laws pose no difficulty, since they can
be stated independently of the syntax, but traversal, monad, and coincidence
laws must be provided by the user. These consist of two traversal lemmas, two
right-identity laws, and four compositionality laws, all similar in shape and
together still contributing to a noticeable amount of boilerplate.

\section{The Agdasubst Library}\label{sec:ags}
This section shows the \textsc{Agdasubst} library. It builds on the ideas from
\cref{sec:mis} but blurs the distinction between variables and terms by
introducing a construct that represents variables \emph{or} terms, which we
call \emph{Kits}. With Kits in place, we can also abstract over renamings and
substitutions by defining renamings \emph{or} substitutions, which we call
\emph{maps}. With Kits and Maps we can unify the definitons and proofs from
\cref{sec:mis} that were similar in shape.

The idea of Kits originates from~\cite{...} and was further developed
in~\cite{...}. We first recall the relevant definitions from these works in
\cref{sec:ags-kit} and \cref{sec:ags-com} and then show how to adapt the
$\sigma$-calculus laws to the setting of Kits. We will omit most implementation
details, except for those directly concerned with the embedding of the
$\sigma$-calculus laws. For a complete account of Kits and their extensions, we
refer the reader to saffrichs work.

In the following we abstract over a set of sorts \ASortParam, a scoped and
multi-sorted syntax \AScopedT \ and variable constructor \AVarCstr.

\subsection{Kits}\label{sec:ags-kit}
A Kit is a structure that abstracts over whether we are dealing with a term or
a variable. The intended usage is to instantiate this structure exactly twice.
Definitions can then be written uniformly, parameterized over a Kit, and are
consequently applicable to both cases. \AKit

To distinguish the only two instances of Kits on the type level, Kits are
indexed over a boolean-like structure \AModeDef.

The \texttt{image} function computes, given a \texttt{Mode}, whether we are
talking about variables or terms: \AImage

A Kit record contains of the following fields:
\begin{itemize}
      \item \textbf{id/‘} converts a variable into an element of \(S ∋/⊢ᴷ s\).
            For the variable Kit, $\_∋/⊢ᴷ\_$ is instantiated to \(\_\ni\_\), so this operation acts as the identity.
            For the term Kit, $\_∋/⊢ᴷ\_$ is instantiated to \(\_⊢\_\), so this operation corresponds to the variable constructor.
      \item \textbf{‘/id} converts an element of \(S ∋/⊢ᴷ s\) into a term \(S ⊢ s\), and is analogous to \textbf{id/‘}.
      \item \textbf{wk} shifts the de Bruijn indices in a variable or term.
            The newly introduced, unused variable \(\text{zero}\) can assume any sort \(s'\).
            For variables, \textbf{wk} is the successor function \(suc\).
            For terms, \textbf{wk} applies a shifting renaming to the term.
\end{itemize}
Because rewrite rules cannot mention record fields directly, we provide for each field a
corresponding version without the $K$-prefix. These are declared as opaque functions, so
they can be used as symbols in rewrite rules, while still unfolding to the actual fields
when needed. Though, inside opaque blocks, we can still use the fields without doing any harm
We assume to have all defintions unfolding all the time throughout this section.
It is only outside the library, when we have the sigma calculus laws in place, when we want all primitives act as uninterpreted symbols.

The Kit record further consists of axioms, that constrain the instantition, for
example, that \AAxiomEx. We will skip over these axioms from now on, since for
both variables and terms all of them hold almost entirely trivially.

We can now define the usual operations for renamings and substitutions. The
definitions are included directly in the record module of \texttt{Kit}, so they
are implicitly parameterized over a kit. All the primitves are marked as
\texttt{opaque}.
\begin{minipage}[t]{0.48\linewidth}
      \raggedright{}
      \AMapA{}
\end{minipage}
\begin{minipage}[t]{0.48\linewidth}
      \raggedright{}
      \AMapB{}
\end{minipage}

The type $S_1 \to_k S_2$ unifies renamings $S_1 \to_r S_2$ and substitutions
$S_1 \to_s S_2$. For $K = V$ the $image$ function computes to the type of
variables and thus the map acts as renaming. Similarly, in the case that $K =
      T$, the Map corresponds to an substitution.

The identity map, lookup, and extension operations behave analogously in both
the renaming and substitution cases.

Since term traversal is not yet available, composition cannot be defined at
this stage. Instead, we introduce a generalized weakening operator $\_;wk$,
which can later be rewritten as a composition with renamings. Similarly, once
term traversal is available, the $id/‘$ and $‘/id$ operators can be expressed
entirely in terms of the primitive operations.

We provide a shorthand for lifting a map:\ALifting{} and for referring to a
specific Kit: \AKitExplicit{}

How a map is applied to a term depends on the concrete object language. The
user therefore has to provide the traversal function and prove manually that
applying the identity map leaves a term unchanged: \ATraversal{}

Equipped with the traversal function we can define the two intended instances
for Kits (axiom proofs omitted):

\begin{minipage}[t]{0.48\linewidth}
      \raggedright{}
      \AInstanceRen{}
\end{minipage}
\begin{minipage}[t]{0.48\linewidth}
      \raggedright{}
      \AInstanceSub{}
\end{minipage}

In the $K-wk$ field for the term Kit, we use the $wk$ operation from the
variable Kit. The kit instance for that is automatically resolved using
instance search.

We can now turn our attention to the composition of maps.

\subsection{Composition}\label{sec:ags-com}
There are four possible compositions between renamings and substitutions. A
ComposeKit $\ K_1\ K_2\ (K_1 \sqcup K_2)$ provides the operation necessary to
define a single, more general composition of a $K_1$-map with a $K_2$-map,
resulting in a $(K_1 \sqcup K_2)$-map: \AComposeKit

The operation $C-\_\&/⋯\_$ unifies lookup and traversal. It takes a variable or
term $x/t$ (according to $K_1$) and a renaming or substitution $\varphi$
(according to $K_2$), and applies $\varphi$ to $x/t$, yielding a variable or
term of type $K_1 \sqcup K_2$. From this operation (for which we again provide
an opaque function version without the $C$ prefix) we derive map composition:
\ACompDef{} The map combination operator is implicitly parameterized over a
Compose Kit. It takes an renaming or substitution $\varphi_1$ (according to
$K_1$) and a renaming or substitution $\varphi_2$ (according to $K_2$) and
composes it to a variable or term of type $K_1 \sqcup K_2$. Depending on
whether $x \& \varphi_1$ yields a variable or term we need to lookup in or
traverse with $\varphi_2$.

We also define versions of both the unified lookup–traversal operator and the
composition operator where the Compose Kit is given explicitly: \ACompExp \ and
\ALoAExp.

As with Kits, we now require a user-provided proof about traversal. We need the
user to show that the chosen syntax and traversal function satisfy
compositionality. This proof can only be established by induction on the
syntax: \ACompositionality

Finally, to cover the four possible instances of Compose Kits, we provide two
generic instances that suffice (axiom proofs omitted):

\begin{minipage}{0.38\linewidth}
      \raggedright{}
      \AInstanceCRen{}
\end{minipage}
\begin{minipage}{0.58\linewidth}
      \raggedright{}
      \AInstanceCSub{}
\end{minipage}

$C^R$ is the Compose Kit between a renaming and another Kit $K$.
$C^S$ is the Compose Kit between a substitution and another Kit $K$, and assumes that composition of a $K$-map with a renaming is already available.

The third parameter $K_1 \sqcup K_2$ of a Compose Kit can be understood as a
functional dependency, uniquely determined by the choice of $K_1$ and $K_2$. It
evaluates to $V$ if and only if both $K_1$ and $K_2$ are $V$, otherwise it is
$T$. In other words, only renamings compose to a renaming, while every other
case yields a substitution. We can put this more algebraically: if $\varphi_1$
is a $K_1$-map and $\varphi_2$ is a $K_2$-map, then the result is a $(K_1
      \sqcup K_2)$-map, where $\sqcup$ refers to the join-semilattice on $\{V, T\}$
generated by $V < T$. One might ask why we do not simply compute $K_1 \sqcup
      K_2$. One might ask why we do not simply compute $K_1 \sqcup K_2$. This is in
fact possible and even necessary to define the $\sigma$-calculus rules in a way
that still adheres to the three restrictions on the shape of rewrite rules.

\subsection{Structural Kit Order}\label{sec:ags-sko}
We can easily define a least upper bound operation on Modes that mirrors the
behaviour of composing two maps with these Modes: \AModeLub{} By
straightforward case analysis, all the join-semilattice laws are verified:
\AModeLubLaws{} These laws can be embedded as reduction rules in Agda using
\AgdaKeyword{REWRITE}, except for $\bigsqcup$-comm, which is evidently
non-terminating.

We then lift this operation from Modes to Kits themselves, using the least
upper bound on Modes in the type of the Kit: \AKitLub{} The same laws that hold
for Modes, also hold for Kits: \AKitLubLaws{} Note that the $\sqcup$-comm law
requires an explicit substitution in its type, since we cannot rewrite
$\bigsqcup$-comm for Modes. All other laws are again added as rewrite rules.

The Kit least upper bound laws follow by case distinction on their Modes. When
one inspects a proof in detail, we get into trouble. As an example, consider
$\sqcup$-idem. The remaining laws follow in the same way: \AKitLubExcerpt{} If
we split on the Mode $M$ of the Kit $K$, we must prove in each case that $K$ is
in fact $V$ or $T$, according to the definition of $\sqcup$ and the chosen
Mode. Here we run into a problem: $K$ is inferred by instance resolution, and
Agda cannot deduce that there is only one possible choice ($V$ for the Mode
$Vᴹ$ and $T$ for the Mode $Tᴹ$). Nor can the uniqueness of instances be assumed
in general, since external code might declare additional instances.

To resolve this, we restrict the Kit instances by requiring a field of type
\ALock \ in the construction of a Kit: \ALockField{}. Because $Lock$ is
private, it can only be constructed from within the library.

We can now postulate the uniqueness of Kits after manually verifying that only
two instances exist and that we rely solely on Kits provided by instance
resolution, never creating them locally: \AUniqueKits{}. There also exists a
version \AgdaFunction{unique–K} that computes whether a given $K$ is
\AgdaFunction{V} or \AgdaFunction{T}, based on its Mode.

\subsection{Computing Compose Kits}\label{sec:ags-cck}
Our original goal was to compute the third parameter of a
\AgdaDatatype{ComposeKit}. To motivate this and, more generally, the
computation of \AgdaDatatype{ComposeKit}s from \AgdaDatatype{Kit}s, we consider
the associativity law of the $\sigma$-calculus, which in our Kit-based setting
corresponds to the associativity of map composition: \AAssocTryO{}

Here, $C_1$ to $C_4$ denote the four \AgdaDatatype{ComposeKit}s required for
the compositions. This yields the most general formulation of the associativity
law for Kits, where arbitrary Kits are admitted as long as both sides reduce to
a map determined by $K_5$.

This law, however, is not a valid rewrite rule. Both $C_3$ and $C_4$, required
for the right-hand side, appear only there, violating the condition that all
arguments of a rewrite rule must occur on the left. Moreover, the law requires
the instance argument $K_6$, used as the result of composing the map determined
by $K_2$ with that of $K_4$, which is only required on the right. Since $K_6$
can be computed via the least upper bound operation defined earlier, the main
difficulty is to compute $C_3$ and $C_4$.

One may attempt to define a function \ACompCKitSafe{} for this purpose. This
function can be readily defined, but its correctness relies not only on the
uniqueness of \AgdaDatatype{Kit} instances, but also on the uniqueness of
\AgdaDatatype{ComposeKit} instances:\AUniqueCKits{} As before we reqiure a
value of \AgdaDatatype{Lock} in the creation of Compose Kits. By manual
inspection we confirm that $C^R$ and $C^S$ are the only instances and that none
are created locally.

Yet the function $\AgdaFunction{\_;ᴷ\_}$ is insufficiently general. Computing
$C_4$ in the associativity law yields $C_4 = K_2 \; \AgdaFunction{;ᴷ} \; K_4 =
      \AgdaDatatype{ComposeKit}\; K_2\; K_4\; (K_2 \sqcup K_4)$. Consequently, $C_3$
evaluates to $ C_3 = K_1 \; \AgdaFunction{;ᴷ} \; (K_2 \; \AgdaFunction{⊔} \;
      K_4) = \AgdaDatatype{ComposeKit}\; K_1\; (K_2 \sqcup K_4)\; (K_1 \sqcup (K_2
      \sqcup K_4))$. The final Map on right-hand side is then determined by $K_1
      \sqcup (K_2 \sqcup K_4)$, whereas the left-hand side is determined by $K_5$.
Adjusting the $K_5$ parameter in $C_1$ or $C_2$ to the result Kit of $C_4$ is
not possible, since ComposeKits whose third parameter involves least upper
bounds would not be inferrable by instance resolution.

We require an even more general, though unsafe, function: \ACKitUnsafe{} Since
it is declared \AgdaKeyword{NON\_COVERING}, evaluation may get stuck, yielding
a value for a Compose Kit that should not exist. Thus, every use of this
function requires manual verification. Computed Compose Kits are reduced to the
actual instances, when applicable, using the rewrite rules \ACKitRenRed{} and
\ACKitSubRed{}. This allows us to state a version of the associativity law
eligible for rewriting: \AAssocTryT{}

We can now connect $K_1 \sqcup (K_2 \sqcup K_4)$ with $K_5$ in the outer
Compose Kit on the right-hand side. It remains to check that whenever the
incoming Compose Kits are valid, the outgoing one is as well. As an
illustrative case, if $K_1 = .. = K_5 = \AgdaFunction{V}$, then all Maps are
renamings and the outgoing Compose Kits correspond to compositions of
renamings. By exhaustively checking all $8$ valid combinations of $C_1$ and
$C_2$, we confirm that in each case the outgoing Compose Kits are valid too.
Inside the associativity proof, the remaining cases must be refuted under the
assumption of valid Compose Kits and the non-existance of invalid Compose Kits:
\AUnqiueCKitsImp{} As before, each use of this postulate requires manual
verification to avoid inconsistencies.

\subsection{The Scoped and Multi-Sorted $\sigma$-Calculus with Kits and Maps}\label{sec:ags-rls}

\begin{figure}[t]
      \centering
      \small
      {\raggedright \textbf{\Large Symbols} \par}
      \vspace{1.5em}

      \begin{minipage}[t]{0.48\textwidth}
            \raggedright{}
            \textbf{Sorts} $s$ \\
            \ESortTy{}\\
            \vspace{0.5em}
            \emph{.. object language sorts ..}\\
            \vspace{0.5em}
            \textbf{Scopes} $S$ \\
            \EScopeDefTy{}\\
            \vspace{0.5em}
            \EScopeDef{}\\
            \vspace{0.5em}
            \textbf{Variables} $x$ \\
            \EVarsTy{}\\
            \vspace{0.5em}
            \EVars{}\\
            \vspace{0.5em}
            \textbf{Terms} $t$ \\
            \ETmC{}\\
            \vspace{0.5em}
            \EVarC{}\\
            \emph{.. object language  ..}\\
            \vspace{1.0em}
      \end{minipage}
      \hfill
      \begin{minipage}[t]{0.48\textwidth}
            \raggedright{}
            \textbf{Kits} $K$ \\
            \EKitDefTy{}\\
            \vspace{0.5em}
            \EKitDef{}\\
            \vspace{0.5em}
            \textbf{Maps} $ϕ$\\
            \EPrimsTy{}\\
            \vspace{0.5em}
            \EPrims{}\\
            \vspace{0.5em}
            \textbf{Variables or Terms} $x/t$\\
            \EVarTrmTy{}\\
            \vspace{0.5em}
            \EVarTrmApp{}
      \end{minipage}

      {\raggedright \vspace{2.0em} \textbf{\Large Reduction} \par}
      \vspace{1.5em}
      \begin{minipage}[t]{0.40\textwidth}
            \raggedright
            \textbf{Kit Laws}\\
            \ETypeLevel{}\\
            \vspace{0.5em}
            \textbf{Definitional Laws}\\
            \vspace{0.5em}
            \EDefLawTy{}\\
            \vspace{0.5em}
            \EDefLaw{}\\
            \vspace{0.5em}
            \textbf{Definitional Composition Specialization Laws}
            \ESpecialDefLaws{}
      \end{minipage}
      \hfill
      \begin{minipage}[t]{0.54\textwidth}
            \raggedright
            \textbf{Interaction Laws}\\
            \EInteraction{}\\
            \vspace{0.5em}
            \textbf{Traversal Laws}\\
            \ETravL{}\\
            \emph{.. object language traversal ..}\\
            \vspace{0.5em}
            \textbf{Monad Laws}\\
            \EMonad{}
            \vspace{0.5em}
            \textbf{Coincidence Laws}\\
            \ECoincidence{}
      \end{minipage}
      \caption{The Scoped and Multi-Sorted $\sigma$-Calculus with Kits and Maps.}
      \label{fig:ags:fin}
\end{figure}

We are now ready to state the $\sigma$-calculus laws in the Kit-setting.
\cref{fig:ags:fin} lists the symbols and laws as they are used in
\textsc{Agdasubst}. These are not the formulations from the library itself. For
readability, we omit the Kit and ComposeKit instance arguments and instead give
them explicitly where they are used. Kits and ComposeKits carry data in the
library, and Kits must be disambiguated on the type level by Tags. Here we
simply assume that Kits act as abstractions over variables and terms, and that
ComposeKits serve as markers for admissible compositions. This allows us to
elide the mode lattice laws and obtain a clearer view of the $\sigma$-calculus
laws.

The symbols are grouped into several categories. Sorts and Scopes provide
intrinsic scoping at the type level. Variables are supplied by the library,
while terms are user-defined, subject only to the requirement of a variable
constructor in the appropriate form. We define the type for Kits and the
variable and term Kit, together with their least upper bound operation. We then
define the type of Maps and the four map primitives, parameterized by Kits. We
only specialize \AgdaFunction{wk} directly to renamings in order to avoid
duplication, since it can always be embedded as a substitution by composing
with the identity substitution. When the Kit in the type of Maps is known to be
$V$ or $T$, we write $\rho$ and $\sigma$, respectively. From Kits we further
obtain the combined type of variables-or-terms, together with the lookup or
traversal operator.\footnote{While there are separate type definitions for
      variables, terms, and variables-or-terms, there is only one type definition for
      Maps, specialized via the Kit index. There is no technical reason not to unify
      variables and terms in the same way. However, unlike renamings, substitutions,
      and maps, terms are user-supplied, whereas variables are library-defined and
      thus qualify for a distinct type definition.}%%

We already discussed the \textbf{Kit Laws} in \cref{sec:ags-sko}, so we now
turn to the remaining ones. We begin with the type-level reduction laws
\AgdaFunction{imgⱽ} and \AgdaFunction{imgᵀ} inside the \textbf{Definitional
      Laws}, which rewrite the type of variables-or-terms to either the type of
variables or the type of terms when the Kit index is known to be $V$ or $T$,
respectively.

Next comes the definition of lookup on the identity substitution. It is given
in terms of the unified lookup/traversal operator. In fact, we never use the
specialized lookup or traversal operators at all.\footnote{ We considered
      keeping the specialized operators, but this would require duplicating many
      laws, once for the specialized operators and once for the general one. There
      are certainly situations where such duplication could help, but in our case
      studies, with the addition of the definitional composition specialization laws,
      we had no need for laws specialized to lookup or traversal.} Identity renaming
ddoes not require its own law, since it is already covered by the now
generalized \AgdaField{right-id} from the \textbf{Monad Laws}, which handles
both a term traversed by an identity substitution and a variable looked up in
the identity renaming.

The laws \AgdaFunction{wk}, \AgdaFunction{ext₀}, and \AgdaFunction{extₛ} are
the straightforward generalizations of their familiar counterparts for renaming
and substitution.

The law \AgdaFunction{comp}, however, is more subtle. As shown in
\cref{fig:ags:fin}, it only reduces when both $\varphi_1$ and $\varphi_2$ are
renamings. This restriction is necessary because of a fundamental duality
between the definition of composition (via the unified lookup/traversal
operator) and the compositionality law in full generality:

\begin{minipage}{0.48\linewidth}
      \raggedright{}
      \ECompGeneral{}
\end{minipage}
\begin{minipage}{0.48\linewidth}
      \raggedright{}
      \ECompoGeneral{}
\end{minipage}

Viewed side by side, these laws are exact duals. If both were admitted into the
rewrite system, they would trigger infinite loops. This is no accident:
intuitively, one law pushes variables into maps, while the other pushes maps
into other maps, away from terms.

To avoid nontermination, we adopt the following heuristic:
\begin{itemize}
      \item \textbf{Variable position ($K_5 = V$):} use the definition of
            composition. In this case all other indices $K_1,\dots,K_4$ are forced
            to be $V$, and $x/t$ must be a variable $x$. This yields exactly the
            specialized \AgdaFunction{comp} law shown in the figure.
      \item \textbf{Term position ($K_5 = T$):} use \AgdaField{compositionality}. Here no
            restrictions on the other indices arise, so the law is preserved in
            most of its generality and appears under the  \textbf{Monad Laws}.
\end{itemize}

\noindent If neither condition applies (i.e.\ we do not know whether we are dealing with
a variable or a term), the user must explicitly choose which law to apply,
either by equational reasoning or by using \AgdaKeyword{rewrite}.

With this heuristic the \AgdaFunction{comp} law is rather restricted. We now
miss cases where we would make progress by applying this law, when $\varphi_1$
and $\varphi_2$ are not both renamings. Hence, we consider all situations where
progress would be made directly after applying \AgdaFunction{comp}, by
specializing on $\varphi_1$ or $\varphi_2$. These appear under the name
\textbf{Definitional Composition Specialization Laws} in \cref{fig:ags-fin}.

The \textbf{Interaction Laws} are the straightforward generalizations of their
original $\sigma$-calculus counterparts to Kits. Next come the
\textbf{Traversal Laws}. Since we are independent of the object syntax except
for the required variable constructor, we only show the law specific to the
traversal of a variable. When a variable constructor is applied to a map, the
variable it holds is mapped accordingly. To turn the result back into a term,
as it may otherwise be either a variable or a term, depending on the map, we
subsequently apply the identity substitution.

To illustrate another case for traversal, consider an object language that
supports $\lambda$-abstractions. The corresponding user-supplied traversal
function, needed to construct a Kit, would then include a clause \FSLambdaExT.
However, this clause does not directly corresponds to a traversal law. It must
first be lifted from the traversal operator to the more general
lookup/traversal operator: \FSLambdaExL. This equality can be proven by
reflexivity, when we unfold the definition of the unfiied lookup/traversal
operator. It can then be added as a rewrite rule.

Finally, the \textbf{Coincidence Laws} generalize from renamings to maps and
allow any map to be pulled out of a substitution when feasible. These laws are
again not a complete account of all possible ways to pull maps out of
substitutions. The search for more sophisticated laws that cover additional
cases where this can be done would certainly be worthwhile.

\subsection{Discussion}
The generalization of the $\sigma$-calculus laws requires unsafe computation
and a careful formulation of the laws so that they remain valid rewrite rules
while still allowing the inference of Kits and ComposeKits to be offloaded to
the instance resolution algorithm. The key difficulty in adapting the laws to
the Kit-setting stems from the duality between the general versions of the
\AgdaFunction{comp} and the \AgdaField{compositionality} laws. Our solution is
a heuristic that resolves the conflict by restricting the laws to cases where
it is known whether we are dealing with a term or a variable. In particular,
when this distinction is unclear, the direction in which the laws should “flow”
becomes ambiguous. In such situations, the user must resort to manual
application of the laws. In contrast to the laws from \cref{sec:mis-sig}, the
Kit-based laws allow us to establish properties directly for Maps, without the
need to first carry out separate but structurally similar proofs for renamings
and substitutions.

\section{Case Studies}\label{sec:ext}
We will discuss three examples for the usage of the \textsc{Agdasubst} library.
The first two is a proof of subject reduction for System F with and without
subtyping. We first discuss the proof, the laws from \cref{sec:mis-sig} and
then with the Kit-based approach from \cref{sec:ags-rls} The secobd is the
based on CITE and gives an instantition for generic universe of scoped syntax.
\subsection{SystemF}\label{sec:ext-syf}

\subsection{SystemF with Subtyping}\label{sec:ext-sfs}

\subsection{Generic Syntax}\label{sec:ext-uni}
CITE
% generic syntax allows to prove substitution lemmas only once. 
% sometimes inconvenient to use in agda
% generics and 
% typing kits
% room for more! e.g. linear types / resources etc or semantics e.g. confluence

% \section{Case Studies}\label{sec:cas}
% \subsection{System F}\label{sec:cas-sf}
% \cite{girard1972}
% \subsection{System F with Subtyping}\label{sec:cas-sfsub}
% 
% \subsection{Martin-Löf Type Theory}\label{sec:cas-mltt}

\section{Related Work}\label{sec:rel}
Most of this work is inspired by and builds upon \textsc{Autosubst~2}~\cite{CITE}. 
\textsc{Autosubst}, like \textsc{Agdasubst}, uses the laws of the $\sigma$-calculus to automatically solve equations between terms with substitutions in the Coq proof assistant. 
With version~2, \textsc{Autosubst} also introduced first-class renamings, making their law set comparable to the one presented in \cref{sec:mis-sig}. 
Since Coq is more tactic-oriented, \textsc{Autosubst} provides a tactic to apply these laws, whereas in Agda we opted to embed the laws natively via the rewrite mechanism to achieve definitional equalities. 
Additionally, \textsc{Autosubst} supports code generation of syntax-dependent functions and laws from a Higher-Order Abstract Syntax (HOAS) definition, while we currently have no such counterpart. If we add code generation in the future we envision a solution use reflection within Agda instead of relying on external code generation.
Unlike our approach, \textsc{Autosubst} does not provide an abstraction similar to Kits for unifying proofs about renamings and substitutions.

The idea of Kits was originally proposed by CITE in an intrinsically typed setting, 
which is also reflected in CITE. 
The latter also experiments with a uniform treatment of renamings and substitutions. 
CITE developed a framework based on the Kit concept for scoped syntax, 
but without providing any strategy for automatically solving equations between terms with substitutions. 
Their focus was instead on abstracting over typing relations and the semantics of the object language, 
while providing a uniform treatment of renamings and substitutions in that context. 
These results apply directly to \textsc{Agdasubst}, as demonstrated in \cref{sec:cas}, 
where we used typing Kits from this work. 
Our laws integrate seamlessly with these extension Kits and substantially simplify the proofs required for their instantiation.

There already exists another Agda library with a similar goal to \textsc{Agdasubst}, called
\textsc{Abstract Binding Trees}, which is, for example, used in CITE. 
It allows one to construct an object language from a generic syntax, similar to 
\cref{sec:ext-uni}, but in contrast it supports only extrinsic scoping, not intrinsic. 
Rather than focusing on specific relations, for instance the typing relation, as in CITE, it abstracts over arbitrary predicates about the language. 
The library also offers experimental support for rewriting, but the current state of this feature is unknown to us. 
In any case, rewriting is only supported in the style of treating renamings and substitutions separately, 
as in \cref{sec:mis-sig}, rather than in a uniform threatement.

More distantly related is \textsc{Nominal 2}~\cite{CITE}, 
an Isabelle package that addresses the $\alpha$-equivalence problem not via de Bruijn indices, 
as in previous work, but using so-called nominal datatypes and freshness conditions. 
This approach is closer to the intuitive notion of variables, but the tool does not provide 
any strategy for solving equations involving terms with substitutions. 

Additionally, there are tools such as \textsc{Needle and Knot}~\cite{CITE}, 
which is a code generator for unscoped syntax using de Bruijn indices. 
This tool generate substitution and interaction lemmas for single-pointed substitutions 
in languages that feature multiple variable sorts and binders capable of binding lists of variables.

\section{Further Work}\label{sec:con-wrk}
This section outlines possible directions for improving \textsc{Agdasubst} in
the future.

\subsection*{Reflection}
In comparison to other libraries discussed in \cref{sec:rel}, users of
\textsc{Agdasubst} still need to manually provide the definitions for
traversal, the proofs of \AgdaField{right-id} and \AgdaField{compositionality},
as well as the lifting of traversal to the more general operator.
\citeauthor{CITE} have shown that these definitions can be derived
automatically using Agda's reflection mechanism. There is no reason why the
same approach could not be applied here.

\subsection*{Extensions}
\citeauthor{CITE} introduced several interesting extensions to the $\sigma$-calculus that could
also be added to \textsc{Agdasubst}. One example is support for variadic binders,
that is, binders of a variable or unknown number of variables, which are useful for languages with
pattern matching. Another possible extension is support for lists of terms, allowing a variable or
unknown number of terms in the syntax type. This would, for instance, enable support for
languages with $n$-ary application.

\subsection*{Meta Theory}
In its current state, \textsc{Agdasubst} completely lacks a meta-theoretical
analysis. While we believe it to be terminating and confluent, it would be
interesting to formally establish these properties.

\subsection*{Completeness}
As mentioned, \textsc{Agdasubst} is not complete once we move beyond reasoning
about only renamings or substitutions. As soon as we cross the boundary between
these two worlds, by considering more general maps or Kits, we definitely
cannot solve all equations. It would therefore be worthwhile to investigate
additional laws, similar in spirit to the coincidence laws, to increase the
space of solvable equations.

\subsection*{Performance}
At the moment, the performance of \textsc{Agdasubst} is, quite frankly, very
poor. For instance, the proof of subject reduction for System~F with subtyping
requires several minutes to check. Our analysis shows that most of this time is
spent on instance resolution and function coverage analysis (i.e., verifying
that pattern matching clauses are exhaustive). Currently, we have no concrete
strategy to improve these aspects, except that we know, that the addition of
the coincidence laws dramatically increases type-checking time.

\section{Conclusion}\label{sec:con}


\bibliographystyle{ACM-Reference-Format}
\bibliography{references}

\end{document}
