
\documentclass[a4paper,UKenglish,cleveref, autoref, thm-restate]{lipics-v2021}
\usepackage{microtype}
\usepackage{amsmath,amssymb}
\usepackage{stmaryrd}
\nolinenumbers

\usepackage{agda}

\usepackage{dsfont}
\usepackage{newunicodechar}
\newunicodechar{λ}{\ensuremath{\mathnormal\lambda}}
\newunicodechar{σ}{\ensuremath{\mathnormal\sigma}}
\newunicodechar{τ}{\ensuremath{\mathnormal\tau}}
\newunicodechar{π}{\ensuremath{\mathnormal\pi}}
\newunicodechar{ℕ}{\ensuremath{\mathbb{N}}}
\newunicodechar{∷}{\ensuremath{::}}
\newunicodechar{≡}{\ensuremath{\equiv}}
\newunicodechar{≅}{\ensuremath{\cong}}
\newunicodechar{∀}{\ensuremath{\forall}}
\newunicodechar{ᴸ}{\ensuremath{^L}}
\newunicodechar{ᴿ}{\ensuremath{^R}}
\newunicodechar{ʳ}{\ensuremath{^r}}
\newunicodechar{ᶻ}{\ensuremath{^Z}}
\newunicodechar{ₗ}{\ensuremath{_l}}
\newunicodechar{ⱽ}{\ensuremath{^V}}
\newunicodechar{⟧}{\ensuremath{\rrbracket}}
\newunicodechar{⟦}{\ensuremath{\llbracket}}
\newunicodechar{⊤}{\ensuremath{\top}}
\newunicodechar{⊥}{\ensuremath{\bot}}
\newunicodechar{₁}{\ensuremath{_1}}
\newunicodechar{₂}{\ensuremath{_2}}
\newunicodechar{₃}{\ensuremath{_3}}
\newunicodechar{₄}{\ensuremath{_4}}
\newunicodechar{₅}{\ensuremath{_5}}
\newunicodechar{₆}{\ensuremath{_6}}
\newunicodechar{₇}{\ensuremath{_7}}
\newunicodechar{₈}{\ensuremath{_8}}
\newunicodechar{₉}{\ensuremath{_9}}
\newunicodechar{∈}{\ensuremath{\in}}
\newunicodechar{₀}{\ensuremath{_0}}
\newunicodechar{′}{\ensuremath{'}}
\newunicodechar{ˢ}{\ensuremath{^S}}
\newunicodechar{ᴬ}{\ensuremath{^A}}
\newunicodechar{∘}{\ensuremath{\circ}}
\newunicodechar{𝟙}{\ensuremath{\mathds{1}}}  
\newunicodechar{𝟘}{\ensuremath{\mathds{O}}}
% \newunicodechar{𝟙}{\ensuremath{\mathbb{I}}}  
% \newunicodechar{𝟘}{\ensuremath{\mathbb{O}}}
\newunicodechar{ᴾ}{\ensuremath{^P}}
\newunicodechar{ᵀ}{\ensuremath{^T}}
\newunicodechar{⊎}{\ensuremath{\uplus}}
\newunicodechar{ι}{\ensuremath{\iota}}
\newunicodechar{⇐}{\ensuremath{\Leftarrow}}
\newunicodechar{⇒}{\ensuremath{\Rightarrow}}
\newunicodechar{➙}{\ensuremath{\to^P}}
\newunicodechar{Δ}{\ensuremath{\Delta}}
\newunicodechar{∅}{\ensuremath{\emptyset}}
\newunicodechar{⁺}{\ensuremath{^+}}
\newunicodechar{𝕏}{\ensuremath{\mathbb{X}}}
%\newunicodechar{·}{\ensuremath{\cdot}} %seems to be defined already!
\newunicodechar{∙}{\ensuremath{\cdot}}
\newunicodechar{⁇}{\ensuremath{?}}
\newunicodechar{‼}{\ensuremath{!}}
\newunicodechar{⊕}{\ensuremath{\oplus}}
\newunicodechar{ℤ}{\ensuremath{\mathbb{Z}}}
\newunicodechar{μ}{\ensuremath{\mu}}
\newunicodechar{∃}{\ensuremath{\exists}}
\newunicodechar{;}{\ensuremath{\fatsemi}}
\newunicodechar{Σ}{\ensuremath{\Sigma}}
\newunicodechar{ᴿ}{\ensuremath{^R}}
\newunicodechar{ᵢ}{\ensuremath{_i}}
\newunicodechar{≢}{\ensuremath{\nequiv}}
\newunicodechar{≟}{\ensuremath{\stackrel{{\tiny?}}{=}}}
\newunicodechar{≤}{\ensuremath{\le}}
\newunicodechar{ᵇ}{\ensuremath{^b}}
\newunicodechar{𝓣}{\ensuremath{\mathcal{T}}}
\newunicodechar{𝓔}{\ensuremath{\mathcal{E}}}
\newunicodechar{Γ}{\ensuremath{\Gamma}}
\newunicodechar{γ}{\ensuremath{\gamma}}
\newunicodechar{⊔}{\ensuremath{\sqcup}}
\newunicodechar{α}{\ensuremath{\alpha}}
\newunicodechar{η}{\ensuremath{\eta}}
\newunicodechar{ω}{\ensuremath{\omega}}
\newunicodechar{◁}{\ensuremath{\lhd}}
%PT: seems to be already defined
%\newcommand{\lambdabar}{{\mkern0.75mu\mathchar '26\mkern -9.75mu\lambda}}
\newunicodechar{ƛ}{\ensuremath{\lambdabar}}
\newunicodechar{Λ}{\ensuremath{\mathnormal\Lambda}}
\newunicodechar{ρ}{\ensuremath{\rho}}
\newunicodechar{𝓖}{\ensuremath{\mathcal{G}}}
\newunicodechar{ℓ}{\ensuremath{\ell}}
\newunicodechar{♯}{\ensuremath{\sharp}}
\newunicodechar{⇓}{\ensuremath{\Downarrow}}
\newunicodechar{𝓥}{\ensuremath{\mathcal{V}}}
\newunicodechar{∧}{\ensuremath{\wedge}}
\newunicodechar{ₛ}{\ensuremath{_s}}
\newunicodechar{χ}{\ensuremath{\chi}}
\newunicodechar{⊨}{\ensuremath{\models}}
\newunicodechar{⦂}{\ensuremath{\mathbf{:}}}
\newunicodechar{ς}{\ensuremath{\varsigma}}
\newunicodechar{𝓓}{\ensuremath{\mathcal{D}}}
\newunicodechar{∎}{\ensuremath{\square}}
\newunicodechar{■}{\ensuremath{\blacksquare}}
\newunicodechar{↠}{\ensuremath{\twoheadrightarrow}}
\newunicodechar{↪}{\ensuremath{\hookrightarrow}}
\newunicodechar{ⁱ}{\ensuremath{^i}}
\newunicodechar{Π}{\ensuremath{\Pi}}
\newunicodechar{∋}{\ensuremath{\ni}}
\newunicodechar{∍}{\ensuremath{\ni}}
\newunicodechar{‵}{\ensuremath{`}}
\newunicodechar{⨆}{\ensuremath{\bigsqcup}}
\newunicodechar{δ}{\ensuremath{\delta}}
\newunicodechar{κ}{\ensuremath{\kappa}}
\newunicodechar{⦃}{\ensuremath{\{\!|}}  % \lBrace from stix
\newunicodechar{⦄}{\ensuremath{|\!\}}}  % \rBrace from stix
\newunicodechar{⌊}{\ensuremath{\lfloor}}
\newunicodechar{⌋}{\ensuremath{\rfloor}}
\newunicodechar{𝟎}{\ensuremath{\mathbf{0}}}
\newunicodechar{𝟏}{\ensuremath{\mathbf{1}}}
\newunicodechar{𝟐}{\ensuremath{\mathbf{2}}}
\newunicodechar{β}{\ensuremath{\beta}}
\newunicodechar{≥}{\ensuremath{\geq}}
\newunicodechar{ₒ}{\ensuremath{_o}}
\newunicodechar{∊}{\ensuremath{\in}}
\newunicodechar{;}{\ensuremath{;}}
\newunicodechar{⋯}{\ensuremath{\mkern2mu \cdotp \mkern-2mu \cdotp \mkern-2mu \cdotp \mkern1mu }}
\newunicodechar{⊢}{\ensuremath{\vdash}}
\newunicodechar{∶}{\ensuremath{:}}
\newunicodechar{★}{\ensuremath{\star}}
\newunicodechar{ᴷ}{\ensuremath{^K}}
\newunicodechar{ϕ}{\ensuremath{\varphi}}
\newunicodechar{ᴮ}{\ensuremath{^B}}
\newunicodechar{+}{}
\newunicodechar{ᵣ}{\ensuremath{_r}}
\newunicodechar{ₓ}{\ensuremath{_x}}
\newunicodechar{ₜ}{\ensuremath{_t}}
\newunicodechar{𝓡}{\ensuremath{\mathcal{R}}}
\newunicodechar{𝓢}{\ensuremath{\mathcal{S}}}
\newunicodechar{𝓚}{\ensuremath{\mathcal{K}}}
\newunicodechar{ᴹ}{\ensuremath{^M}}
\newunicodechar{ᵗ}{\ensuremath{^t}}
\newunicodechar{ξ}{\ensuremath{\xi}}
\newunicodechar{≐}{\ensuremath{\doteq}}
\newunicodechar{⸴}{\ensuremath{,}}
\newunicodechar{ᴱ}{\ensuremath{^E}}
\newunicodechar{⨟}{\ensuremath{\fatsemi}}
\newunicodechar{𝒜}{\ensuremath{\mathcal{A}}}
\newunicodechar{𝒮}{\ensuremath{\mathcal{S}}}
\newunicodechar{∪}{\ensuremath{\cup}}
\newunicodechar{∨}{\ensuremath{\lor}}
\input{agdamacros}

\definecolor{agdablue}{HTML}{0000CD}
\DisableLigatures[-]{encoding=T1}
\newcommand{\bsym}[1]{\textcolor{agdablue}{#1}}
\newcommand{\tdot}{\mkern2mu \textcolor{agdablue}{\cdotp} \mkern-2mu \textcolor{agdablue}{\cdotp} \mkern-2mu \textcolor{agdablue}{\cdotp} \mkern1mu}

\bibliographystyle{plainurl}
\title{Strictifying the Substitution Calculus for Scoped Syntax in Agda}

\author{Marius Weidner}{University of Freiburg, Germany}{weidner@cs.uni-freiburg.de}{0009-0008-1152-165X}{}
\authorrunning{Marius Weidner}
\Copyright{Marius Weidner}

\keywords{Explicit Substitutions, Rewrite Systems, Agda}

\ccsdesc[300]{Theory of computation~Equational logic and rewriting}

\begin{document}
\AgdaNoSpaceAroundCode{}

\input{systemf}
\input{examples}

\maketitle

\begin{abstract}
      Mechanizing the metatheory of programming languages in proof assistants is
      difficult because of recurring challenges with variable binding. Every
      formalization requires proving and correctly applying standard lemmas about
      substitutions such as those describing composition and interaction with binders.

      Historically researchers treated substitution as a meta level operation but
      explicit substitution calculi bring these operations into the syntax. The
      $σ_{SP}$-calculus provides reduction rules that ensure confluence and strong
      normalization so that users can compare normal forms instead of applying lemmas
      manually.

      \textsc{Autosubst 2} provide this procedure via tactics to produce
      \emph{propositional} equalities in the Rocq proof assistant. We instead show how
      to register the $σ_{SP}$-calculus rules as computational assumptions in Agda using
      \AgdaPragma{REWRITE} rules to produce \emph{definitional} equalities.
      We found this to be challenging because it requires precise control over
      reduction behavior, care to maintain the logical consistency of Agda,
      and we must adhere to the restriction of rewrite rules.
\end{abstract}

\section{Introduction}\label{sec:introduction}

% MAKE THIS PART HALF THE LENGTH 
Mechanizing the metatheory of programming languages in proof assistants is
difficult because of recurring challenges with variable binding. Every
formalization requires proving and correctly applying standard lemmas about
substitutions such as those describing composition and interaction with
binders.

Historically researchers treated substitution as a meta level operation but
explicit substitution calculi, first introduced bring these operations into the
syntax. The $σ_{SP}$-calculus provides reduction rules that ensure confluence
and strong normalization so that users can compare normal forms instead of
applying lemmas manually.

% - sigma calculus solves have confluence and termination (though no completion) (cite schäfer)
% -> but we need native renmaings ings (both useful and to obtain structural recrusion), autpsubst also considered this

% we base our work on their extension (they believed confluent and we prove it using agda confluence checker)

% go deeper into features.. (vector substitutions etc)

% we try to implement as much of autosubsts behaviour as possible 

% using is actually perfect for embeding via rewrite rules (.. some background) (in copnstrast to stark autosubst),
% -> agdasubst
% not only want we to have the rules postulated, but also an underlying model 

\subsection*{Structure}
The paper is organized into several sections. \Cref{sec:pre} introduces the
required background in Agda and explicit substitution calculi. \Cref{sec:mis}
presents an embedding of the $\sigma$-calculus with first class renamings into
Agda for multi-sorted and scoped syntax exemplified by System F. \Cref{sec:ags}
introduces the \textsc{Agdasubst} library which is a code generation tool to
automate producing the code from the previous section for custom syntax.
Finally we outline directions for future work in \cref{sec:wrk}, discuss
related work in \cref{sec:rwk} and conclude the paper in \cref{sec:con}.

\subsection*{Contributions}
The main contributions of this work are as follows.
\begin{itemize}
      \item A \emph{safe} embedding of the $\sigma$-calculus with first class renamings
            into Agda using \AgdaKeyword{REWRITE} rules in a scoped and multi-sorted
            setting.
      \item The presented version of the $\sigma$-calculus with first class renamings which
            constitutes the first \emph{proven} confluent version known to us.
      \item The introduction of the first version of the \textsc{Agdasubst} library which
            is a code generation tool that produces the presented code for arbitrary custom
            scoped syntaxes from a higher-order abstract syntax (HOAS) specification.
\end{itemize}

\noindent\textbf{The supplement contains the complete Agda code underlying this paper.}

\section{Preliminaries}\label{sec:pre}
We begin with a review of background material from the literature. We summarize
\begin{itemize}
      \item Agda and language features relevant to our work: rewrite rules and opaque
            blocks;
      \item Multi-sorted and scoped syntaxes in Agda.
      \item The $σ_{SP}$-calculus;
\end{itemize}

\subsection{Agda and Relevant Language Features}\label{sec:pre-agd}
Agda is a dependently typed functional language and proof assistant based on
Martin-Löf type theory~\cite{MARTINLOF197573}, where proofs are verified via
type checking. Proofs are significantly simplified when terms are
\textit{definitionally equal} (i.e.\ reduce to the same term). When terms are
not definitionally equal, users must resort to manual equational reasoning
using propositional equality, which is a cumbersome process as Agda lacks
automated proof-discharge tactics. Maximizing definitional equalities thus
directly reduces manual effort and allows for more complex intrinsic
(Agda-typical) constructions that avoid `transfer hell'.

Beyond standard features like indexed inductive types and dependent pattern
matching~\cite{10.1145/3236770}, this work is based on two more Agda-specific
language features.

\subsection*{Rewrite Rules}
User-defined rewrite rules~\cite{10.1145/3434341, cockx:LIPIcs.TYPES.2019.2}
extend Agda’s computation by treating propositional equalities as reduction
rules. A function $\AgdaFunction{eq} : (x_1 : A_1) \ldots \to
      \AgdaFunction{f}\, p_1 \ldots p_n \, \AgdaFunction{≡} \, v$ can be registered
as a rewrite rule if:
\begin{enumerate}
      \item The head \AgdaFunction{f} is a postulate, defined function, or constructor with
            distinct variable type parameters.
      \item Every variable $x_i$ appears in a pattern position among $p_1, \ldots, p_n$.
            The following pattern positions are relevent to us\footnote{The Agda
                  documentation includes a complete definition for pattern positions:
                  \url{https://agda.readthedocs.io/en/latest/language/rewriting.html}}:
            \begin{enumerate}
                  \item $x \, y_1 \ldots y_n$, where $x$ is a pattern variable and $y_1, \ldots, y_n$ are distinct variables that are bound locally in the pattern.
                  \item $\AgdaFunction{f} \, p_1 \ldots p_n$, where $f$ is a postulate, a defined function, a constructor, or a data/record type, and $p_1, \ldots, p_n$ are again patterns.
                  \item $\lambda \, x \to p$, where $p$ is again a pattern.
                  \item $y \, p_1 \ldots p_n$, where $y$ is a variable bound locally  and $p_1, \ldots, p_n$ are again patterns.
            \end{enumerate}
      \item The left side is neutral (it does not further reduce).
\end{enumerate}

\noindent For example, the right identity of addition satisfies these criteria: \ERewrite{} Once registered via \ERewriteIt{}, Agda treats \AgdaBound{n} \AgdaFunction{+} \AgdaNumber{0} as definitionally equal to \AgdaBound{n}: \ERewriteEx{}

Users must ensure rules are terminating and confluent to retain logical
consistency. Agda provides confluence checks via \texttt{--confluence-check}
and \texttt{--local-confluence-check}, termination is not verified, but also
not a critical property~\cite{10.1145/3434341}. When the rewrite system is
terminating, local confluence preserves consistency, otherwise full confluence
is required~\cite{10.1145/3434341}.

\subsection*{Opaque Blocks}
Opaque definitions~\cite{gratzer2022controlling} treat code as non-unfolding
symbols. Definitions within an \AgdaKeyword{opaque} block are visible but
treated abstractly during type checking: \EOpaque{} Outside this block,
\AgdaFunction{forty-two} does not reduce to \AgdaNumber{42}: \EOpaqueExO{}
However, the \AgdaKeyword{unfolding} keyword can be used to expose the
computational behavior of the underlying definition when needed: \EOpaqueExT{}

\subsection{Multi-Sorted and Scoped Syntaxes in Agda}\label{sec:pre-syn}
\begin{figure}[t]
      \footnotesize
      \setlength{\mathindent}{0pt}
      \setlength{\AgdaEmptySkip}{\smallskipamount}
      \begin{minipage}[t]{0.43\linewidth}
            \raggedright{}
            \EScoped{}
      \end{minipage}
      \hfill
      \begin{minipage}[t]{0.55\linewidth}
            \raggedright{}
            \EMultiSorted{}
      \end{minipage}
      \caption{Classically Scoped Syntax vs. Multi-Sorted and Scoped Syntax}\label{fig:pre-svm}
\end{figure}

Many languages feature multiple syntactic categories.
System~F~\cite{girard1972, 10.1007/3-540-06859-7_148}, for instance, has
expressions, types, and kinds. To prevent scoping errors, we use
\textit{intrinsically scoped syntax}, where inductive syntax definitions are
indexed by free variables to ensure terms are correctly scoped by construction.
\Cref{fig:pre-svm} (left) shows a standard implementation: types are indexed by
type-variable counts, and expressions by both type and expression variable
counts. Variables use de Bruijn indices of type \AgdaDatatype{Fin}
\AgdaBound{n}, that is, natural numbers strictly less than the number of free
variables. This representation ensures that only well-scoped terms can be
constructed, but also requires defining many substitution permutations (e.g.,
type-in-type, type-in-expr, expr-in-expr).

To avoid this combinatorial explosion later, we unify the syntax in two
orthogonal dimensions, presented also in \cref{fig:pre-svm} (right). Variables
and terms form a single datatype \AgdaBound{S} \AgdaDatatype{⊢[} \AgdaBound{m}
\AgdaDatatype{]} \AgdaBound{s} indexed by a \AgdaDatatype{Mode} \AgdaBound{m},
a target \AgdaDatatype{Sort} \AgdaBound{s} and a \AgdaDatatype{Scope}
\AgdaBound{S}. Modes distinguish constructors for variables and terms.
Variables are represented across multiple sorts simultaneously by natural
numbers of type \AgdaBound{S} \AgdaDatatype{∋} \AgdaBound{s} that act as de
Bruijn indices. The position of \AgdaBound{s} in \AgdaBound{S} directly
corresponds to the debruijn index of the variable. Using the unified syntax, we
can later define scoped and multi-sorted renamings and substitutions that
operate uniformly across all sorts, thereby avoiding vector substitions.

\subsection{The $\sigma_{SP}$-Caclulus}\label{sec:pre-sig}
\begin{figure}[t]
      \centering
      \footnotesize
      \begin{minipage}[t]{0.44\textwidth}
            \raggedright
            \textbf{Traversal Rules}
            \begin{alignat*}{2}
                  \mathsf{lam} \quad & (\textcolor{agdablue}{λ.}\, t)                                         & \tdot σ
                                     & = \textcolor{agdablue}{λ.} (t \tdot (\textcolor{agdablue}{\uparrow}σ))           \\
                  \mathsf{app} \quad & (t_1 \ t_2)                                                            & \tdot σ
                                     & = (t_1 \tdot σ) \ (t_2 \tdot σ)
            \end{alignat*}
            \textbf{Monad Rules}
            \begin{alignat*}{2}
                  \mathsf{right\text{-}id} \quad  & t \tdot \mathsf{id}     &  & = t                                         \\
                  \mathsf{compositionality} \quad & (t \tdot σ_1) \tdot σ_2 &  & = t \tdot (σ_1 \textcolor{agdablue}{;} σ_2)
            \end{alignat*}
      \end{minipage}
      \hfill
      \begin{minipage}[t]{0.48\textwidth}
            \raggedright{}
            \textbf{Interaction Rules}
            \begin{alignat*}{2}
                  \mathsf{id}_\mathsf{L} \quad   & \textcolor{agdablue}{\mathsf{id}} \ \textcolor{agdablue}{;} \ σ                                                                             &  & = σ                                                                               \\
                  \mathsf{id}_\mathsf{R} \quad   & σ \ \textcolor{agdablue}{;} \ \textcolor{agdablue}{\mathsf{id}}                                                                             &  & = σ                                                                               \\
                  \mathsf{associativity} \quad   & (σ_1 \ \textcolor{agdablue}{;} \ σ_2) \ \textcolor{agdablue}{;} \ σ_3                                                                       &  & = σ_1 \ \textcolor{agdablue}{;} \ (σ_2 \ \textcolor{agdablue}{;} \ σ_3)           \\
                  \mathsf{distributivity} \quad  & (t \textcolor{agdablue}{\cdot} σ_1) \ \textcolor{agdablue}{;} \ σ_2                                                                         &  & = (t \tdot σ_2) \textcolor{agdablue}{\cdot} (σ_1 \ \textcolor{agdablue}{;} \ σ_2) \\
                  \mathsf{interaction} \quad     & \textcolor{agdablue}{\mathsf{wk}} \ \textcolor{agdablue}{;} \ (t \textcolor{agdablue}{\cdot} σ)                                             &  & = σ                                                                               \\
                  \eta\mathsf{\text{-}id} \quad  & (\textcolor{agdablue}{\mathsf{zero}}) \textcolor{agdablue}{\cdot} \textcolor{agdablue}{\mathsf{wk}}                                         &  & = \textcolor{agdablue}{\mathsf{id}}                                               \\
                  \eta\mathsf{\text{-}law} \quad & (\textcolor{agdablue}{\mathsf{zero}} \tdot σ) \textcolor{agdablue}{\cdot} (\textcolor{agdablue}{\mathsf{wk}} \ \textcolor{agdablue}{;} \ σ) &  & = σ
            \end{alignat*}
      \end{minipage}
      \caption{The three-sorted $σ_{SP}$-calculus rules}\label{fig:pre-sig}
\end{figure}

The $\sigma$-calculus~\cite{10.1145/96709.96712} formalizes substitutions via
explicit reduction rules. The
$\sigma_{SP}$-calculus~\cite{10.1145/2676724.2693163} succeeds the original and
provides a terminating and confluent and complete~\cite{10.1145/226643.226675,
      10.1145/2676724.2693163} foundation for our purposes. We refer to the PhD
thesis of Stark~\cite{Stark:2020:Mechanising} for a complete account of the
history of explicit substitution calculi.

We review the variant of the $\sigma_{SP}$-calculus which also appears in the
first version of \textsc{Autosubst}~\cite{schafer2015autosubst}. Variables $x$
represent natural numbers and terms $t$ include the standard lambda calculus
constructors as an example for any single sorted syntax with binding. A
substitution $\sigma$ maps variables to terms. We view a substitution as a
stream of terms $(t_0, t_1, \dots)$ that supports a shift operation for
binders. This perspective motivates the definition of the extension operation
$t \textcolor{agdablue}{\cdot} \sigma$ and the identity
$\textcolor{agdablue}{\mathsf{id}}$ and the weakening substitution
$\textcolor{agdablue}{\mathsf{wk}}$
\begin{align*}
      (t \textcolor{agdablue}{\cdot} \sigma)(0)   & := t         & \textcolor{agdablue}{\mathsf{id}}(x) & := x   \\
      (t \textcolor{agdablue}{\cdot} \sigma)(x+1) & := \sigma(x) & \textcolor{agdablue}{\mathsf{wk}}(x) & := x+1
\end{align*}

We introduce the application of a substitution $\tdot$ to a term and refer to
this operation as instantiation. We define composition as $(\sigma_1 \
      \textcolor{agdablue}{⨟} \ \sigma_2)(x) := \sigma_1(x) \tdot \sigma_2$. This
definition allows us to define the \bsym{\textsf{lift}} operator
$\textcolor{agdablue}{\uparrow}\sigma := \textcolor{agdablue}{\mathsf{zero}}
      \textcolor{agdablue}{\cdot} (\sigma \ \textcolor{agdablue}{⨟} \
      \textcolor{agdablue}{\mathsf{wk}})$ which pushes substitutions under binders.
Lifting preserves the bound variable while it weakens the terms of the
substitution to maintain correct scoping.

\Cref{fig:pre-sig} presents the accompanying rewrite system.
We follow Stark~\cite{Stark:2020:Mechanising} and categorize the reduction rules into three groups.
\begin{itemize}
      \item \textbf{Traversal} rules govern propagation through terms and under binders.
      \item \textbf{Monad rules} define the behavior of substitutions applied to terms.
      \item \textbf{Interaction rules} describe equalities between substitution expressions.
\end{itemize}
We refer the reader to the original $\sigma_{SP}$-calculus paper~\cite{10.1145/2676724.2693163} for a detailed description of the rule set selection.

The definition of instantiation via mutual recursion with composition
complicates termination. Agda specifically cannot determine that this
definition is structurally recursive. Because of that we follow
Adams~\cite{10.1007/11617990_1} to resolve this issue and first define
instantiation for renamings which are substitutions that map variables to
variables. We then extend the above categories to handle first class renamings
and to connect substitutions with renamings in the spirit of Autosubst
2~\cite{10.1145/3293880.3294101}. The addition of first class renamings to the
theory breaks completeness so we do not consider this property important for
going forward. We prioritize a large space of solvable equations.

\section{Stricification of the Sigma Calculus Rules}\label{sec:mis}
We now proceed to examine the direct embedding of the $σ_{SP}$-calculus rules
with first class renamings. Even if instantiation could be realized immediately
we require renamings to state some laws because the view of renamings as second
class constructs obscures the proofs. For instance substitution properties
might require proving the corresponding instance for renamings. Furthermore
certain statements only hold for renamings~\cite{goguen1997candidates}. We use
the multi-sorted and scoped version of System F from \cref{fig:pre-svm} as an
example for any custom scoped syntax. The approach described in this section
translates directly to any definition of multi-sorted and scoped syntax as well
as simple intrinsically typed syntax where the type index of the syntax has
type \AgdaDatatype{Set}.

\subsection{Scoped and Multi-Sorted Renaming and Substitution}\label{sec:mis-sub}
\begin{figure}[t]
      \setlength{\mathindent}{0pt}
      \setlength{\AgdaEmptySkip}{\smallskipamount}
      \begin{minipage}[t]{0.33\linewidth}
            \raggedright{}
            \ERen{}
      \end{minipage}
      \hfill
      \begin{minipage}[t]{0.55\linewidth}
            \raggedright{}
            \ESub{}
      \end{minipage}
      \caption{Scoped and Multi-Sorted Renaming and Substitution (excerpt)}\label{fig:mis-ras}
\end{figure}
The key to the rewriting of the $σ_{SP}$-calculus rules already lies in the definitions of renamings and substitutions, presented in \Cref{fig:mis-ras}.

We begin with the definition of renamings \AgdaFunction{\_→ᴿ\_} which map
variables of sort \AgdaBound{s} from scope \AgdaBound{S₁} to variables of the
same sort in scope \AgdaBound{S₂}. Renamings constitute a special case of
substitutions. We proceed to define the renaming primitives identity
\AgdaFunction{id} and weakening \AgdaFunction{wk} and forward composition
\AgdaFunction{\_∘\_}.

We define lifting \AgdaFunction{\_↑ᴿ\_} directly to avoid a definition for the
extension of renamings because renamings merely shift variables. The traversal
operation \AgdaFunction{\_⋯ᴿ\_} propagates the renaming through the syntax to
the variables and applies lifting when it passes under binders. The
computational behavior of renamings follows the behavior of functions almost
entirely. We require only a few additional \AgdaPragma{REWRITE} rules later to
extend the definitional equalities of renamings.

Substitutions map variables to terms rather than variables. The operator
\AgdaFunction{⟨\_⟩} embeds any renaming into a substitution by wrapping the
resulting variable of the renaming in the variable constructor. We define the
identity substitution \AgdaFunction{idˢ} and weakening substitution
\AgdaFunction{wkˢ} directly from renamings. We \AgdaKeyword{INLINE} these
symbols to prevent Agda from rejecting rewrite rules due to reductions on the
left hand side. We define the extension of substitutions \AgdaFunction{\_∙\_}
by case analysis on the incoming variable.

We declare extension and all other substitution primitives as
\AgdaKeyword{opaque} to block their reduction. This abstraction hides the fact
that substitutions are functions. We will later use the instantiation of
variables with substitutions for the functionality of lookup in a substitution
instead of native function application. In other words, we use \AgdaBound{x}
\AgdaFunction{⋯ˢ} \AgdaBound{σ} instead of normal function application
\AgdaBound{σ} \AgdaBound{\_} \AgdaBound{x} to lookup a variable inside a
substitution. The important difference is that we can rewrite all rules on the
blocking symbol \AgdaFunction{\_⋯ˢ\_} but not on function application.
\Cref{sec:mis-ex} details this discrepancy.

We use renaming with a weakening to define the lifting of substitutions
\AgdaFunction{\_↑ˢ\_} to satisfy the requirements of structural recursion.
Instantiation with a substitution \AgdaFunction{\_⋯ˢ\_} is analogous to the
case of renamings.

We design \AgdaFunction{\_⋯ˢ\_} so it works for variables and terms
simultaneously as discussed\footnote{We could have introduced a separate symbol
      for variable lookup in a substitution but the unified approach allows us to
      state some rules across both modes.}. Finally we define the forward composition
of substitutions \AgdaFunction{\_⨟\_} via instantiation with the weakening
renaming.

\subsection{The Sigma Calculus with First Class Renamings}\label{sec:mis-sig}
\begin{figure}[t]
      \setlength{\mathindent}{0pt}
      \begin{minipage}[t]{0.48\linewidth}
            \raggedright{}
            \EDefLaws{}
            \EInteractLaws{}
            \ERewriteSys{}
      \end{minipage}
      \hfill
      \begin{minipage}[t]{0.48\linewidth}
            \raggedright{}
            \ECoincidenceLaws{}
            \ETraversalLaws{}
            \EMonadLaws{}
      \end{minipage}
      \caption{The $σ_{SP}$-Calculus Rules with First Class Renamings for Scoped Syntax}\label{fig:mis-sig}
\end{figure}

\Cref{fig:mis-sig} presents the $\sigma_{SP}$-calculus rules with first class renamings
as Agda equalities that we add as rewrite rules.
Agda accepts the rule set as confluent which is important for safety.
Before rewriting we \emph{prove} all the rules to guarantee logical consistency~\cite{10.1145/3434341}.
We now discuss the changes to the standard $\sigma_{SP}$-calculus variant from \Cref{sec:pre-sig}.

First class renamings introduce three new categories specifically First Class
Renaming rules and Definitional rules and Coincidence rules.

The \textbf{first class Renamings rules} include a single rule which states
that the lifting of the identity renaming is the identity. This property does
not hold definitionally because the lifting of renamings is defined via a case
split on the incoming variable but it is strictly required for a confluent
account. All other properties we expect of renamings such as associativity hold
definitionally.

The \textbf{Definitional rules} include definitions we blocked earlier in a
slightly modified form. We include the clauses from the definition of extension
but the instantiation operator handles the lookup of variables in a
substitution instead of function application. Lifiting of substitutions is
defined in terms of substitution composition. We could not do this earlier due
to the structural dependence of substitutions on renamings but the equality is
provable in the underlying model. We also define lookup in composed
substitutions in terms of the instantiation operator for variable lookup in
substitutions.

The \textbf{Interaction rules} include a minor change. Since instantiation with
renamings is available as a first class construct we must add a version of the
$\eta$-law for renamings where instantiation appears as a blocking symbol.

The \textbf{Coincidence rules} emerge from the interaction between renamings
and substitutions~\cite{10.1145/3293880.3294101} and allow us to recover
renamings from their embedding as substitutions. We state them uniformly for
variables and terms \AgdaBound{x/t} traversed by a substitution because we also
want to coerce variable lookup in substitutions to variable lookup in
renamings. The extraction of arbitrary renamings from deeply nested
substitutions requires a dedicated solving strategy which we leave for future
work. However we require a specific coincidence variant called
\AgdaFunction{coincidence-fold} for the proof of subject reduction of System F.
Every proof might require more coincidence lemmas and we are currently trying
to figure out the patterns. For instance subject reduction might require the
inclusion of lemmas in the style of \AgdaFunction{coincidence-fold} with $n$
extensions where $n$ is the maximum arity of binders in the object language.

\textbf{Traversal rules} mirror the clauses from the instantiation
definitions in the previous section directly.
We include all clauses except the one for variables.
This exclusion is important because we wanted the term \AgdaBound{x} \AgdaFunction{⋯ˢ} \AgdaBound{σ} to remain stuck and not reduce to function application.

We extend the \textbf{Monad rules} to handle all four combinations of
instantiation with renamings and substitutions. We state the right identity law
only for renamings because the instantiation with the identity substitution
reduces to instantiation with the identity renaming via
\AgdaFunction{coincidence}.

The proofs of all rules are standard and we refer the interested reader to the
supplement. We prove First Class Renaming, Definitional and Interaction laws
syntax independently while Traversal, Monad and Coincidence rules require
structural induction over the syntax. Within the proofs we unfold the opaque
definitions and make use of function extensionality which is conservative over
the theory underlying Agda~\cite{Hofmann1997}.

\subsection{An Examination of the Issue with Unfolding and Rewrite Rules}\label{sec:mis-ex}
We have seen that we try to hide the fact that substitutions are functions in
contrast to renamings and use the instantiation operator to apply a variable to
a substitution rather than function application We do this specifically because
of the definition for composition of two substitutions.

To examine the issue more closely, we look at the LHS of the
\AgdaFunction{η-lawˢ}: \EEtaLaw{}. If we let the definition of composition
reduce and inline its definition we would obtain the term \EEtaLawUnfolded{} on
the LHS. But this LHS is not a valid rewrite rule since $\sigma$ does not
appear in pattern position. The first appearance of $\sigma$ is not in pattern
position because it is applied to \AgdaInductiveConstructor{zero} which
violates the constraint of the restriction of rewrite rules from \textbf{2.a}
in \Cref{sec:pre-agd}. The second appearance suffers the same issue after we
apply condition \textbf{2.c}. We fix the issue by blocking the definition of
instantiation for variables and then using instantiation instead of function
application. This successfully fulfills condition \textbf{2.b} and makes
$\sigma$ appear as a pattern variable. This change to use \AgdaFunction{\_⋯ˢ\_}
instead of function application must propagate through all definitions so the
rules also apply when we want to solve equations that use the composition
operation.

We would still need to block the reduction of composition even if the issue
with the $\eta$-law were fixable. The \AgdaFunction{comp-idₗ} rule for instance
reduces to a lambda and would not constitute a valid rewrite rule\footnote{This
      issue alone does not justify the workaround using instantiation instead of
      function application.}

While we could still use function application for the actual definition of
lifting in the subsequent code, we then would obtain fewer definitional
equalities. As a consequence, we recommend using only instantiation for
variable lookup and only the already defined composition operators.

We also considered an alternative representation for substitutions based on a
vector structure. However the $\eta$ law is not provable in this setting
because then extension corresponds to the cons constructor of the vector. As a
consequence, it would prevent the instantiation of our rewrite system with
proofs to guarantee consistency.

\subsection{The Simga Calculus Rules in Action}
\begin{figure}[!t]
      \setlength{\mathindent}{0pt}
      \begin{minipage}[t]{0.48\linewidth}
            \raggedright{}
            \ESPT{}
      \end{minipage}
      \begin{minipage}[t]{0.48\linewidth}
            \raggedright{}
            \ESR{}
      \end{minipage}
      \caption{Subject Reduction for System F}\label{fig:ext-rst}
\end{figure}

The equations of the $\sigma_{SP}$ calculus apply automatically throughout the
subsequent code via rewrite rules. We demonstrate the proof of subject
reduction which proceeds without explicit management of substitutions. We
provide the proof for a standard algorithmic type system for System F
\AgdaDatatype{\_⊢\_:\_} and call-by-value semantics \AgdaDatatype{\_↪\_}. We
refer to the supplement for the full example and all definitions.

We establish subject reduction and first prove that renaming and substitution
preserve typing. The left side of \cref{fig:ext-rst} displays the proofs for
the preservation lemmas. We omit the proof that renaming preserves typing
\AgdaFunction{\_⊢⋯ᴿ\_} as it is analogous to substitution
\AgdaFunction{\_⊢⋯ˢ\_}. We use this result to prove that the lifting of a
substitution preserves typing \AgdaFunction{⊢↑ˢ} and subsequently show that
general substitution preserves typing \AgdaFunction{\_⊢⋯ˢ\_}. The proof relies
on the type for well typed substitutions \EWTS{} and we omit its definition.
The variable cases require a lookup of the well typed variable in the
corresponding renaming or substitution. We apply the preservation lemma for the
lifting of renamings and substitutions when the traversal passes under a
binder. Both proofs follow directly by induction and include the cases under
binders because the $\sigma$ calculus laws apply automatically. This approach
removes the need to manually move the substitution under a weakening for
binders or under a singleton substitution in the case of type application which
is required in standard proofs.

Subject reduction \AgdaFunction{sr} directly from the substitution preserves
typing lemma as shown on the right of \cref{fig:ext-rst}. The rewrite rules
automatically handle the otherwise required application of interaction rules in
the cases of $β$-reductions. Here \AgdaFunction{⊢[]} is the well typed single
substitution which can be derived directly from the substitution preserves
typing lemma.

This example demonstrates the applicability on a small scale. We also applied
our approach to the $π$-calculus in an unrelated project to build a complex
logical relation. In that setting we also avoided manual reasoning over
substitutions entirely, sparing us probably hundreds of lines of tedious manual
substition reasoning.

\subsection{Discussion}\label{sec:mis-dis}
The rules presented in \cref{sec:mis-sig} are an extension of the classic
$σ_{SP}$-calculus rules to a setting with renamings, closely following the work
done by the \textsc{Autosubst} team. After careful construction, marking the
relevant symbols opaque, they fulfill the requirements to be registered as
rewrite rules. Inside the proofs of the rules themselves, the definitions can
still unfold when needed, while externally they behave as uninterpreted symbols
and are reduced only by the $σ$-calculus rules. Separating renamings and
substitutions causes some duplication.

This construction reveals a broader concept that serves as a general recipe for
integrating algebraic laws over complex structures into proof assistants:
\begin{itemize}
      \item Define symbols for the structure.
      \item Establish a corresponding confluent rewrite system.
      \item Instantiate the structure with an underlying model to prove the laws of the
            rewrite system inside an \AgdaKeyword{opaque} block.
      \item With the reduction of the model hidden, \AgdaKeyword{REWRITE} the algebraic
            equations.
\end{itemize}

The structure is instantiated to maintain consistency but the underlying
reductions need to be hidden from the outside. This abstraction prevents the
system from introducing unwanted simplifications during type checking. In our
case we need a \emph{symbol} for variables instantiated to substiutions, for
which native function application is not suitable, due to the restrictions on
rewrite rules.

This concept of Dedukti~\cite{assaf2023deduktilogicalframeworkbased} style
constants and rewrite rules with hidden models could be applied to other
structures like natural numbers, lists or sets. Such an approach would provide
more definitional equalities for indexed types like vectors. To this end, Agda
would benefit greatly from an extension with rewriting modulo commutativity to
support more confluent rewrite systems for common algeraic structures.

\section{Agdasubst: Autosubst for Agda with Definitional Equalities}\label{sec:ags}

\section{Related Work}\label{sec:rwk}

There exist a variety of explicit substitution calculi starting with the
original $\sigma$-calculus~\cite{10.1145/96709.96712} which includes primitives
for identity and extension and shifting and composition. The
$\sigma_{SP}$-calculus~\cite{10.1145/2676724.2693163} adds further reductions
to ensure confluence. Only the $\sigma_{SP}$-calculus is both sound and
complete which enables the derivation of all equations between terms via
reduction. Mechanized proofs establish both completeness and confluence+
termination~\cite{10.1145/2676724.2693163, Stark:2020:Mechanising}. The
extension with first class renamings has not yet been formally verified to be
confluent but Stark~\cite{Stark:2020:Mechanising} conjectured that their
addition preserves confluence. Our implementation confirms this conjecture.

We discussed \textsc{Autosubst 2} and related
research~\cite{10.1145/3293880.3294101, schafer2015autosubst,
      Stark:2020:Mechanising} extensively throughout this work as it forms the direct
theoretical foundation of our contribution.

Recent work on local rewrite rules~\cite{10.1145/3776704} embeds the rules of
\textsc{Autosubst 2} with first class renamings into Rocq via rewrite rules
similar to our approach. They provide a model for renamings and substitutions
as functions as we do but they are unable to prove the laws given the
underlying model or lack a mechanism to do so. Their example also misses rules
of the $\sigma_{SP}$-calculus, required for a complete account.

Another line of research concerns generic programming
approaches~\cite{allais2021typescopesafeuniverse, 10.1145/3018610.3018613}
which allow the construction of syntax trees from general building blocks that
support sums and products and binder introduction. This approach also allows
for a generic structure that abstracts over renamings and substitutions. The
generic syntax approach adapts to a scoped and multi sorted
setting~\cite{saffrich:LIPIcs.ITP.2024.32}. Importantly generic programming and
the rewriting of $\sigma_{SP}$ rules are orthogonal and we strongly believe our
rewriting approach could be applied to a universe of syntaxes with binding
setting.

The \textsc{Abstract Binding Trees} library shares a similar goal with
\textsc{Agdasubst} and is used for instance in Siek's
work~\cite{siek2021parameterized}. It permits the construction of an object
language from a generic syntax similar to the universe of syntaxes with binding
approach~\cite{allais2021typescopesafeuniverse, 10.1145/3018610.3018613} but it
supports only extrinsic scoping and single sorted syntaxes. The library mainly
focuses on abstractions over arbitrary predicates about the object language.
The library also offers incomplete experimental support for rewriting using
$\sigma$-calculus laws but misses some rules of the $\sigma_{SP}$-calculus and
does not pass the confluence check, therby lacking safety guarantees.

Wadler experiments with explicit substitution directly in his contribution to
the Festschrift in Honor of Peter Thiemann~\cite{Wadler_2024} and implements
weakening as a constructor in the syntax but his work has two shortcomings. He
rewrites only some of the sigma calculus laws but cannot pass the confluence
checker. A more severe issue is that variables in his representation can be
represented by multiple terms that are not definitionally equal which results
in the loss of important definitional equalities.

Wadler and colleagues also investigate the uniform treatment of renamings and
substitutions in a typed setting~\cite{altenkirch2025substitution} which
originates from work by McBride~\cite{ren-sub}. This approach reduces repeated
similar lemmas and allows the unification of properties such as the
preservation of typing for renamings and substitutions. They also unify
variables and terms and we drew inspiration from this concept. They hide the
structural dependence of renamings and substitutions in the Mode structure
itself and use termination checking directly to make Agda see this dependence.
Saffrich~\cite{saffrich:LIPIcs.ITP.2024.32} adopts a different approach and
hides it in instance resolution. Wood~\cite{Wood_2021} provides a further
application of the uniform treatment approach and uses it to reason about
metatheory in the context of linear languages. Future work must determine if
the $\sigma_{SP}$-calculus rules can be adapted to a setting with a uniform
treatment of renamings and substitutions.

The \textsc{Nominal 2} Isabelle package~\cite{Nominal2-AFP} is more distantly
related. It addresses the $\alpha$-equivalence problem using nominal datatypes
and freshness conditions rather than de Bruijn indices. Their approach is
closer to the intuitive notion of variables but the tool does not provide any
strategy for solving equations involving terms and substitutions.

\section{Further Work}\label{sec:wrk}
\subsection{Substitutions for Intriniscally Typed Syntax}
Rewriting $σ$-calculus laws improves intrinsically typed syntax in addition to
scoped syntax. We demonstrate this by reformulating System F as an example for
any complex syntax, where substitution appears on indices, and prove the monad
laws for expression substitution. These proofs are simplified significantly if
the $σ$-calculus laws for types are rewritten first. This development
represents a significant step toward using intrinsically typed syntax over
complex languages for proofs in Agda. Previous works like
\cite{10.1145/3678000.3678201} required thousands of lines of code to prove
substitution laws. The length of these proofs stems from the transfer lemma
\AgdaFunction{subst} appearing inside the \emph{signature} of expression
substitution lemmas. This leads to the infamous transfer hell where users must
manually manage shifting transfers. Our rewriting approach mitigates this
issue.

% PROBLEM WITH NON STRICT SUBST CALC.. e.g. tyde paper and strictifcation of CwF?
\section{Conclusion}\label{sec:con}

\bibliography{references}

\end{document}
