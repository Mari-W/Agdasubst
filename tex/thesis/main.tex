\documentclass[screen,nonacm]{acmart}
\usepackage[capitalize,noabbrev]{cleveref}
\usepackage{microtype}
\usepackage{agda}

\usepackage{dsfont}
\usepackage{newunicodechar}
\newunicodechar{λ}{\ensuremath{\mathnormal\lambda}}
\newunicodechar{σ}{\ensuremath{\mathnormal\sigma}}
\newunicodechar{τ}{\ensuremath{\mathnormal\tau}}
\newunicodechar{π}{\ensuremath{\mathnormal\pi}}
\newunicodechar{ℕ}{\ensuremath{\mathbb{N}}}
\newunicodechar{∷}{\ensuremath{::}}
\newunicodechar{≡}{\ensuremath{\equiv}}
\newunicodechar{≅}{\ensuremath{\cong}}
\newunicodechar{∀}{\ensuremath{\forall}}
\newunicodechar{ᴸ}{\ensuremath{^L}}
\newunicodechar{ᴿ}{\ensuremath{^R}}
\newunicodechar{ʳ}{\ensuremath{^r}}
\newunicodechar{ᶻ}{\ensuremath{^Z}}
\newunicodechar{ₗ}{\ensuremath{_l}}
\newunicodechar{ⱽ}{\ensuremath{^V}}
\newunicodechar{⟧}{\ensuremath{\rrbracket}}
\newunicodechar{⟦}{\ensuremath{\llbracket}}
\newunicodechar{⊤}{\ensuremath{\top}}
\newunicodechar{⊥}{\ensuremath{\bot}}
\newunicodechar{₁}{\ensuremath{_1}}
\newunicodechar{₂}{\ensuremath{_2}}
\newunicodechar{₃}{\ensuremath{_3}}
\newunicodechar{₄}{\ensuremath{_4}}
\newunicodechar{₅}{\ensuremath{_5}}
\newunicodechar{₆}{\ensuremath{_6}}
\newunicodechar{₇}{\ensuremath{_7}}
\newunicodechar{₈}{\ensuremath{_8}}
\newunicodechar{₉}{\ensuremath{_9}}
\newunicodechar{∈}{\ensuremath{\in}}
\newunicodechar{₀}{\ensuremath{_0}}
\newunicodechar{′}{\ensuremath{'}}
\newunicodechar{ˢ}{\ensuremath{^S}}
\newunicodechar{ᴬ}{\ensuremath{^A}}
\newunicodechar{∘}{\ensuremath{\circ}}
\newunicodechar{𝟙}{\ensuremath{\mathds{1}}}  
\newunicodechar{𝟘}{\ensuremath{\mathds{O}}}
% \newunicodechar{𝟙}{\ensuremath{\mathbb{I}}}  
% \newunicodechar{𝟘}{\ensuremath{\mathbb{O}}}
\newunicodechar{ᴾ}{\ensuremath{^P}}
\newunicodechar{ᵀ}{\ensuremath{^T}}
\newunicodechar{⊎}{\ensuremath{\uplus}}
\newunicodechar{ι}{\ensuremath{\iota}}
\newunicodechar{⇐}{\ensuremath{\Leftarrow}}
\newunicodechar{⇒}{\ensuremath{\Rightarrow}}
\newunicodechar{➙}{\ensuremath{\to^P}}
\newunicodechar{Δ}{\ensuremath{\Delta}}
\newunicodechar{∅}{\ensuremath{\emptyset}}
\newunicodechar{⁺}{\ensuremath{^+}}
\newunicodechar{𝕏}{\ensuremath{\mathbb{X}}}
%\newunicodechar{·}{\ensuremath{\cdot}} %seems to be defined already!
\newunicodechar{∙}{\ensuremath{\cdot}}
\newunicodechar{⁇}{\ensuremath{?}}
\newunicodechar{‼}{\ensuremath{!}}
\newunicodechar{⊕}{\ensuremath{\oplus}}
\newunicodechar{ℤ}{\ensuremath{\mathbb{Z}}}
\newunicodechar{μ}{\ensuremath{\mu}}
\newunicodechar{∃}{\ensuremath{\exists}}
\newunicodechar{;}{\ensuremath{\fatsemi}}
\newunicodechar{Σ}{\ensuremath{\Sigma}}
\newunicodechar{ᴿ}{\ensuremath{^R}}
\newunicodechar{ᵢ}{\ensuremath{_i}}
\newunicodechar{≢}{\ensuremath{\nequiv}}
\newunicodechar{≟}{\ensuremath{\stackrel{{\tiny?}}{=}}}
\newunicodechar{≤}{\ensuremath{\le}}
\newunicodechar{ᵇ}{\ensuremath{^b}}
\newunicodechar{𝓣}{\ensuremath{\mathcal{T}}}
\newunicodechar{𝓔}{\ensuremath{\mathcal{E}}}
\newunicodechar{Γ}{\ensuremath{\Gamma}}
\newunicodechar{γ}{\ensuremath{\gamma}}
\newunicodechar{⊔}{\ensuremath{\sqcup}}
\newunicodechar{α}{\ensuremath{\alpha}}
\newunicodechar{η}{\ensuremath{\eta}}
\newunicodechar{ω}{\ensuremath{\omega}}
\newunicodechar{◁}{\ensuremath{\lhd}}
%PT: seems to be already defined
%\newcommand{\lambdabar}{{\mkern0.75mu\mathchar '26\mkern -9.75mu\lambda}}
\newunicodechar{ƛ}{\ensuremath{\lambdabar}}
\newunicodechar{Λ}{\ensuremath{\mathnormal\Lambda}}
\newunicodechar{ρ}{\ensuremath{\rho}}
\newunicodechar{𝓖}{\ensuremath{\mathcal{G}}}
\newunicodechar{ℓ}{\ensuremath{\ell}}
\newunicodechar{♯}{\ensuremath{\sharp}}
\newunicodechar{⇓}{\ensuremath{\Downarrow}}
\newunicodechar{𝓥}{\ensuremath{\mathcal{V}}}
\newunicodechar{∧}{\ensuremath{\wedge}}
\newunicodechar{ₛ}{\ensuremath{_s}}
\newunicodechar{χ}{\ensuremath{\chi}}
\newunicodechar{⊨}{\ensuremath{\models}}
\newunicodechar{⦂}{\ensuremath{\mathbf{:}}}
\newunicodechar{ς}{\ensuremath{\varsigma}}
\newunicodechar{𝓓}{\ensuremath{\mathcal{D}}}
\newunicodechar{∎}{\ensuremath{\square}}
\newunicodechar{■}{\ensuremath{\blacksquare}}
\newunicodechar{↠}{\ensuremath{\twoheadrightarrow}}
\newunicodechar{↪}{\ensuremath{\hookrightarrow}}
\newunicodechar{ⁱ}{\ensuremath{^i}}
\newunicodechar{Π}{\ensuremath{\Pi}}
\newunicodechar{∋}{\ensuremath{\ni}}
\newunicodechar{∍}{\ensuremath{\ni}}
\newunicodechar{‵}{\ensuremath{`}}
\newunicodechar{⨆}{\ensuremath{\bigsqcup}}
\newunicodechar{δ}{\ensuremath{\delta}}
\newunicodechar{κ}{\ensuremath{\kappa}}
\newunicodechar{⦃}{\ensuremath{\{\!|}}  % \lBrace from stix
\newunicodechar{⦄}{\ensuremath{|\!\}}}  % \rBrace from stix
\newunicodechar{⌊}{\ensuremath{\lfloor}}
\newunicodechar{⌋}{\ensuremath{\rfloor}}
\newunicodechar{𝟎}{\ensuremath{\mathbf{0}}}
\newunicodechar{𝟏}{\ensuremath{\mathbf{1}}}
\newunicodechar{𝟐}{\ensuremath{\mathbf{2}}}
\newunicodechar{β}{\ensuremath{\beta}}
\newunicodechar{≥}{\ensuremath{\geq}}
\newunicodechar{ₒ}{\ensuremath{_o}}
\newunicodechar{∊}{\ensuremath{\in}}
\newunicodechar{;}{\ensuremath{;}}
\newunicodechar{⋯}{\ensuremath{\mkern2mu \cdotp \mkern-2mu \cdotp \mkern-2mu \cdotp \mkern1mu }}
\newunicodechar{⊢}{\ensuremath{\vdash}}
\newunicodechar{∶}{\ensuremath{:}}
\newunicodechar{★}{\ensuremath{\star}}
\newunicodechar{ᴷ}{\ensuremath{^K}}
\newunicodechar{ϕ}{\ensuremath{\varphi}}
\newunicodechar{ᴮ}{\ensuremath{^B}}
\newunicodechar{+}{}
\newunicodechar{ᵣ}{\ensuremath{_r}}
\newunicodechar{ₓ}{\ensuremath{_x}}
\newunicodechar{ₜ}{\ensuremath{_t}}
\newunicodechar{𝓡}{\ensuremath{\mathcal{R}}}
\newunicodechar{𝓢}{\ensuremath{\mathcal{S}}}
\newunicodechar{𝓚}{\ensuremath{\mathcal{K}}}
\newunicodechar{ᴹ}{\ensuremath{^M}}
\newunicodechar{ᵗ}{\ensuremath{^t}}
\newunicodechar{ξ}{\ensuremath{\xi}}
\newcommand\Aamp{\AgdaFunction{\ensuremath{\&}}}
\newcommand\Att{\AgdaInductiveConstructor{tt}}
\newcommand\Ado{\AgdaKeyword{do}}
\newcommand\AZ{\AgdaDatatype{ℤ}}
\newcommand\Aint{\AgdaInductiveConstructor{int}}
\newcommand\Abool{\AgdaInductiveConstructor{bool}}
\newcommand\Atend{\AgdaInductiveConstructor{end}}
\newcommand\Atsend[2]{\AgdaInductiveConstructor{‼{\textcolor{black}{\ensuremath{#1}}}∙{\textcolor{black}{\ensuremath{#2}}}}}
\newcommand\Atrecv[2]{\AgdaInductiveConstructor{⁇{\textcolor{black}{\ensuremath{#1}}}∙{\textcolor{black}{\ensuremath{#2}}}}}
\newcommand\Atcfsend[1]{\AgdaInductiveConstructor{‼{\textcolor{black}{\ensuremath{#1}}}}}
\newcommand\Atcfrecv[1]{\AgdaInductiveConstructor{⁇{\textcolor{black}{\ensuremath{#1}}}}}
\newcommand\Atcfcomp[2]{\AgdaInductiveConstructor{{\textcolor{black}{\ensuremath{#1}}};{\textcolor{black}{\ensuremath{#2}}}}}
\newcommand\Atcfskip{\AgdaInductiveConstructor{skip}}
\newcommand\ACEND{\AgdaInductiveConstructor{END}}
\newcommand\ACCLOSE{\AgdaInductiveConstructor{CLOSE}}
\newcommand\ACfork{\AgdaInductiveConstructor{fork}}
\newcommand\ACconnect{\AgdaInductiveConstructor{connect}}
\newcommand\ACterminate{\AgdaInductiveConstructor{terminate}}
\newcommand\ACdelegateIN{\AgdaInductiveConstructor{delegateIN}}
\newcommand\ACdelegateOUT{\AgdaInductiveConstructor{delegateOUT}}
\newcommand\ACtransmit{\AgdaInductiveConstructor{transmit}}
\newcommand\ACbranch{\AgdaInductiveConstructor{branch}}
\newcommand\ACclose{\AgdaInductiveConstructor{close}}
\newcommand\ACSEND{\AgdaInductiveConstructor{SEND}}
\newcommand\ACRECV{\AgdaInductiveConstructor{RECV}}
\newcommand\ACSELECT{\AgdaInductiveConstructor{SELECT}}
\newcommand\ACCHOICE{\AgdaInductiveConstructor{CHOICE}}
\newcommand\AFin{\AgdaDatatype{Fin}}
\newcommand\AXCommand{\AgdaDatatype{XCmd}}
\newcommand\ACommand{\AgdaDatatype{Cmd}}
\newcommand\ASession{\AgdaDatatype{Session}}
\newcommand\ASplit{\AgdaDatatype{Split}}
\newcommand\AMSession{\AgdaDatatype{MSession}}
\newcommand\ASetOne{\AgdaDatatype{Set$_1$}}
\newcommand\ACheck{\AgdaFunction{Check}}
\newcommand\ACausality{\AgdaFunction{Causality}}
\newcommand\ACheckDual{\AgdaFunction{CheckDual0}}
\newcommand\Aexec{\AgdaFunction{exec}}
\newcommand\Amu{\AgdaInductiveConstructor{\ensuremath{\mu}}}
\newcommand\AMU{\AgdaInductiveConstructor{LOOP}}
\newcommand\AUNROLL{\AgdaInductiveConstructor{UNROLL}}
\newcommand\ACONTINUE{\AgdaInductiveConstructor{CONTINUE}}
\newcommand\Aback{\AgdaInductiveConstructor{\ensuremath{`}}}
\newcommand\Amanyunaryp{\AgdaFunction{many-unaryp}}
\newcommand\Arestart{\AgdaFunction{restart}}
\newcommand\ASerialize{\AgdaRecord{Serialize}}
\newcommand\ARawMonad{\AgdaRecord{RawMonad}}
\newcommand\AReaderT{\AgdaRecord{ReaderT}}
\newcommand\AStateT{\AgdaRecord{StateT}}
\newcommand\Aput{\AgdaFunction{put}}
\newcommand\Amodify{\AgdaFunction{modify}}
\newcommand\Aget{\AgdaFunction{get}}
\newcommand\Aproject{\AgdaFunction{project}}
\newcommand\AlocateSplit{\AgdaFunction{locate-split}}
\newcommand\Aadjust{\AgdaFunction{adjust}}
\newcommand\Aid{\AgdaFunction{id}}
\newcommand\AprimRecv{\AgdaFunction{primRecv}}
\newcommand\Aleafp{\AgdaFunction{leafp}}
\newcommand\Abranchp{\AgdaFunction{branchp}}
\newcommand\Atreep{\AgdaFunction{treep}}
\newcommand\AIntTree{\AgdaDatatype{IntTree}}
\newcommand\AIntTreeF{\AgdaFunction{IntTreeF}}
\newcommand\ACLeaf{\AgdaInductiveConstructor{Leaf}}
\newcommand\ACBranch{\AgdaInductiveConstructor{Branch}}
\newcommand\AtoN{\AgdaFunction{toℕ}}
\newcommand\Asplit{\AgdaFunction{split}}
\newcommand\AisValue{\AgdaDatatype{isValue}}
\newcommand\AValue{\AgdaDatatype{Value}}
\newcommand\Ajoin{\AgdaFunction{join}}
\newcommand\AExpr{\AgdaDatatype{Expr}}
\newcommand\AVar{\AgdaDatatype{Var}}
\newcommand\ACExpr{\AgdaFunction{CExpr}}
\newcommand\Acont{\AgdaFunction{cont}}
\newcommand\AlevelEnv{\AgdaFunction{levelEnv}}
\newcommand\Adollar{\AgdaOperator{\AgdaFunction{\AgdaUnderscore{}\$\AgdaUnderscore{}}}}
\newcommand\AVSem{\AgdaOperator{\AgdaFunction{𝓥⟦\AgdaUnderscore{}⟧}}}
\newcommand\AESem{\AgdaOperator{\AgdaFunction{𝓔⟦\AgdaUnderscore{}⟧}}}
\newcommand\AESemx[1]{\AgdaOperator{\AgdaFunction{𝓔⟦#1⟧}}}
\newcommand\ADSemx[1]{\AgdaOperator{\AgdaFunction{𝓓⟦#1⟧}}}
\newcommand\AGSem{\AgdaOperator{\AgdaDatatype{𝓖⟦\AgdaUnderscore{}⟧}}}
\newcommand\ATSem{\AgdaOperator{\AgdaDatatype{𝓣⟦\AgdaUnderscore{}⟧}}}
\newcommand\Ainn{\AgdaDatatype{inn}}
\newcommand\ADEnv{\AgdaDatatype{Env\ensuremath{^*}}}
\newcommand\AapplyEnv{\AgdaFunction{apply-env}}
\newcommand\AREL{\AgdaFunction{REL}}
\newcommand\ARelEnv{\AgdaFunction{RelEnv}}
\newcommand{\AV}{\AgdaFunction{𝓥}}
\newcommand{\Asubst}{\AgdaFunction{subst}}
\newcommand{\Asubstlo}{\AgdaFunction{substl$\omega$}}
\newcommand{\Acong}{\AgdaFunction{cong}}
\newcommand{\Atrans}{\AgdaFunction{trans}}
\newcommand{\Arefl}{\AgdaInductiveConstructor{refl}}
\newcommand{\AValueDown}{\AgdaFunction{Value-\ensuremath{\Downarrow}}}
\newcommand{\AGLookup}{\AgdaFunction{𝓖-lookup}}
\newcommand{\ACsubClosed}{\AgdaFunction{Csub-closed}}
\newcommand{\ATsubClosed}{\AgdaFunction{Tsub-closed}}
\newcommand{\Ahere}{\AgdaInductiveConstructor{here}}
\newcommand{\Athere}{\AgdaInductiveConstructor{there}}
\newcommand{\Anull}{\AgdaInductiveConstructor{∅}}
\newcommand{\ATRen}{\AgdaFunction{TRen}}
\newcommand{\ATSub}{\AgdaFunction{TSub}}
\newcommand{\ATren}{\AgdaFunction{Tren}}
\newcommand{\ATsub}{\AgdaFunction{Tsub}}
\newcommand{\ATliftR}{\AgdaFunction{Tliftᴿ}}
\newcommand{\ATliftS}{\AgdaFunction{Tliftˢ}}
\newcommand{\ATidR}{\AgdaFunction{Tidᴿ}}
\newcommand{\ATidS}{\AgdaFunction{Tidˢ}}
\newcommand{\AERen}{\AgdaFunction{ERen}}
\newcommand{\AESub}{\AgdaFunction{ESub}}
\newcommand{\AEren}{\AgdaFunction{Eren}}
\newcommand{\AEsub}{\AgdaFunction{Esub}}
\newcommand{\AEliftR}{\AgdaFunction{Eliftᴿ}}
\newcommand{\AEliftS}{\AgdaFunction{Eliftˢ}}
\newcommand{\AEliftRL}{\AgdaFunction{Eliftᴿ-l}}
\newcommand{\AEliftSL}{\AgdaFunction{Eliftˢ-l}}
\newcommand{\AEidR}{\AgdaFunction{Eidᴿ}}
\newcommand{\AEidS}{\AgdaFunction{Eidˢ}}
\newcommand{\AcompositionalityTSubTSub}{\AgdaFunction{compositionality-TSub-TSub}}
\newcommand{\AcompositionalityESubESub}{\AgdaFunction{compositionality-ESub-ESub}}
\newcommand{\ALRVsub}{\AgdaFunction{LRVsub}}
\newcommand{\AEEsingleSubstPreserves}{\AgdaFunction{EEsingle-subst-preserves}}
\newcommand{\AETsingleSubstPreserves}{\AgdaFunction{ETsingle-subst-preserves}}

\newcommand*\ACode[1]{\AgdaFontStyle{#1}}
\newcommand*\AField[1]{\AgdaField{#1}}
\newcommand*\ACon[1]{\AgdaInductiveConstructor{#1}}
\newcommand*\AKw[1]{\AgdaKeyword{#1}}
\newcommand*\AFun[1]{\AgdaFunction{#1}}
\newcommand*\ACom[1]{\AgdaComment{#1}}

\newcommand\ALvl{\AgdaDatatype{Lvl}}
\newcommand\ASet{\AgdaDatatype{Set}}
\newcommand\AProp{\AgdaDatatype{Prop}}
\newcommand\ASetzero{\AgdaDatatype{zero}}
\newcommand\ASetsuc{\AgdaDatatype{suc}}
\newcommand\ASetlub{\AgdaDatatype{⊔}}
\newcommand\ASeti[1]{\AgdaDatatype{Set}\ensuremath{\,{#1}}}
\newcommand\ASeto{\AgdaDatatype{Setω}}
\newcommand\ASetoo{\AgdaDatatype{Setω\ensuremath{_1}}}
\newcommand\ASetooo{\AgdaDatatype{Setω\ensuremath{_2}}}
\newcommand\Acoe{\AgdaFunction{coe}}
\newcommand\ALevel{\AgdaFunction{Level}}
\newcommand\Atskip{\AgdaInductiveConstructor{tskip}}
\newcommand\Alskip{\AgdaInductiveConstructor{lskip}}
\newcommand{\ATwk}{\AgdaFunction{Twk}}
\newcommand\Aeq{\AgdaFunction{eq}}
\newcommand\AMode{\AgdaDatatype{Mode}}
\newcommand\Afin{\AgdaInductiveConstructor{fin}}
\newcommand\Aany{\AgdaInductiveConstructor{any}}
\newcommand\Asuc{\AgdaInductiveConstructor{suc}}
\newcommand\Azero{\AgdaInductiveConstructor{zero}}
\newcommand\AMutualOrd{\AgdaDatatype{MutualOrd}}
\newcommand\AAcc{\AgdaDatatype{Acc}}
\newcommand\Aacc{\AgdaInductiveConstructor{acc}}
\newcommand\Alookup{\AgdaFunction{lookup}}
\newcommand\ALwk{\AgdaFunction{Lwk}}
\newcommand\ANat{\AgdaInductiveConstructor{Nat}}
\newcommand\AElir{\AgdaDatatype{El\ensuremath{^{ir}}}}
\newcommand\AUp{\AgdaInductiveConstructor{U'}}
\newcommand\APi{\AgdaInductiveConstructor{\ensuremath{\Pi}'}}
\newcommand\ANatp{\AgdaInductiveConstructor{ℕ'}}
\newcommand\ANN{\AgdaInductiveConstructor{ℕ}}
\newcommand\ATL{\AgdaFunction{TL}}
\newcommand\AElLiftLe{\AgdaFunction{ElLift\ensuremath{\le}}}
\newcommand\ATTT{\AgdaFunction{⟦[]TT⟧T}}
\newcommand\ALTT{\AgdaFunction{⟦[]LT⟧T}}
\newcommand\Acrucial{\AgdaFunction{crucial}}
\newcommand\AtoLevel{\AgdaFunction{toLevel}}
\newcommand\ALeLubLub{\AgdaFunction{≤-lublub}}
\newcommand\ALtSucLim{\AgdaFunction{<-suc-lim}}
\newcommand\ALim{\AgdaDatatype{Lim}}
\newcommand\ALift{\AgdaDatatype{Lift}}
\newcommand\ABoundedLevel{\AgdaDatatype{BoundedLevel}}

%%% Local Variables:
%%% mode: latex
%%% TeX-master: "main"
%%% End:

\input{agda-generated}
\input{agda-generated-examples}

\definecolor{agdablue}{HTML}{0000CD}
\DisableLigatures[-]{encoding=T1}
\newcommand{\bsym}[1]{\textcolor{agdablue}{#1}}
\newcommand{\tdot}{\mkern2mu \textcolor{agdablue}{\cdotp} \mkern-2mu \textcolor{agdablue}{\cdotp} \mkern-2mu \textcolor{agdablue}{\cdotp} \mkern1mu}


\begin{document}

\title{Agdasubst: Automated Reasoning for Unified Renaming and Substitution over Scoped and Multi-Sorted Syntax}

\author{Marius Weidner}
\email{weidner@cs.uni-freiburg.de}
\affiliation{%
      \institution{University of Freiburg}
      \country{Germany}}
\orcid{0009-0008-1152-165X} % chktex 8
\authornote{
      I hereby declare that I am the sole author and composer of my thesis and that
      no other sources or learning aids, other than those listed, have been used.\\
      Furthermore, I declare that I have acknowledged the work of others by providing
      detailed references of said work.\\
      I hereby also declare that my thesis has not been prepared for another
      examination or assignment, either wholly or excerpts thereof.
      \\
      \\
      \begin{tabular}{p{\textwidth/2} l}
            Freiburg, XX.XX.2025       & yadda yadda                \\ % \includegraphics[width=0.1\textwidth]{signature.png} \\
            \rule{\textwidth/3}{0.4pt} & \rule{\textwidth/3}{0.4pt} \\
            Place, Date                & Signature
      \end{tabular}
}

\begin{abstract}

      Mechanizing the metatheory of programming languages in proof assistants poses
      recurring challenges for variable binding and substitution. Every formalization
      requires proving standard lemmas about substitutions and applying them
      correctly. Explicit substitution calculi, for instance the $\sigma$-calculus,
      provide syntactic reduction rules that ensure confluence and strong
      normalization, allowing equations between terms to be compared via normal forms
      and giving an automatic procedure that eliminates manual lemma application.

      Directly embedding explicit substitution laws as reduction rules in Agda using
      its rewrite mechanism is challenging due to the need for precise control over
      which definitions reduce and which remain symbolic, the dependence of
      substitutions on renamings for structurally recursive definitions, and the
      requirement to support multiple variable sorts. We address the latter by
      adapting the $\sigma$-calculus to intrinsically scoped, multi-sorted syntax,
      first handling renamings and substitutions separately, then presenting a
      uniform treatment that allows results about renamings and substitutions to be
      proven once for both, avoiding duplication.
\end{abstract}

\maketitle

\section{Introduction}\label{sec:introduction}

Mechanizing the metatheory of programming languages within interactive theorem
provers often presents recurring uninteresting technical challenge, namely the
handeling of variable binding and substitution. While such details are usually
ignored in pen-and-paper proofs, their mechanization in proof assistants
requires precise definitions and a significant amount of manual effort.

Historically, substitution has been treated as an meta-level operation. To
formalize this, explicit substitution calculi~\cite{10.1145/96709.96712} were
introduced to bring substitutions down to a syntactic level. These calculi
define substitution behavior through a set of reduction rules. With the
$σ$-calculus, a terminating, confluent~\cite{10.1145/226643.226675} and
complete~\cite{10.1145/2676724.2693163} rewrite system for $λ$-calculus terms
with substitution has been discovered. Repeatedly applying the rules
left-to-right yields a unique normal form, which can then be syntactically
compared for equality.

The $σ$-calculus laws generalize to custom syntaxes without much
difficulty~\cite{schafer2015autosubst}, and it is tempting to assume that this
would also make it easy to embed these laws into proof assistants and then
solve equations between terms with substitution automatically by applying them
from left to right, either via tactics or embedding as native reduction. In
practice, however, when we try to embed the $\sigma$-calculus into
Agda~\cite{norellPracticalProgrammingLanguage2007}, two major challenges
remain.

First, we need to distinguish between renamings, that is substitutions mapping
variables to variables, and general substitutions, which map variables to
terms, in order to obtain structurally recursive
definitions~\cite{10.1007/11617990_1}. In fact, this separation is useful in
any way, there are situations where we want to reason about renamings alone.
However, the separation leads to code duplication, since lemmas such as
substitution preserves typing must be proven twice, once for the special case
of renamings, and once again for substitutions, relying on the result about
renamings.

Second, in languages with multiple variable sorts, the combination of several
syntactic categories and the separation of renamings and substitutions can lead
to a combinatorial explosion of near-identical substitution definitions and
lemmas.

When choosing a syntax representation in Agda, several options are available:
extrinsically scoped and typed, intrinsically scoped but extrinsically typed,
or intrinsically scoped and typed. We argue that the intrinsically scoped but
extrinsically typed approach is the sweatspot between these extremes. In Agda,
using a multi-sorted, intrinsically scoped syntax helps to avoid the
combinatorial explosion of lemmas that arises from having multiple variable
sorts, since we can define parallel, multi-sorted renaming and substitution
uniformly across them. As we will see, adapting the σ-calculus to such a
setting with multi-sorted, scoped syntax and first-class renamings is not
particularly difficult when done carefully, yet it already yields significant
benefits.

The remaining challenge caused by the separation of renamings and substitutions
is more difficult to address. We build on the notion of Kits~\cite{ren-sub,
      stronglytyped, altenkirch2025substitution}, a structure that captures the
behaviour common to variables and terms, originally in an intrisnically typed
setting. From Kits, one can define a uniform structure for renamings and
substitutions, called Maps. This idea was later extended to multi-sorted and
scoped syntax~\cite{saffrich:LIPIcs.ITP.2024.32}. Our work follows the latter
developments. We adapt the $σ$-calculus to a setting with Kits and Maps over
multi-sorted and scoped syntax, treating renamings and substitutions uniformly.

\subsection*{Structure}
The rest of this paper is structured in the following way:
\begin{itemize}
      \item \cref{sec:pre} introduces the theory and tools used in this work;
      \item \cref{sec:mis} shows how to embed a scoped and multi-sorted version of the $\sigma$-calculus in Agda, treating renamings and substitutions separately;
      \item \cref{sec:ags} explains the internals of the \textsc{Agdasubst} library, where renamings and substitutions can be treated uniformly;
      \item \cref{sec:ext} shows how to use the \textsc{Agdasubst} library and includes case studies;
      \item \cref{sec:rel} compares the \textsc{Agdasubst} library to similar works;
      \item \cref{sec:wrk} outlines possible future improvements to the \textsc{Agdasubst} library.
\end{itemize}

\noindent\textbf{The supplement contains the complete Agda code underlying this paper.}

\section{Preliminaries}\label{sec:pre}
This section introduces the concepts and tools required for the developments in
this work. We summarize the following components:
\begin{itemize}
      \item Lesser known Agda features: instance arguments, opaque blocks, and rewrite
            rules;
      \item The $\sigma$-calculus;
      \item Multi-sorted and scoped syntaxes in Agda.
\end{itemize}

\subsection{Agda}\label{sec:pre-agd} Agda is a dependently typed programming language and proof assistant based on
Martin-Löf type theory~\cite{MARTINLOF197573}. It allows users to write
programs and proofs in a functional language, verifying correctness through
type checking.

In addition to its well-known features such as indexed and parameterized
inductive data types, dependent (co)pattern matching~\cite{10.1145/3236770},
implicit arguments and records, Agda supports three less commonly used but
important features relevant to this work.

\subsection*{Instance Arguments}

Instance arguments, written using double braces \AgdaSymbol{\{\{\dots\}\}},
differ from implicit arguments \AgdaSymbol{\{\dots\}} in that they are
automatically resolved via an instance search algorithm, rather than the more
simple implicit argument resolution limited to the current scope. When a
function expects an instance argument, Agda attempts to infer and insert it
from values marked with the \AgdaKeyword{instance} keyword. This mechanism
enables type-directed programming patterns similar to type classes in
Haskell~\cite{10.1145/75277.75283}. As an example, consider the following
definition of a record for types with a default value: \EDefault{}The record
can be opened using instance argument resolution for its fields:
\EDefFields{}We can now declare an instance for natural numbers:
\EDefInst{}This enables implicit usage of the default value via instance
search: \EDefEx{} Here, Agda infers from the type \AgdaDatatype{ℕ} of
\AgdaNumber{0} that the instance \AgdaFunction{default–ℕ} applies for
\AgdaFunction{default}, inserting it automatically to conclude the equation
holds definitionally.

\subsection*{Opaque Blocks}

Agda supports opaque definitions~\cite{gratzer2022controlling}, which allow
certain parts of the code to be treated as non-unfolding symbols, similar to
abstract definitions. This feature is useful for controlling definitional
equality and can be used to prevent specific definitions from being unfolded
during type or conversion checking.

A block is marked as opaque by preceding it with the \AgdaKeyword{opaque}
keyword. All definitions within the block become opaque, meaning they are
excluded from definitional equality checks. However, their names and types
remain visible and can still be used throughout the code. As an example
consider: \EOpaque{}In this example, \AgdaFunction{forty-two} is defined as the
numeral \AgdaNumber{42}, but it is opaque outside the block. This means that
although \AgdaFunction{forty-two} still has type \AgdaDatatype{ℕ} and can be
used in expressions and proofs, it does not reduce to \AgdaNumber{42} outside
the block during definitional equality checks: \EOpaqueExO{}In subsequent
opaque blocks, the type checker can be instructed to selectively unfold
previously opaque definitions using the \AgdaKeyword{unfolding} keyword.
Unfolding the definition will reduce the definition to \AgdaNumber{42} again,
as expected: \EOpaqueExT{}

\subsection*{Rewrite Rules}

In the context of this work, the most import feature of Agda are user defined
rewrite rules~\cite{10.1145/3434341, cockx:LIPIcs.TYPES.2019.2} that extend the
system’s notion of definitional equality by integrating them directly into type
checking. A function definition in Agda qualifies to be added as rewrite rule,
if it is of the form
\[
      \AgdaFunction{eq} : (x_1 : A_1)\, \ldots\, (x_k : A_k) \to \AgdaFunction{f}\, p_1\, \ldots\, p_n = v
\]
and satisfies the following three conditions:
\begin{enumerate}
      \item The head symbol \AgdaFunction{f} must be a postulate, a defined function, or a
            constructor applied to fully general parameters, i.e., the parameters must be
            distinct variables.
      \item Each variable $x_1, \ldots, x_k$ must occur at least once in a pattern position
            among $p_1, \ldots, p_n$\footnote{See
                  \url{https://agda.readthedocs.io/en/latest/language/rewriting.html} for a
                  precise definition of pattern positions.}.
      \item The left side $\AgdaFunction{f}\, p_1\, \ldots\, p_n$ must be neutral, i.e., it
            should not reduce any further under existing definitions.
\end{enumerate}

As an example, consider the right identity law for addition: \ERewrite{}The
type of the lemma satisfies the conditions above and can be registered as a
rewrite rule using the \AgdaKeyword{REWRITE} pragma: \ERewriteIt{}From this
point forward, Agda treats $n$ \AgdaFunction{+} \AgdaNumber{0} as
definitionally equal to $n$: \ERewriteEx{}To ensure consistency, Agda requires
that all rewrite rules satisfy the three conditions stated above, and that the
combined set of rules is both terminating and confluent. However, termination
is not currently checked by Agda, it is only observable indirectly when type
checking fails to terminate. Confluence can be checked using the
\texttt{--local-confluence-check} and \texttt{--confluence-check} flags, but
both checks are necessarily conservative.

\subsection{The Sigma Calculus}\label{sec:pre-sig}
\begin{figure}[t]
      \centering
      \small
      {\raggedright \textbf{\Large Symbols} \par}
      \vspace{1.5em}

      \begin{minipage}[t]{0.48\textwidth}
            \raggedright
            \textbf{Variables} $x$
            \begin{align*}
                  \text{innermost variable} \quad & \textcolor{agdablue}{\mathsf{zero}}    \\
                  \text{outer variables} \quad    & \textcolor{agdablue}{\mathsf{suc}}\; x
            \end{align*}\\
            \vspace{1em}
            \textbf{Terms} $t$
            \begin{align*}
                  \text{variable usage} \quad           & \textcolor{agdablue}{\texttt{\`{}}}x  \\
                  \text{abstraction} \quad              & \textcolor{agdablue}{\lambda.}\, t    \\
                  \text{application} \quad              & t_1 \ t_2                             \\
                  \text{substitution lookup} \quad      & x \textcolor{agdablue}{\;\&\;} \sigma \\
                  \text{substitution application} \quad & t \tdot{} \sigma
            \end{align*}
      \end{minipage}
      \hfill
      \begin{minipage}[t]{0.48\textwidth}
            \raggedright
            \textbf{Substitutions} $\sigma$
            \begin{align*}
                  \text{identity} \quad    & \textcolor{agdablue}{\mathsf{id}}              \\
                  \text{weakening} \quad   & \textcolor{agdablue}{\mathsf{wk}}              \\
                  \text{extension} \quad   & t \textcolor{agdablue}{\cdot} \sigma           \\
                  \text{composition} \quad & \sigma_1 \ \textcolor{agdablue}{;} \  \sigma_2
            \end{align*}
      \end{minipage}

      {\raggedright \vspace{2.0em} \textbf{\Large Reduction} \par}
      \vspace{1.5em}
      \begin{minipage}[t]{0.48\textwidth}
            \raggedright
            \textbf{Definitional Laws}
            \begin{alignat*}{2}
                  \mathsf{id}      \quad          & x \textcolor{agdablue}{\;\&\;} \textcolor{agdablue}{\mathsf{id}}                                          &  & = \textcolor{agdablue}{\texttt{\`{}}} x                    \\
                  \mathsf{wk}      \quad          & x \textcolor{agdablue}{\;\&\;} \textcolor{agdablue}{\mathsf{wk}}                                          &  & = \texttt{\`{}}(\textcolor{agdablue}{\mathsf{suc}} \ x)    \\
                  \mathsf{ext}_\mathsf{Z}  \quad  & \textcolor{agdablue}{\mathsf{zero}} \textcolor{agdablue}{\;\&\;} (t \textcolor{agdablue}{\cdot} \sigma)   &  & = t                                                        \\
                  \mathsf{ext}_\mathsf{S}   \quad & \textcolor{agdablue}{\mathsf{suc}}\;x \textcolor{agdablue}{\;\&\;} (t \textcolor{agdablue}{\cdot} \sigma) &  & = x \textcolor{agdablue}{\;\&\;} \sigma                    \\
                  \mathsf{comp}    \quad          & x \textcolor{agdablue}{\;\&\;} (\sigma_1 \ \textcolor{agdablue}{;} \  \sigma_2)                           &  & = (x \textcolor{agdablue}{\;\&\;} \sigma_1) \tdot \sigma_2
            \end{alignat*}\\
            \vspace{1em}
            \textbf{Traversal Laws}
            \begin{alignat*}{2}
                  \mathsf{var} \quad & (\textcolor{agdablue}{\texttt{\`{}}} x)                                            & \tdot \sigma
                                     & = x \textcolor{agdablue}{\;\&\;} \sigma                                                            \\
                  \mathsf{lam} \quad & (\textcolor{agdablue}{\lambda.}\, t)                                               & \tdot  \sigma
                                     & = \textcolor{agdablue}{\lambda.} (t  \tdot (\textcolor{agdablue}{\uparrow}\sigma))                 \\
                  \mathsf{app} \quad & (t_1 \ t_2)                                                                        & \tdot \sigma
                                     & = (t_1  \tdot \sigma) \ (t_2  \tdot \sigma)
            \end{alignat*}

      \end{minipage}
      \hfill
      \begin{minipage}[t]{0.48\textwidth}
            \raggedright
            \textbf{Interaction Laws}
            \begin{alignat*}{2}
                  \mathsf{comp\text{-}id}_\mathsf{L} \quad & \textcolor{agdablue}{\mathsf{id}} \ \textcolor{agdablue}{;} \  \sigma                                                                                                         &  & = \sigma                                                                                                                                                                                                           \\
                  \mathsf{comp\text{-}id}_\mathsf{R} \quad & \sigma \ \textcolor{agdablue}{;} \  \textcolor{agdablue}{\mathsf{id}}                                                                                                         &  & = \sigma                                                                                                                                                                                                           \\
                  \mathsf{associativity}          \quad    & (\sigma_1 \ \textcolor{agdablue}{;} \  \sigma_2) \ \textcolor{agdablue}{;} \  \sigma_3                                                                                        &  & = \sigma_1 \ \textcolor{agdablue}{;} \  (\sigma_2 \ \textcolor{agdablue}{;} \  \sigma_3)                                                                                                                           \\
                  \mathsf{distributivity}           \quad  & (t \textcolor{agdablue}{\cdot} \sigma_1) \ \textcolor{agdablue}{;} \  \sigma_2                                                                                                &  & = (t \mkern2mu \textcolor{agdablue}{\cdotp} \mkern-2mu \textcolor{agdablue}{\cdotp} \mkern-2mu \textcolor{agdablue}{\cdotp} \sigma_2) \textcolor{agdablue}{\cdot} (\sigma_1 \ \textcolor{agdablue}{;} \  \sigma_2) \\
                  \mathsf{interact}       \quad            & \textcolor{agdablue}{\mathsf{wk}} \ \textcolor{agdablue}{;} \  (t \textcolor{agdablue}{\cdot} \sigma)                                                                         &  & = \sigma                                                                                                                                                                                                           \\
                  \eta\mathsf{\text{-}id}        \quad     & (\textcolor{agdablue}{\texttt{\`{}}\mathsf{zero}}) \textcolor{agdablue}{\cdot} \textcolor{agdablue}{\mathsf{wk}}                                                              &  & = \textcolor{agdablue}{\mathsf{id}}                                                                                                                                                                                \\
                  \eta\mathsf{\text{-}law}       \quad     & (\textcolor{agdablue}{\mathsf{zero}} \textcolor{agdablue}{\;\&\;} \sigma) \textcolor{agdablue}{\cdot} (\textcolor{agdablue}{\mathsf{wk}} \ \textcolor{agdablue}{;} \  \sigma) &  & = \sigma
            \end{alignat*}\\
            \vspace{1em}
            \textbf{Monad Laws}
            \begin{alignat*}{2}
                  \mathsf{right\text{-}id} \quad          & t  \tdot
                  \mathsf{id}                             &           & = t        \\
                  \mathsf{compositionality}         \quad & (t  \tdot
                  \sigma_1)  \tdot
                  \sigma_2                                &           & = t  \tdot
                  (\sigma_1 ; \sigma_2)
            \end{alignat*}
      \end{minipage}
      \caption{The three-sorted $\sigma$-calculus.}
      \label{fig:pre-sig}
\end{figure}

Explicit substitution calculi were introduced to formalize the traditionally
meta-theoretic treatment of substitutions. Unlike the standard approach, where
substitution is defined at the meta-level, these calculi treat substitution
primitives as syntactic constructs and define their behavior by reduction
rules. Several explicit substitution systems have been proposed. The
$\sigma$-calculus~\cite{10.1145/96709.96712}, the first of its kind, introduces
substitution primitives such as identity, extension, shifting, and composition.
The $\sigma^\Uparrow$-calculus~\cite{10.1145/226643.226675} extends this with a
primitive for lifting.

The $\sigma_{SP}$-calculus is of particular interest because it includes
additional rules to ensure confluence. It is sound, confluent, and
complete~\cite{schafer2015autosubst}. As a result, any valid equality involving
substitutions can be derived using its rules alone. In the following, we
present our own three-sorted version of the $\sigma$-calculus, using the
$\lambda$-calculus as the object language in \cref{fig:pre-sig}. Our system
closely follows the $\sigma_{SP}$-calculus.

The syntax is sorted into de~Bruijn variables, terms, and substitution
primitives. We use de~Bruijn variables to obtain $\alpha$-equivalence for free.
The expression $\textcolor{agdablue}{\uparrow} \sigma$ is syntactic sugar for
$(\textcolor{agdablue}{\texttt{\`{}}}\textcolor{agdablue}{\mathsf{zero}})
      \textcolor{agdablue}{\cdot} (\sigma \ \textcolor{agdablue}{;} \
      \textcolor{agdablue}{\mathsf{wk}})$. Unlike most presentations in the
literature, we include explicit syntax for variable usage within terms. This
requires distinguishing between \bsym{\textsf{suc}} and \bsym{\textsf{wk}},
depending on whether they appear in the sort of variables or substitutions. The
reduction rules are grouped into four categories.
\begin{itemize}
      \item \textbf{Definitional Laws} describe how variables are looked up in
            substitutions;
      \item \textbf{Interaction Laws} express equalities between substitution expressions;
      \item \textbf{Traversal Laws} describe how substitutions propagate through term syntax and under binders;
      \item \textbf{Monad Laws} describe the behavior of substitutions when applied to terms.
\end{itemize}
We do not include reduction rules for the $\lambda$-calculus itself, such as
$\beta$-reduction, as done for example in the $\lambda\sigma$-calculus~\cite{10.1145/226643.226675}.
Our focus is not on the reduction of the object language, but on the behavior
of substitutions applied to terms. The reduction strategy of the object
language is part of its definition and is left unspecified. The only rules involving the syntax that we require are those traversing terms, when a substitution is applied.

We believe the three-sorted $\sigma$-calculus is terminating and confluent.
Although we do not give a formal proof, it is a reasonable variant of the
$\sigma_{SP}$-calculus. Renamings, needed to define substitutions structurally
in Agda, are believed to break completeness~\cite{10.1145/3293880.3294101}. For
this reason, we do not consider completeness a central property in this work.
Still, we try to keep the space of solvable equations as large as possible.

\subsection{Mutli-Sorted and Scoped Syntaxes}\label{sec:pre-syn}
\begin{figure}[t]
      \centering
      \begin{minipage}[t]{0.48\linewidth}
            \raggedright{}
            \EScoped{}
      \end{minipage}
      \hfill
      \begin{minipage}[t]{0.48\linewidth}
            \raggedright{}
            \EMultiSorted{}
      \end{minipage}
      \caption{Classically Scoped Syntax vs. Multi-Sorted and Scoped Syntax}
      \label{fig:pre-svm}
\end{figure}

Unlike the simply typed $\lambda$-calculus, many formal languages have more
than one sort of syntax. For example, System~F~\cite{girard1972} has kinds,
types and expressions. Further, to avoid scoping related errors, we want
syntaxes to enforce correct scoping intrinsically. This is usually done in Agda
by a definition of the syntax as a family of types indexed by the set of its
free variables.

On the left side of \cref{fig:pre-svm}, we start with a standard intrinsically
scoped syntax for System~F. Types are indexed by the number of free type
variables, and expressions are indexed by both the number of free type and
expression variables. Variables are represented using de Bruijn indices of type
\AgdaDatatype{Fin} $n$, that is, as natural numbers strictly less than the
number of free variables. This guarantees that only well scoped terms can be
formed.

This approach is simple but has drawbacks. Types and expressions are separate
datatypes, so we would need to define multiple substitution operations:
type-in-type, expr-in-expr, and type-in-expr. These substitutions need to be
handled and related manually, which adds boilerplate.

To avoid this, we use a multi-sorted syntax. Terms form a single datatype
indexed by a list of variable sorts and a target sort. The type $S$ \AgdaDatatype{⊢} $s$
describes a term of sort $s$ with free variables of sorts in $S$. Variable
positions are given by $S$ \AgdaDatatype{∋} $s$, a proof that a variable of sort $s$ appears
in $S$ at some position. This position acts as the de Bruijn index. This gives
us well scoped and well sorted terms by construction. The right side of
\cref{fig:pre-svm} shows this version. With this unified syntax, we can define
scoped and multi-sorted renamings and substitutions that work uniformly across
sorts.

\section{The Multi-Sorted and Scoped Sigma Calculus with First-Class Renamings}\label{sec:mis}

This section gives a direct embedding of the $\sigma$-calculus rules as native
reduction rules in Agda using the rewrite mechanism for the
muti-sorted and scoped version of System~F from \cref{fig:pre-svm} as an
example for any custom syntax. When adapting the $\sigma$-calculus from
\cref{sec:pre-sig} to custom multi-sorted and scoped syntaxes in Agda, we face
the problem that, in Agda, substitutions for syntaxes with binders cannot be
defined directly by structural recursion. To obtain well-founded definitions,
we need to define renamings first, which are substitutions that only map to
variables.muti-sorted and scoped version of System~F from \cref{sec:pre-syn} as
an example for any custom syntax.

When adapting the $\sigma$-calculus from \cref{sec:pre-sig} to custom
multi-sorted and scoped syntaxes in Agda, we face the problem that, in Agda,
substitutions for syntaxes with binders cannot be defined directly by
structural recursion. To obtain well-founded definitions, we need to define
renamings first, which are substitutions that only contain variables. After we
defined renamings and substitutions, we revisit the $\sigma$-calculus and adapt
its laws to this setting.

\subsection{Scoped and Multi-Sorted Renamings and Substitutions}\label{sec:mis-sub}
\begin{figure}[t]
      \centering
      \small
      \begin{minipage}[t]{0.48\linewidth}
            \small
            \raggedright{}
            \ERen{}
      \end{minipage}
      \hfill
      \begin{minipage}[t]{0.48\linewidth}
            \small
            \raggedright{}
            \ESub{}
      \end{minipage}
      \caption{Scoped and Multi-Sorted Renamings and Substitutions}
      \label{fig:mis-ras}
\end{figure}

We define renamings and substitutions uniformly over all variable sorts by
indexing them over scopes, as shown in \cref{fig:mis-ras}. A renaming from
scope $S_1$ to $S_2$ maps each variable of a given sort in $S_1$ to a variable
of the same sort in $S_2$. Similarly, a substitution maps variables to terms of
the corresponding sort.

The order of definitions is chosen carefully. We first define the renaming type
and its primitive operations: identity, extension and weakening. Substitutions
are then defined in terms of renamings, using weakening to define lifting.

To support all composition combinations of renamings and substitutions, we
provide four forward composition primitives: renaming composed with renaming,
renaming with substitution\footnote{The first two composition operators can
      technically be defined as a single definition together with composition of two
      renamings, they are simply forward function composition, but we choose to keep
      them separated.}, substitution with renaming, and substitution with
substitution.

The traversal functions, \AgdaFunction{\_⋯ᴿ\_} for renaming and
\AgdaFunction{\_⋯ˢ\_} for substitution, are defined by structural induction on
the syntax. Each clause applies a renaming or substitution to a object language
term, lifting the renaming or substitution, when going under a binder. The
clauses in both definitions correspond directly to the traversal laws of the
$\sigma$-calculus, and therefore, we want these functions to reduce during type
checking. When adapting this framework to arbitrary custom syntaxes, the
traversal functions are the only definitions that must be provided by the user.
All other definitions can be stated independently of the concrete syntax.

Renaming and substitution primitives are declared opaque, which blocks their
reduction. This abstraction is essential for formulating the $\sigma$-calculus
laws, where the primitives must act as uninterpreted symbols rather than as
concrete functions. Blocking reduction in this way allows rewrite rules to
apply predictably. The types for renamings and substitutions are also declared
opaque to prevent misuse as actual functions. Their map-like behavior remains
accessible through the \AgdaFunction{\_\&\_} operator.

\subsection{Scoped and Multi-Sorted Sigma Calculus with First-Class Renamings and Substitutions}\label{sec:mis-sig}
\begin{figure}[t]
      \centering

      \begin{minipage}[t]{0.38\linewidth}
            \raggedright{}
            \EDefLaws{}
            \ERewriteSys{}
      \end{minipage}
      \hfill
      \begin{minipage}[t]{0.58\linewidth}
            \raggedright{}
            \EInteractLaws{}
            \EMonadLaws{}
      \end{minipage}

      % \vspace{1em}
      % \begin{minipage}[t]{0.58\linewidth}
      %       \raggedright{}
      %       \EInteractLaws{}
      % \end{minipage}
      % \hfill 
      % \begin{minipage}[t]{0.38\linewidth}
      %       \raggedright{}
      %       \ERewriteSys{}
      % \end{minipage}
      \caption{The Scoped and Multi-Sorted $\sigma$-calculus with First-Class Renamings and Substitutions.}
      \label{fig:mis-sig}
\end{figure}

We now adapt the rules from \cref{sec:pre-sig} to the setting with renamings.
One might restrict the relevance of renamings to the structural definition of
substitutions and state the laws only for substitutions. However, it is often
useful to reason about renamings directly, for instance in the context morphism
lemma~\cite{goguen1997candidates}. We therefore treat renamings as first-class
constructs. \cref{fig:mis-sig} presents the adapted $\sigma$-calculus laws.
They all satisfy the requirements from \cref{sec:pre-agd} to serve as valid
rewrite rules.

The definitional laws now involve primitives for both renamings and
substitutions. The interaction laws likewise cover all operators. Many of these
laws are similar in shape, e.g.\ we require eight associativity laws. Both
definitional and interaction laws can be proven independently of the concrete
syntax. In contrast, the monad laws must be shown by induction over the syntax
and are thus syntax-specific. We highlight this by marking these laws in red
throughout this work.

\citeauthor{10.1145/3293880.3294101} noticed that a further category emerges from the interaction between
renamings and substitutions. The coincidence laws govern the case where
substitutions are intertwinded with renamings, but the renamings can be
separated from the substitutions. The coincidence and the coincidence-fold laws
allow us to recover renamings from their embedding as substitutions via the
identity substitution. Extracting arbitrary renamings from deeply nested
substitutions would require a dedicated solving strategy, which we leave for
future work, but consider important for a more complete account.

The proofs for the laws can be established by unfolding the definitions of the
primitives for renamings and substitutions. In the implementation we make use
of function extensionality in order to prove the $\sigma$-calculus laws. It is
known that with a little more effort, this assumption is not necessary, as
shown for instance by~\citet{saffrich:LIPIcs.ITP.2024.32}.

\subsection{Discussion}\label{sec:mis-dis}
The laws presented in \cref{sec:mis-sig} are an extension of the classic
$\sigma$-calculus rules to a setting with renamings. They fulfill the
requirements for valid rewrite rules without further modification. This is a
consequence of marking the relevant symbols opaque, inside proofs of the laws
themselves, the definitions can still unfold when needed, while externally they
behave as uninterpreted symbols and are reduced only by the $\sigma$-calculus
laws.

When reasoning solely about renamings or substitutions, the laws suffice to
solve all equations automatically. Only when crossing the boundary between
renamings and substitutions, and thus relying on coincidence laws, additional
manual effort might sometimes become necessary.

The main drawback of this approach is the unavoidable duplication of similar
definitions and proofs. While associativity laws do not cause difficulties for
the user, even though there are many of them, since they can be stated
independently of the syntax, the situation is different for traversal, monad,
and coincidence laws. These must be provided explicitly by the user and consist
of two traversal definitions, two right-identity laws, and four
compositionality laws. Although each of them is similar in shape, taken
together they still contribute a noticeable amount of boilerplate. Even if we
would be able to derive the syntax-dependent definitions and laws, we still
would need to prove every later result about substitutions twice, first for the
special case of renamings and then for substitutions.

\section{The Agdasubst Library}\label{sec:ags}
This section introduces the \textsc{Agdasubst} library. It builds on the ideas
from \cref{sec:mis}, but removes the strict separation between variables and
terms by introducing a construct that can represent either, which we call Kits.
Using Kits, we can define renamings and substitutions uniformly through a
single notion, which we call Maps. With Kits and Maps in place, we can unify
the definitions and proofs from \cref{sec:mis}, which were previously similar
in shape.

The idea of Kits originates from~\citet{ren-sub}, where it was first formulated
in an intrinsically typed setting. It was later adapted by
\citet{saffrich:LIPIcs.ITP.2024.32} to a multi-sorted and scoped setting. We
frist recall the relevant definitions from the latter in
\cref{sec:ags-kit,sec:ags-com}, and then show how the $\sigma$-calculus laws
can be adapted to the setting of Kits and Maps. We omit most implementation
details of Kits and Maps, except for those directly relevant to the embedding
of the $\sigma$-calculus laws. For a full account of Kits, Maps, and their
extensions, we refer the reader to the work of
\citeauthor{saffrich:LIPIcs.ITP.2024.32}.

The library, and with it the whole of the next chapter, abstracts over a set of
sorts \ASortParam, a scoped and multi-sorted syntax \AScopedT, and a variable
constructor \AVarCstr.

\subsection{Kits}\label{sec:ags-kit}
A Kit is a structure that abstracts over terms and variables. The intended
usage is to instantiate this structure exactly twice, first for variables and
then for terms. Definitions can be written uniformly, parameterized over a Kit,
and are consequently applicable to both cases. \AKit{}To distinguish the only
two instances of Kits on the type level, Kits are indexed over a boolean-like
structure \AModeDef. The \AgdaFunction{image} function computes, given a
\AgdaDatatype{Mode}, whether the Kit corresponds to a variable or term:
\AImage{}The Kit record has multiple fields:
\begin{itemize}
      \item \AgdaField{K-id/‘} converts a variable into an element of $S$ \AgdaArgument{∋/⊢ᴷ} $s$.
            For the variable Kit, \AgdaArgument{\_∋/⊢ᴷ\_} is instantiated to \AgdaDatatype{\_∋\_}, so this operation acts as the identity.
            For the term Kit, it is instantiated to \AgdaField{\_⊢\_}, so this operation corresponds to the variable constructor.
      \item \AgdaField{K-‘/id} converts an element of $S$ \AgdaArgument{∋/⊢ᴷ} $s$ into a term $S$ \AgdaField{\_⊢\_} $s$, and is the dual to \AgdaField{K-id/‘}.
      \item \AgdaField{K-wk} shifts the de Bruijn indices in a variable or term.
            The newly introduced, unused variable \AgdaInductiveConstructor{zero} can assume any sort $s'$.
            For variables, \AgdaField{K-wk} corresponds to the successor constructor \AgdaInductiveConstructor{suc}.
            For terms, \AgdaField{K-wk} need to apply a weakening renaming to the term using traversal.
\end{itemize}
Because rewrite rules cannot mention record fields directly, we provide for each field a
corresponding version without the $K$-prefix. These are declared as opaque functions, so
they can be used as symbols in rewrite rules, while still unfolding to the actual fields
when needed. Going forwards, we assume all opaque defintions unfolding any time throughout this section.
It is only outside the library, when we have the $\sigma$-calculus laws in place, when we want
definitions to act as uninterpreted symbols. The Kit record further consists of axioms, that constrain the instantition, for
example, that \AAxiomEx. We will skip over these axioms from now on, since for
both variables and terms all of them hold almost entirely trivially.

We can now define the usual operations for renamings and substitutions. The
definitions are included directly in the \AgdaDatatype{Kit} record, so they are
implicitly parameterized over a kit. All the primitves are marked
\AgdaKeyword{opaque}.

\begin{minipage}[t]{0.48\linewidth}
      \raggedright{}
      \AMapA{}
\end{minipage}
\begin{minipage}[t]{0.48\linewidth}
      \raggedright{}
      \AMapB{}
\end{minipage}

\noindent The type for Maps $S_1$ \AgdaFunction{→ᴷ} $S_2$ unifies renamings $S_1$
\AgdaFunction{→ᴿ} $S_2$ and substitutions $S_1$ \AgdaFunction{→ˢ} $S_2$. For $M
      = $ \AgdaInductiveConstructor{Vᴹ} the \AgdaFunction{image} function computes to
the type of variables and thus the Map acts as renaming. Similarly, in the case
that $K = $ \AgdaInductiveConstructor{Tᴹ}, the Map corresponds to a
substitution. The identity map, lookup and extension operations behave
analogously in both the renaming and substitution cases. Since term traversal
is not yet available, composition cannot be defined at this stage. Instead, we
introduce a generalized weakening operator \AgdaFunction{\_;wk}, which can
later be rewritten as a composition with renamings. Similarly, once term
traversal is available, the \AgdaFunction{id/‘} and \AgdaFunction{‘/id}
operators can be expressed entirely in terms of the primitive operations.
We provide a shorthand for lifting a map:\ALifting{}and for referring to a
Kit explicitly in certain cases: \AKitExplicit{}To define the term Kit instances,
we need to know how a Map is applied to a term. The
user therefore has to provide a traversal function and proof that
applying the identity map leaves a term unchanged: \ATraversal{}Equipped with the traversal function we can define the two intended instances
for Kits:

\begin{minipage}[t]{0.48\linewidth}
      \raggedright{}
      \AInstanceRen{}
\end{minipage}
\begin{minipage}[t]{0.48\linewidth}
      \raggedright{}
      \AInstanceSub{}
\end{minipage}

\noindent In the \AgdaField{K-wk} field for the \AgdaFunction{T} Kit, we use the \AgdaFunction{wk} operation from the
\AgdaFunction{T} Kit. The kit instance for that is automatically resolved using
instance search. We can now turn our attention to the composition of maps.

\subsection{Composition}\label{sec:ags-com}
Inside the \AgdaDatatype{ComposeKit} record we can define a single, more
general composition of a $K_1$-Map with a $K_2$-Map, resulting in a $(K_1
      \sqcup K_2)$-Map: \AComposeKit{}The operation \AgdaField{C-\_\&/⋯\_} unifies
lookup and traversal. It takes a variable or term $x/t$ (according to $K_1$)
and a renaming or substitution $\varphi$ (according to $K_2$), and applies
$\varphi$ to $x/t$, yielding a variable or term (according to $K_1 \sqcup
      K_2$). From this operation, for which we again provide an opaque function
version without the $C$-prefix, we derive map composition: \ACompDef{}The map
combination operator is implicitly parameterized over a Compose Kit. It takes
an renaming or substitution $\varphi_1$ (according to $K_1$) and a renaming or
substitution $\varphi_2$ (according to $K_2$) and composes it to a renaming or
substitution (according to $K_1 \sqcup K_2$). When $x$ \AgdaFunction{\&}
$\varphi_1$ yields a variable, we need to lookup the resulting variable in
$\varphi_2$, or if we have a term, we traverse the term with $\varphi_2$. We
also define versions of both the unified lookup–traversal operator and the
composition operator where the Compose Kit is given explicitly: \ACompExp \ and
\ALoAExp.

As with Kits, we now require a user-provided proof about traversal. We need the
user to show that the chosen syntax and traversal function satisfy
compositionality. This proof can only be established by induction on the
syntax: \ACompositionality{}Finally, to cover the four possible instances of
Compose Kits, we provide two generic instances that suffice (axiom proofs
omitted):

\begin{minipage}{0.38\linewidth}
      \raggedright{}
      \AInstanceCRen{}
\end{minipage}
\begin{minipage}{0.58\linewidth}
      \raggedright{}
      \AInstanceCSub{}
\end{minipage}

\noindent \AgdaFunction{Cᴿ} is the Compose Kit for composition between a renaming and another $K$-Map.
\AgdaFunction{Cˢ} is the Compose Kit for composition between a substitution and another $K$-Map, and assumes that composition of a $K$-Maps with renamings is already available.

The third parameter $K_1 \sqcup K_2$ of a Compose Kit can be understood as a
functional dependency, uniquely determined by the choice of $K_1$ and $K_2$. It
evaluates to \AgdaFunction{V} if and only if both $K_1$ and $K_2$ are
\AgdaFunction{V}, otherwise it is \AgdaFunction{T}. In other words, only
renamings compose to a renaming, while every other case yields a substitution.
We can put this more algebraically: if $\varphi_1$ is a $K_1$-Map and
$\varphi_2$ is a $K_2$-Map, then the result is a $(K_1 \sqcup K_2)$-Map, where
$\sqcup$ refers to the join-semilattice on $\{V, T\}$ generated by $V < T$. One
might ask why we do not simply compute $K_1 \sqcup K_2$. This is in fact
possible and even necessary to define the $\sigma$-calculus rules in a way that
still adheres to the three restrictions on the shape of rewrite rules.

\subsection{Structural Kit Order}\label{sec:ags-sko}
We can easily define a least upper bound operation on Modes that mirrors the
behaviour of composing two maps with these Modes: \AModeLub{}By case analysis,
all the join-semilattice laws are verified easily: \AModeLubLaws{}All laws can
be embedded as reduction rules in Agda using \AgdaKeyword{REWRITE}, except for
\AgdaFunction{⨆-comm}, which is evidently non-terminating. We then lift this
operation from Modes to Kits themselves, using the least upper bound on Modes
in the type of the Kit: \AKitLub{}The same laws that hold for Modes, also hold
for Kits: \AKitLubLaws{}Note that the \AgdaFunction{⊔-comm} law requires the
transfer lemma in its type, since we could not rewrite \AgdaFunction{⨆-comm}
for Modes. All other laws are again added as rewrite rules. The Kit least upper
bound laws follow by case distinction on their Modes. When one inspects a proof
in detail, we get into trouble. As an example, consider \AgdaFunction{⊔-idem}.
The remaining laws follow in the same way: \AKitLubExcerpt{}If we split on the
Mode $M$ of the Kit $K$, we must prove in both cases that $K$ is either
\AgdaFunction{V} or \AgdaFunction{T}, according to the definition of
\AgdaFunction{\_⊔\_} and the chosen Mode. Here we run into a problem: $K$ is
inferred by instance resolution, and Agda cannot deduce that there is only one
possible choice, \AgdaFunction{V} for \AgdaInductiveConstructor{Vᴹ} and
\AgdaFunction{T} for \AgdaInductiveConstructor{Tᴹ}. Nor can the uniqueness of
instances be assumed in general, since external code might declare additional
instances. To resolve this, we restrict the Kit instances by requiring a field
\ALockField{} of type \ALock \ in the construction of a Kit. Because
\AgdaDatatype{Lock} is private, it can only be constructed from within the
library. We can now postulate the uniqueness of Kits after manually verifying
that only two instances exist and that we rely solely on Kits provided by
instance resolution, never creating them locally: \AUniqueKits{}There also
exists a version \AgdaFunction{unique–K} that computes whether a given $K : $
\AgdaDatatype{Kit} $M$ is \AgdaFunction{V} or \AgdaFunction{T}, based on its
Mode.

\subsection{Computing Compose Kits}\label{sec:ags-cck}
Our original goal was to compute the third parameter of a Compose Kit. To
motivate this and, more generally, the computation of Compose Kits from Kits,
we consider a first attempt of the defintion for the associativity law of the
$\sigma$-calculus, which in our Kit-based setting corresponds to the
associativity of map composition:\AAssocTryO{}Here $C_1$ to $C_6$ denote the
four Compose Kits that are required for the compositions. For clarity we apply
them to the composition operators explicitly, although instance resolution
would perform this step automatically. This yields the most general formulation
of the associativity law for Kits, where arbitrary Kits are admitted as long as
both sides reduce to a $K_5$-Map. This law, however, is not a valid rewrite
rule. Both $C_3$ and $C_4$, required for the right side, appear only there,
violating the condition that all arguments of a rewrite rule must occur on the
left. Moreover, the law requires the instance argument $K_6$, used as the
result of composing the $K_2$-Map with that of $K_4$, which is only required on
the right. Since $K_6$ can be computed via the least upper bound operation
defined earlier, the main difficulty is to compute $C_3$ and $C_4$.

One may attempt to define a function \ACompCKitSafe{} for this purpose. This
function can be readily defined, but its correctness relies not only on the
uniqueness of Kit instances, but also on the uniqueness of Compose Kit
instances:\AUniqueCKits{} As before we reqiure a value of \AgdaDatatype{Lock}
in the creation of Compose Kits. By manual inspection we confirm that
\AgdaFunction{Cᴿ} and \AgdaFunction{Cˢ} are the only instances and that none
are created locally.

Yet the function $\AgdaFunction{\_;ᴷ\_}$ is insufficiently general. Computing
$C_4$ in the associativity law yields $C_4 = K_2 \; \AgdaFunction{;ᴷ} \; K_4 =
      \AgdaDatatype{ComposeKit}\; K_2\; K_4\; (K_2$ \AgdaFunction{⊔} $ K_4)$.
Consequently, $C_3$ evaluates to $ C_3 = K_1 \; \AgdaFunction{;ᴷ} \; (K_2 \;
      \AgdaFunction{⊔} \; K_4) = \AgdaDatatype{ComposeKit}\; K_1\; (K_2$
\AgdaFunction{⊔} $ K_4)\; (K_1$ \AgdaFunction{⊔} $ (K_2 $ \AgdaFunction{⊔} $
      K_4))$. The final Map on right side is then a $K_1 $ \AgdaFunction{⊔} $ (K_2$
\AgdaFunction{⊔} $ K_4)$-Map, whereas the left side is a $K_5$-Map. Adjusting
the $K_5$ parameter in $C_1$ or $C_2$ to the result Kit of $C_4$ is not
possible, since Compose Kits whose third parameter involves least upper bounds
cannot be inferred by instance resolution\footnote{Adding instances that would
      allow the inference of Compose Kits with least upper bounds did not work out
      for us. In addition, Agda generally refused to infer such instances during
      higher-order pattern unification when checking if a rewrite rule applies.}. We
require an even more general, though unsafe, function: \ACKitUnsafe{}Since it
is declared non-covering, evaluation may get stuck, yielding a value for a
Compose Kit that should not exist. Thus, every use of this function requires
manual verification. Computed Compose Kits are reduced to the actual instances,
when applicable, using the rewrite rules \ACKitRenRed{} and \ACKitSubRed{}.
This allows us to state a version of the associativity law eligible for
rewriting: \AAssocTryT{}We can now connect $K_1 \sqcup (K_2 \sqcup K_4)$ with
$K_5$ in the outer Compose Kit on the right side. It remains to check that
whenever the incoming Compose Kits $C_1$ and $C_2$ are valid, the outgoing
ones, computed using our unsafe function, are as well. As an illustrative case,
if $K_1 = .. = K_5 = \AgdaFunction{V}$, then all Maps are renamings and the
outgoing Compose Kits correspond to compositions of renamings. By exhaustively
checking all $8$ valid combinations of values for $C_1$ and $C_2$, we confirm
that in each case the outgoing Compose Kits are valid. Inside the associativity
proof, the remaining cases must be refuted under the assumption of unique
incoming Compose Kits and the non-existance of invalid Compose Kits:
\AUnqiueCKitsImp{}As before, each use of this postulate requires manual
verification to avoid inconsistencies.

All other $\sigma$-calculus laws, generalized to Maps, follow the same
structure as the \AgdaFunction{associativity} law. On the left side of the
equations we rely on Kits and Compose Kits inferred via instance resolution,
while on the right side we compute the resulting Kits and Compose Kits using
the least upper bound operation together with the unsafe
\AgdaFunction{\_,\_,\_} operator, where we manually verify that its use is
sound. The proofs are carried out by case analysis on the Modes of the Kit
instance arguments and rely on the uniqueness of (Compose-) Kits as well as the
refutation of invalid ComposeKit Kits. Apart from these aspects, their
structure closely resembles the familiar proofs for separate renamings and
substitutions.

\subsection{The Scoped and Multi-Sorted $\sigma$-Calculus with Kits and Maps}\label{sec:ags-rls}
\begin{figure}[tp]
      \centering
      \small
      {\raggedright \textbf{\Large Symbols} \par}
      \vspace{1.5em}

      \begin{minipage}[t]{0.48\textwidth}
            \raggedright{}
            \textbf{Sorts} $s$ \\
            \ESortTy{}\\
            \vspace{0.5em}
            \emph{.. object language sorts ..}\\
            \vspace{0.5em}
            \textbf{Scopes} $S$ \\
            \EScopeDefTy{}\\
            \vspace{0.5em}
            \EScopeDef{}\\
            \vspace{0.5em}
            \textbf{Variables} $x$ \\
            \EVarsTy{}\\
            \vspace{0.5em}
            \EVars{}\\
            \vspace{0.5em}
            \textbf{Terms} $t$ \\
            \ETmC{}\\
            \vspace{0.5em}
            \EVarC{}\\
            \emph{.. object language  ..}\\
            \vspace{1.0em}
      \end{minipage}
      \hfill
      \begin{minipage}[t]{0.48\textwidth}
            \raggedright{}
            \textbf{Kits} $K$ \\
            \EKitDefTy{}\\
            \vspace{0.5em}
            \EKitDef{}\\
            \vspace{0.5em}
            \textbf{Maps} $ϕ$\\
            \EPrimsTy{}\\
            \vspace{0.5em}
            \EPrims{}\\
            \vspace{0.5em}
            \textbf{Variables or Terms} $x/t$\\
            \EVarTrmTy{}\\
            \vspace{0.5em}
            \EVarTrmApp{}
      \end{minipage}

      {\raggedright \vspace{2.0em} \textbf{\Large Reduction} \par}
      \vspace{1.5em}
      \begin{minipage}[t]{0.40\textwidth}
            \raggedright
            \textbf{Kit Laws}\\
            \ETypeLevel{}\\
            \vspace{0.5em}
            \textbf{Definitional Laws}\\
            \vspace{0.5em}
            \EDefLawTy{}\\
            \vspace{0.5em}
            \EDefLaw{}\\
            \vspace{0.5em}
            \textbf{Definitional Composition Specialization Laws}
            \ESpecialDefLaws{}
      \end{minipage}
      \hfill
      \begin{minipage}[t]{0.54\textwidth}
            \raggedright
            \textbf{Interaction Laws}\\
            \EInteraction{}\\
            \vspace{0.5em}
            \textbf{Traversal Laws}\\
            \ETravL{}\\
            \emph{.. object language traversal ..}\\
            \vspace{0.5em}
            \textbf{Monad Laws}\\
            \EMonad{}
            \vspace{0.5em}
            \textbf{Coincidence Laws}\\
            \ECoincidence{}
      \end{minipage}
      \caption{The Scoped and Multi-Sorted $\sigma$-Calculus with Kits and Maps.}
      \label{fig:ags-fin}
\end{figure}

We are now ready to state the $\sigma$-calculus laws in the Kit-setting.
\cref{fig:ags-fin} lists the symbols and laws as they are used in
\textsc{Agdasubst}. These are not the formulations from the library itself. For
readability, we omit the Kit and ComposeKit instance arguments. Kits and
ComposeKits carry data in the library, and Kits must be disambiguated on the
type level by Modes. Here we simply assume that Kits act as markers, which
allows us to elide the mode lattice laws. Similarly, Compose Kits serve as
markers for admissible compositions and are given explicitly for every operator
in square brackets.

The symbols are grouped into several categories. Sorts and Scopes provide
intrinsic scoping at the type level. Variables are supplied by the library,
while terms are user-defined, subject only to the requirement of a variable
constructor in the appropriate form. We define the type for Kits and the
variable and term Kit, together with the least upper bound operation. We then
define the type of Maps and the four map primitives, parameterized by Kits. We
specialize \AgdaFunction{wk} directly to a renaming in order to avoid
duplication, since it can always be embedded as a substitution by composition
with the identity substitution. When the Kit in the type of Maps is known to be
\AgdaFunction{V} or \AgdaFunction{T}, we use the variables $\rho$ and $\sigma$,
respectively. From Kits we further obtain the combined type of
variables-or-terms, together with the unified lookup/traversal operator. While
there are separate type definitions for variables, terms, and
variables-or-terms, there is only one type definition for Maps, specialized via
the Kit index. There is no technical reason not to unify variables and terms in
the same way. However, unlike renamings, substitutions, and maps, terms are
user-supplied, whereas variables are library-defined and thus qualify for a
distinct type definition.

We already discussed the \textbf{Kit Laws} in \cref{sec:ags-sko}, so we now
turn to the remaining ones. We begin with the type-level reduction laws
\AgdaFunction{imgⱽ} and \AgdaFunction{imgᵀ} inside the \textbf{Definitional
      Laws}, which rewrite the type of variables-or-terms to either the type of
variables or the type of terms when the Kit index is known to be
\AgdaFunction{V} or \AgdaFunction{T}, respectively.

Next comes the definition of lookup on the identity substitution
\AgdaFunction{idˢ}. It is given in terms of the unified lookup/traversal
operator. In fact, we never use the specialized lookup or traversal operators
at all. We considered keeping the specialized operators, but this would require
duplicating many laws, once for the specialized operators and once for the
general one\footnote{There are certainly situations where such duplication
      could help, but in our case studies, with the addition of the definitional
      composition specialization laws, we had no need for laws specialized to lookup
      or traversal.}. Identity renaming does not require its own law, since it is
already covered by the now generalized \AgdaField{right-id} from the
\textbf{Monad Laws}, which handles both a term traversed by an identity
substitution and a variable looked up in the identity renaming.

The laws \AgdaFunction{wk}, \AgdaFunction{ext₀}, and \AgdaFunction{extₛ} are
the direct generalizations of their familiar counterparts for renaming and
substitution. The law \AgdaFunction{comp}, however, is more subtle. As shown in
\cref{fig:ags-fin}, it only reduces when both $\varphi_1$ and $\varphi_2$ are
renamings. This restriction is necessary because of a fundamental duality
between the definition of composition (via the unified lookup/traversal
operator) and the compositionality law in full generality:

\begin{minipage}{0.45\linewidth}
      \raggedright{}
      \ECompGeneral{}
\end{minipage}
\begin{minipage}{0.55\linewidth}
      \raggedright{}
      \ECompoGeneral{}
\end{minipage}

\noindent Viewed side by side, these laws are exact duals. If both were admitted into the
rewrite system, they would trigger infinite loops. This is no accident,
intuitively, one law pushes variables into maps, while the other pushes maps
into other maps, away from terms.

To avoid nontermination, we adopt the following heuristic:
\begin{itemize}
      \item In variable position ($K_5 = $ \AgdaFunction{V}) use the definition of
            composition. In this case all other indices $K_1,\dots,K_4$ are forced to
            \AgdaFunction{V}, and $x/t$ must be a variable. This yields exactly the
            specialized \AgdaFunction{comp} law shown in the figure.
      \item In term position ($K_5 = $ \AgdaFunction{T}) use \AgdaField{compositionality}.
            Here no restrictions on the other indices arise, so the law is preserved in
            most of its generality and appears under the \textbf{Monad Laws}.
\end{itemize}

\noindent If neither condition applies (i.e.\ we do not know whether we are dealing with
a variable or a term and $K$ is a variable), the user must explicitly choose from the fully general laws which law to apply,
either by equational reasoning or by using \AgdaKeyword{rewrite}. With this heuristic the \AgdaFunction{comp} law is rather restricted. We now
miss cases where we would make progress by applying the \AgdaFunction{comp} law, when $\varphi_1$
and $\varphi_2$ are not both renamings. Hence, we consider all situations where
progress would be made directly after applying \AgdaFunction{comp}, by
specializing on $\varphi_1$ or $\varphi_2$. These appear under the name
\textbf{Definitional Composition Specialization Laws}.

The \textbf{Interaction Laws} are the direct generalizations of their original
$\sigma$-calculus counterparts to Kits. Next come the \textbf{Traversal Laws}.
Since we are independent of the object syntax except for the required variable
constructor, we only show the law specific to the traversal of a variable. When
a variable constructor is applied to a map, the variable it holds is mapped
accordingly. To turn the result back into a term, as it may otherwise be either
a variable or a term, depending on the map, we subsequently apply the identity
substitution.

To illustrate another case for traversal, consider an object language that
supports $\lambda$-abstractions. The corresponding user-supplied traversal
function, needed to construct the term Kit, would then include a clause
\FSLambdaExT. However, this clause does not directly corresponds to a traversal
law. It must first be lifted from the traversal operator to the unified
lookup/traversal operator: \FSLambdaExL. This equality can be proven by
reflexivity, when we unfold the definition of the unfied lookup/traversal
operator. It can then be added as a rewrite rule.

Finally, the \textbf{Coincidence Laws} generalize from renamings to maps and
allow maps to be pulled out of a substitution when feasible. These laws are not
a complete account of all possible ways to pull maps out of substitutions. The
search for more sophisticated laws that cover additional cases where this can
be done would certainly be worthwhile.

\subsection{Discussion}
The generalization of the $\sigma$-calculus laws required unsafe computation
and a careful formulation of the laws so that they remain valid rewrite rules
while still allowing the inference of Kits and ComposeKits to be offloaded to
the instance resolution algorithm. The key difficulty in adapting the laws to
the Kit-setting stems from the duality between the general versions of the
\AgdaFunction{comp} and the \AgdaField{compositionality} laws. Our solution is
a heuristic that resolves the conflict by restricting the laws to cases where
it is known whether we are in variable or term position. In particular, when we
dont know, the direction in which the laws should flow becomes ambiguous. In
such situations, the user must resort to manual application of the laws. In
contrast to the laws from \cref{sec:mis-sig}, the Kit-based laws allow us to
establish properties directly for Maps, without the need to first carry out
separate but structurally equal proofs for renamings and substitutions.

\section{Case Studies}\label{sec:ext}
We will discuss two examples demonstrating the usage of the \textsc{Agdasubst}
library. The first concerns proofs of subject reduction for System F with
Subtyping. We begin by presenting the proof using the laws from
\cref{sec:mis-sig}, and then revisit it using the Kit-based approach from
\cref{sec:ags-rls}. Afterwards, we extend the \textsc{Agdasubst} version with
subtyping. The second example, based on CITE, provides an instantiation for a
generic universe of scoped syntax.

\subsection{System F (+ Subtyping)}\label{sec:ext-syf}

\begin{figure}[!t]
      \small
      \centering
      \begin{minipage}[t]{0.48\linewidth}
            \small
            \raggedright{}
            \ETyping{}
      \end{minipage}
      \hfill
      \begin{minipage}[t]{0.48\linewidth}
            \small
            \raggedright{}
            \ESemantics{}
      \end{minipage}
      \caption{Typing and semantics for System F.}
      \label{fig:ext-tas}
\end{figure}

Recall the multi-sorted and scoped syntax from \cref{fig:pre-svm}. We briefly
outline a proof of subject reduction, first using the laws from
\cref{sec:mis-sig}, and then using the \textsc{Agdasubst} library. The type
system and semantics for both cases are essentially the same, and we present
them once in \cref{fig:ext-tas}. In both settings, we make use of functions
\EWk{} and \ESubst{}. They are straightforward to define from the primitves in
both cases. Within the typing relation, we rely on the lookup relation
\ELookup{} for retrieving the type of a variable from the typing context. The
type of the lookup and typing relation uses the operator \AgdaFunction{\_∶⊢\_},
defined as:
\begin{minipage}[t]{0.48\linewidth}
      \raggedright{}
      \EUpArrow{}
\end{minipage}
\begin{minipage}[t]{0.48\linewidth}
      \raggedright{}
      \ETypeOf{}
\end{minipage}

\subsection*{Separate Renamings and Substitutions}
We have already seen the instantiation of the $\sigma$-calculus for System F in
\cref{sec:mis-sig}, where renamings and substitutions are treated separately.
To establish subject reduction, we first prove that renaming preserves typing,
and then, using this result in the proof for the typed lifiting of a
substitution, show that general substitution preserves typing as well.

\begin{minipage}[t]{0.48\linewidth}
      \raggedright{}
      \ERPT{}
\end{minipage}
\begin{minipage}[t]{0.48\linewidth}
      \raggedright{}
      \ESPT{}
\end{minipage}

Both proofs rely on the types for well-typed renamings \EWTR and well-typed
substitutions \EWTS, whose definitions we omit here. Within the respective
cases, we look up the well-typed variable in the typed renaming/substitution,
and when going under a binder we use the typed lifting of renamings and
substitutions. Thanks to the automatic application of the $\sigma$-calculus
laws, both proofs follow directly by induction, even in the cases where we go
under a binder. Using the two results, we can then prove subject reduction for
System~F: \ESR{}All cases go through without requiring any explicit law about
substitutions.

\subsection*{Using \textsc{Agdasubst}}
We first instantiate the library by providing the traversal function and the syntax dependent laws.
The traversal function for maps in System F is given by:\SFTraversal{}In the case of variables, we need to turn the variable into either a variable or a term. We can now proceed to prove the monad laws:

\begin{minipage}[t]{0.48\linewidth}
      \small
      \raggedright{}
      \SFId{}
\end{minipage}
\begin{minipage}[t]{0.48\linewidth}
      \small
      \raggedright{}
      \SFCompo{}
\end{minipage}

\noindent For right identity, in the case where we go under a binder, we need to show that
\AIdLift{}, and for compositionality we require \ADistLift{}. Both of these
laws \AgdaFunction{id↑≡id} and \AgdaFunction{dist–↑–;} are provided by the library and can already be rewritten.
Both follow from the rules of the
$\sigma$-calculus and thus do not cause any issues when rewritten. With these
two laws applied, all cases in the proofs of both lemmas, including those where
we go under a binder, follow directly by structural induction without any need
for manual application of lemmas.

% TODO: coincidence lemmas and lifting to general operator
In the end we can re-define lookup and traversal in terms of the unified
lookup/traversal operator:

\begin{minipage}[t]{0.48\linewidth}
      \small
      \raggedright{}
      \SFLookupSH{}
\end{minipage}
\begin{minipage}[t]{0.48\linewidth}
      \small
      \raggedright{}
      \SFTraversalSH{}
\end{minipage}

\noindent To prove that renaming and substitution preserve typing in a single lemma, we
make use of Typing Kits, following the approach of \citet{saffrich:LIPIcs.ITP.2024.32}. Typing
Kits abstract over typed renamings and substitutions depending on a Kit, in a
way analogous to how Maps abstract over renamings and substitutions based on
a Kit. Within the library, they are instantiated exactly twice, once for the
variables Kit (i.e.\ typed renamings) and once for term Kit (i.e.\ typed substitutions). For a
more detailed description, we refer the reader to the already mentioned work.
The unified proof that maps preserve typing mirrors the shape of the separate
proofs: \SFSPT{}\textsc{Agdasubst} automatically solves all equations occuring
from lifting. In the variable case, we additionally need to convert a typed
variable or term into a typed term. The proof of subject reduction is exactly the
same as the one presented before. The only difference is that, instead of using
the substitution preserves typing lemma, we use the more general lemma that
maps preserve typing in the cases of $\beta$-reductions: \SFSR{}

\subsection*{Subtyping}
When we extend System F with subtyping\cite{CARDELLI19944}, substitution does
not generally preserve typing because type variables must respect subtyping
bounds. To address this, we can encode a type variable binding $\alpha < t$ as
two bindings, $\alpha : \star$ and a constraint $c : (\alpha < t)$. This
restricted, first-class-constraint–style encoding is isomorphic to the original
formalization and makes substitution-preserves-typing hold again. With this
formulation, violating a bound simply yields a context with an unsatisfiable
constraint~\cite{saffrich:LIPIcs.ITP.2024.32}.

The extension with subtyping substantially complicates the proof of subject
reduction. In this setting we must reason about an injective subset of
renamings as well as about various inversion lemmas that arise due to the
additional subtyping constraints. We also require the fact that renaming
preserves reduction. All additional lemmas together with the subject reduction
theorem itself can be established within \textsc{Agdasubst} without any need
for manual application of substitution lemmas and can be found in the
supplement.

\subsection{Generic Syntax}\label{sec:ext-uni}
The class of object languages supported by \textsc{Agdasubst} can be described
the definition of a generic syntax à la
\citet{allais2021typescopesafeuniverse}, adapted to the intrinsically scoped
setting~\cite{saffrich:LIPIcs.ITP.2024.32}:

\begin{minipage}[t]{0.48\linewidth}
      \raggedright{}
      \GDesc{}
\end{minipage}
\begin{minipage}[t]{0.48\linewidth}
      \raggedright{}
      \GDenot{}
      \GTms{}
\end{minipage}

\noindent This definition includes constructors for sums \AgdaInductiveConstructor{`σ} and for object language
constructors that introduce $0$ to $n$ binders \AgdaInductiveConstructor{`X}.\footnote{The
      \AgdaInductiveConstructor{`X} constructor allows the addition of $S'$ many
      binders. In its current state, \textsc{Agdasubst} does not support variadic
      binders, the rewrite rules only apply when $S'$ is fixed.}
All required laws for instantiating \textsc{Agdasubst} can be proven once and for
all. Users of the library can define their own syntax directly from the
building blocks of the generic syntax, without implementing traversals or reproving syntax-specific
laws. For example, the sigma-calculus laws apply automatically to the following
encoding of the lambda calculus:

\begin{minipage}[t]{0.48\linewidth}
      \raggedright{}
      \GDescL{}
\end{minipage}
\begin{minipage}[t]{0.48\linewidth}
      \raggedright{}
      \GPattern{}
\end{minipage}

Here, the generic encoding has a single sort of expressions, \GSort{}, and two
labels, one for abstraction and one for application, \GLabel{}. While the
generic approach is very convenient, the main drawback is that Agda sometimes
unfolds the pattern definitions inside holes, leaking details of the encoding,
which results in proof states that are hard to read.

\section{Related Work}\label{sec:rel}
Most of this work is inspired by and builds upon the work theoretical work
conducted in the development of \textsc{Autosubst}~\cite{schafer2015autosubst}.
\textsc{Autosubst}, like \textsc{Agdasubst}, uses the laws of the
$\sigma$-calculus to automatically solve equations between terms with
substitutions in the Rocq proof assistant. With
version~2~\cite{10.1145/3293880.3294101}, \textsc{Autosubst} also introduced
scoped syntax and first-class renamings, making their law set comparable to the
one presented in \cref{sec:mis-sig}. Instead of using multi-sorted parallel
substitutions, \textsc{Autosubst} uses vector substitutions, that bundle
substitutions over multiple variable sorts. Since Rocq is more tactic-oriented,
\textsc{Autosubst} provides a tactic to apply these laws, whereas in Agda we
opted to embed the laws natively via the rewrite mechanism to achieve
definitional equalities. Additionally, \textsc{Autosubst} supports code
generation of syntax-dependent functions and laws from a Higher-Order Abstract
Syntax (HOAS) definition, while we currently have no such counterpart. If we
add code generation in the future we prefer a solution that uses reflection, to
generate correct code within Agda instead of relying on external code
generation. Unlike our approach, \textsc{Autosubst} does not provide an
abstraction similar to Kits for unifying proofs about renamings and
substitutions.

The idea of Kits was originally proposed by \citeauthor{ren-sub} in an
intrinsically typed setting, which is also reflected new work, for instance in
\cite{altenkirch2025substitution}. The latter also experiments with a uniform
treatment of renamings and substitutions in an intrinsically typed setting.
\citet{saffrich:LIPIcs.ITP.2024.32} developed a framework based on the Kit
concept for scoped syntax, but without providing any strategy for automatically
solving equations between terms with substitutions. Their focus was instead on
abstracting over typing relations and the semantics of the object language,
while providing a uniform treatment of renamings and substitutions in that
context. These results apply directly to \textsc{Agdasubst}, as demonstrated in
\cref{sec:ext-syf}, where we use the typing Kits from their work. Our laws
integrate seamlessly with these extension Kits and substantially simplify the
proofs required for their instantiation.

There already exists another Agda library with a similar goal to
\textsc{Agdasubst}, called \textsc{Abstract Binding Trees}, which is, for
example, used by \citet{siek2021parameterized}. It allows one to construct an
object language from a generic syntax, similar to the one in
\cref{sec:ext-uni}, but in contrast it supports only extrinsic scoping, not
intrinsic. Rather than focusing on specific relations, for instance the typing
relation, as in~\cite{saffrich:LIPIcs.ITP.2024.32}, it abstracts over arbitrary
predicates about the language. The library also offers experimental support for
rewriting, but the current state of this feature is unknown to us. In any case,
rewriting is only supported in the style of treating renamings and
substitutions separately, as in \cref{sec:mis-sig}, rather than in a uniform
threatement.

More distantly related is \textsc{Nominal 2}~\cite{Nominal2-AFP}, an Isabelle
package that addresses the $\alpha$-equivalence problem not via de Bruijn
indice, but using nominal datatypes and freshness conditions. This approach is
closer to the intuitive notion of variables, but the tool does not provide any
strategy for solving equations involving terms with substitutions.

Additionally, there are tools such as \textsc{Needle and
      Knot}~\cite{10.1007/978-3-662-49498-1_17}, which is a code generator for
unscoped syntax using de Bruijn indices. This tool generate substitution and
interaction lemmas for single-pointed substitutions in languages that feature
multiple variable sorts and binders capable of binding lists of variables.

\section{Further Work}\label{sec:wrk}
This section outlines possible directions for improving \textsc{Agdasubst} in
the future.

\subsection*{Reflection}
In comparison to other libraries discussed in \cref{sec:rel}, users of
\textsc{Agdasubst} still need to manually provide the definitions for
traversal, the syntax-dependent proofs (e.g. \AgdaField{right-id} and
\AgdaField{compositionality}). \citeauthor{saffrich:LIPIcs.ITP.2024.32} has
shown that these definitions can be derived automatically using Agda's
reflection mechanism. There is no reason why the same approach could not be
applied for \textsc{Agdasubst}.

\subsection*{Sigma Calculus Extensions}
\citeauthor{Stark:2020:Mechanising} introduced several interesting extensions to the $\sigma$-calculus that could
also be added to \textsc{Agdasubst}. One example is support for variadic binders,
that is, binders of a variable or unknown number of variables, which are for instance useful for languages with
pattern matching. Another possible extension is support for lists of terms, allowing a variable or
unknown number of terms in the syntax type. This would, for instance, enable support for
languages with $n$-ary application.
Both of these features can be encoded in the current system by inlining the list data structure constructors
directly into the syntax definition, treating them as terms of a new sort.

\subsection*{Meta Theory}
In its current state, \textsc{Agdasubst} completely lacks a meta-theoretical
analysis. While we conjecture it to be terminating and confluent, it would be
interesting to formally establish these properties.

\subsection*{Completeness}
As mentioned, \textsc{Agdasubst} is not complete once we move beyond reasoning
about only renamings or substitutions. As soon as we cross the boundary between
these two worlds, by considering more general maps or Kits, we definitely
cannot expect \textsc{Agdasubt} to solve all possible equations. It would
therefore be worthwhile to investigate additional laws, similar in spirit to
the coincidence laws, to increase the space of solvable equations.

\subsection*{Performance}
At the moment, the performance of \textsc{Agdasubst} is, to be honest, very
poor. For instance, the proof of subject reduction for System~F with subtyping
requires several minutes to check. Our analysis shows that most of this time is
spent on instance resolution and function coverage analysis (i.e., verifying
that pattern matching clauses are exhaustive). Currently, we have no concrete
strategy to improve these aspects, without decreasing user-experience, except
that we know, that the addition of the coincidence laws dramatically increases
type-checking time.

\section{Conclusion}\label{sec:con}
We have shown how to take advantage of the laws of the $\sigma$-calculus to
automatically solve equations inside proofs about syntax with binders, in Agda. 
We first embedded the $\sigma$-calculus with
first-class renamings directly into Agda as reduction rules, using its rewrite
mechanism to yield definitional equalities.

We then wanted to unify proofs about renamings and substitutions, which led to
the current state of the \textsc{Agdasubst} library. To achieve our goal, we
introduced the notions of Kits and Maps from prior work and adapted the
$\sigma$-calculus laws to this setting. We demonstrated how \textsc{Agdasubst}
can be applied to establish properties of formal languages, such as subject
reduction for System F with subtyping. The \textsc{Agdasubst} library is still
far away from a stable release candidate. We have outlined some directions for
future development and plan to improve the library in the time to come.

\bibliographystyle{ACM-Reference-Format}
\bibliography{references}

\end{document}
