\documentclass[screen,nonacm]{acmart}
\usepackage[capitalize,noabbrev]{cleveref}
\usepackage{agda}

\input{agda-generated}
\input{agdamacros}
\usepackage{dsfont}
\usepackage{newunicodechar}
\newunicodechar{λ}{\ensuremath{\mathnormal\lambda}}
\newunicodechar{σ}{\ensuremath{\mathnormal\sigma}}
\newunicodechar{τ}{\ensuremath{\mathnormal\tau}}
\newunicodechar{π}{\ensuremath{\mathnormal\pi}}
\newunicodechar{ℕ}{\ensuremath{\mathbb{N}}}
\newunicodechar{∷}{\ensuremath{::}}
\newunicodechar{≡}{\ensuremath{\equiv}}
\newunicodechar{≅}{\ensuremath{\cong}}
\newunicodechar{∀}{\ensuremath{\forall}}
\newunicodechar{ᴸ}{\ensuremath{^L}}
\newunicodechar{ᴿ}{\ensuremath{^R}}
\newunicodechar{ʳ}{\ensuremath{^r}}
\newunicodechar{ᶻ}{\ensuremath{^Z}}
\newunicodechar{ₗ}{\ensuremath{_l}}
\newunicodechar{ⱽ}{\ensuremath{^V}}
\newunicodechar{⟧}{\ensuremath{\rrbracket}}
\newunicodechar{⟦}{\ensuremath{\llbracket}}
\newunicodechar{⊤}{\ensuremath{\top}}
\newunicodechar{⊥}{\ensuremath{\bot}}
\newunicodechar{₁}{\ensuremath{_1}}
\newunicodechar{₂}{\ensuremath{_2}}
\newunicodechar{₃}{\ensuremath{_3}}
\newunicodechar{₄}{\ensuremath{_4}}
\newunicodechar{₅}{\ensuremath{_5}}
\newunicodechar{₆}{\ensuremath{_6}}
\newunicodechar{₇}{\ensuremath{_7}}
\newunicodechar{₈}{\ensuremath{_8}}
\newunicodechar{₉}{\ensuremath{_9}}
\newunicodechar{∈}{\ensuremath{\in}}
\newunicodechar{₀}{\ensuremath{_0}}
\newunicodechar{′}{\ensuremath{'}}
\newunicodechar{ˢ}{\ensuremath{^S}}
\newunicodechar{ᴬ}{\ensuremath{^A}}
\newunicodechar{∘}{\ensuremath{\circ}}
\newunicodechar{𝟙}{\ensuremath{\mathds{1}}}  
\newunicodechar{𝟘}{\ensuremath{\mathds{O}}}
% \newunicodechar{𝟙}{\ensuremath{\mathbb{I}}}  
% \newunicodechar{𝟘}{\ensuremath{\mathbb{O}}}
\newunicodechar{ᴾ}{\ensuremath{^P}}
\newunicodechar{ᵀ}{\ensuremath{^T}}
\newunicodechar{⊎}{\ensuremath{\uplus}}
\newunicodechar{ι}{\ensuremath{\iota}}
\newunicodechar{⇐}{\ensuremath{\Leftarrow}}
\newunicodechar{⇒}{\ensuremath{\Rightarrow}}
\newunicodechar{➙}{\ensuremath{\to^P}}
\newunicodechar{Δ}{\ensuremath{\Delta}}
\newunicodechar{∅}{\ensuremath{\emptyset}}
\newunicodechar{⁺}{\ensuremath{^+}}
\newunicodechar{𝕏}{\ensuremath{\mathbb{X}}}
%\newunicodechar{·}{\ensuremath{\cdot}} %seems to be defined already!
\newunicodechar{∙}{\ensuremath{\cdot}}
\newunicodechar{⁇}{\ensuremath{?}}
\newunicodechar{‼}{\ensuremath{!}}
\newunicodechar{⊕}{\ensuremath{\oplus}}
\newunicodechar{ℤ}{\ensuremath{\mathbb{Z}}}
\newunicodechar{μ}{\ensuremath{\mu}}
\newunicodechar{∃}{\ensuremath{\exists}}
\newunicodechar{;}{\ensuremath{\fatsemi}}
\newunicodechar{Σ}{\ensuremath{\Sigma}}
\newunicodechar{ᴿ}{\ensuremath{^R}}
\newunicodechar{ᵢ}{\ensuremath{_i}}
\newunicodechar{≢}{\ensuremath{\nequiv}}
\newunicodechar{≟}{\ensuremath{\stackrel{{\tiny?}}{=}}}
\newunicodechar{≤}{\ensuremath{\le}}
\newunicodechar{ᵇ}{\ensuremath{^b}}
\newunicodechar{𝓣}{\ensuremath{\mathcal{T}}}
\newunicodechar{𝓔}{\ensuremath{\mathcal{E}}}
\newunicodechar{Γ}{\ensuremath{\Gamma}}
\newunicodechar{γ}{\ensuremath{\gamma}}
\newunicodechar{⊔}{\ensuremath{\sqcup}}
\newunicodechar{α}{\ensuremath{\alpha}}
\newunicodechar{η}{\ensuremath{\eta}}
\newunicodechar{ω}{\ensuremath{\omega}}
\newunicodechar{◁}{\ensuremath{\lhd}}
%PT: seems to be already defined
%\newcommand{\lambdabar}{{\mkern0.75mu\mathchar '26\mkern -9.75mu\lambda}}
\newunicodechar{ƛ}{\ensuremath{\lambdabar}}
\newunicodechar{Λ}{\ensuremath{\mathnormal\Lambda}}
\newunicodechar{ρ}{\ensuremath{\rho}}
\newunicodechar{𝓖}{\ensuremath{\mathcal{G}}}
\newunicodechar{ℓ}{\ensuremath{\ell}}
\newunicodechar{♯}{\ensuremath{\sharp}}
\newunicodechar{⇓}{\ensuremath{\Downarrow}}
\newunicodechar{𝓥}{\ensuremath{\mathcal{V}}}
\newunicodechar{∧}{\ensuremath{\wedge}}
\newunicodechar{ₛ}{\ensuremath{_s}}
\newunicodechar{χ}{\ensuremath{\chi}}
\newunicodechar{⊨}{\ensuremath{\models}}
\newunicodechar{⦂}{\ensuremath{\mathbf{:}}}
\newunicodechar{ς}{\ensuremath{\varsigma}}
\newunicodechar{𝓓}{\ensuremath{\mathcal{D}}}
\newunicodechar{∎}{\ensuremath{\square}}
\newunicodechar{■}{\ensuremath{\blacksquare}}
\newunicodechar{↠}{\ensuremath{\twoheadrightarrow}}
\newunicodechar{↪}{\ensuremath{\hookrightarrow}}
\newunicodechar{ⁱ}{\ensuremath{^i}}
\newunicodechar{Π}{\ensuremath{\Pi}}
\newunicodechar{∋}{\ensuremath{\ni}}
\newunicodechar{∍}{\ensuremath{\ni}}
\newunicodechar{‵}{\ensuremath{`}}
\newunicodechar{⨆}{\ensuremath{\bigsqcup}}
\newunicodechar{δ}{\ensuremath{\delta}}
\newunicodechar{κ}{\ensuremath{\kappa}}
\newunicodechar{⦃}{\ensuremath{\{\!|}}  % \lBrace from stix
\newunicodechar{⦄}{\ensuremath{|\!\}}}  % \rBrace from stix
\newunicodechar{⌊}{\ensuremath{\lfloor}}
\newunicodechar{⌋}{\ensuremath{\rfloor}}
\newunicodechar{𝟎}{\ensuremath{\mathbf{0}}}
\newunicodechar{𝟏}{\ensuremath{\mathbf{1}}}
\newunicodechar{𝟐}{\ensuremath{\mathbf{2}}}
\newunicodechar{β}{\ensuremath{\beta}}
\newunicodechar{≥}{\ensuremath{\geq}}
\newunicodechar{ₒ}{\ensuremath{_o}}
\newunicodechar{∊}{\ensuremath{\in}}
\newunicodechar{;}{\ensuremath{;}}
\newunicodechar{⋯}{\ensuremath{\mkern2mu \cdotp \mkern-2mu \cdotp \mkern-2mu \cdotp \mkern1mu }}
\newunicodechar{⊢}{\ensuremath{\vdash}}
\newunicodechar{∶}{\ensuremath{:}}
\newunicodechar{★}{\ensuremath{\star}}
\newunicodechar{ᴷ}{\ensuremath{^K}}
\newunicodechar{ϕ}{\ensuremath{\varphi}}
\newunicodechar{ᴮ}{\ensuremath{^B}}
\newunicodechar{+}{}
\newunicodechar{ᵣ}{\ensuremath{_r}}
\newunicodechar{ₓ}{\ensuremath{_x}}
\newunicodechar{ₜ}{\ensuremath{_t}}
\newunicodechar{𝓡}{\ensuremath{\mathcal{R}}}
\newunicodechar{𝓢}{\ensuremath{\mathcal{S}}}
\newunicodechar{𝓚}{\ensuremath{\mathcal{K}}}
\newunicodechar{ᴹ}{\ensuremath{^M}}
\newunicodechar{ᵗ}{\ensuremath{^t}}
\newunicodechar{ξ}{\ensuremath{\xi}}
\newunicodechar{≐}{\ensuremath{\doteq}}
\newunicodechar{⸴}{\ensuremath{,}}
\newunicodechar{ᴱ}{\ensuremath{^E}}
\newunicodechar{⨟}{\ensuremath{\fatsemi}}
\newunicodechar{𝒜}{\ensuremath{\mathcal{A}}}
\newunicodechar{𝒮}{\ensuremath{\mathcal{S}}}
\newunicodechar{∪}{\ensuremath{\cup}}
\newunicodechar{∨}{\ensuremath{\lor}}

\begin{document}

\title{Every Way to \AgdaPragma{REWRITE} the $\sigma$-Calculus in Agda}

\author{Marius Weidner}
\email{weidner@cs.uni-freiburg.de}
\affiliation{%
  \institution{University of Freiburg}
  \country{Germany}}
\orcid{0009-0008-1152-165X} % chktex 8
\authornote{
  I hereby declare that I am the sole author and composer of my thesis and that
  no other sources or learning aids, other than those listed, have been used.
  Furthermore, I declare that I have acknowledged the work of others by providing
  detailed references of said work. \\
  I hereby also declare that my Thesis has not been prepared for another
  examination or assignment, either wholly or excerpts thereof.
  \\
  \\
  \begin{tabular}{p{\textwidth/2} l}
    Freiburg, XX.XX.2025       & yadda yadda                \\ % \includegraphics[width=0.1\textwidth]{signature.png} \\
    \rule{\textwidth/3}{0.4pt} & \rule{\textwidth/3}{0.4pt} \\
    Place, Date                & Signature
  \end{tabular}
}

\begin{abstract}
  yadda yadda
\end{abstract}

% \keywords{Agda, Dependent types, rewrite systems, explicit substiutions, sigma calculus, meta theory}

\maketitle

\section{Introduction}\label{sec:introduction}

% - generic handling of substitutions is an well known ongoing research problem.
% - 

\subsection*{Contributions}

\begin{itemize}
  \item \cref{sec:multi} introduces the core idea of embedding an extension of the sigma
        calculus into Agda using rewrite rules on the example of System F TBD
  \item \cref{sec:agdasubst} explains how to derive the rewrite rules for multi-sorted
        syntax automatically via abstractions for multi-sorted substitutions~\cite{saffrich:LIPIcs.ITP.2024.32}.
  \item In \cref{sec:intrinsic}, we investigate System F in the intrinsically typed
        setting. We observe that reasoning about substitutions at deeper levels of the
        hierarchy substantially simplifies, but also requires an intuitive
        extension of the rule set.
\end{itemize}

\textbf{The supplement contains the complete Agda code underlying this paper.}

\section{Preliminaries}\label{sec:preliminaries}

\begin{itemize}
  \item Some basics of Agda and its rewrite mechanism, which enables the embedding of
        custom $\beta$-reduction rules.
  \item Explicit substitutions, in particular the $\sigma$-calculus, a terminating and
        confluent rewrite system for terms with substitutions. We consider an
        incomplete extension with first-class renamings, better suited for use in
        theorem provers.
  \item Multi-Sorted syntax as a way of only have to define renaming and substitutions
        once for every sort
\end{itemize}

\subsection{Agda and Its Ability to \AgdaPragma{REWRITE}}

\subsection{Explicit Substiutions and the $\sigma$-Calculus}

\subsection{The Advantage of Mutli-Sorted Syntax}

\section{Reasoning with Mutli-Sorted Substiutions}\label{sec:multi}

\subsection{Discussion}\label{sec:discussion-1}

\section{Agdasubst: Reasoning and Abstraction over Mutli-Sorted Substiutions}\label{sec:agdasubst}

\subsection{Limitations}\label{sec:limitations}

\subsection{Discussion}\label{sec:discussion-2}

\section{The Intriguing Case of Intrinsically \emph{Typed} Syntax}\label{sec:intrinsic}

\subsection{Soundness of the Big Step Semantics with Respect to Native Agda Semantics}\label{sec:soundness}

\subsection{Discussion}\label{sec:discussion-3}

\section{Related Work}\label{sec:related-work}

\subsection{Autosubst}\label{sec:autosubst}

\citet{10.1145/3293880.3294101}

\subsection{Other Framkeworks}\label{sec:related-frameworks}

\section{Conclusions}\label{sec:conclusions}

\begin{acks}
  Thank you, Peter, Hannes, and Janek, for listening to me ramble about this for a few weeks at our weekly Agda meeting.
\end{acks}

\bibliographystyle{ACM-Reference-Format}
\bibliography{references}

\end{document}
\endinput
